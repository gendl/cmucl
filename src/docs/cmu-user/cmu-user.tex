%% cmu-user.tex --- CMUCL User's Manual
%%
%% 2001-04-05  Eric Marsden
%% Modifications to work with hevea and pdflatex. 
%%
%% Aug 1997   Raymond Toy
%% This is a modified version of the original CMUCL User's Manual.
%% The key changes are modification of this file to use standard
%% LaTeX2e.  This means latexinfo isn't going to work anymore.
%% However, Latex2html support has been added.
%%
%% Jan 1998 Paul Werkowski
%% A few of the packages below are not part of the standard LaTeX2e
%% distribution, and must be obtained from a repository. At this time
%% I was able to fetch from
%% ftp.cdrom.com:pub/tex/ctan/macros/latex/contrib/supported/
%%                     camel/index.ins
%%                     camel/index.dtx
%%                     calc/calc.ins
%%                     calc/calc.dtx
%%                     changebar/changebar.ins
%%                     changebar/changebar.dtx
%% One runs latex on the .ins file to produce .tex and/or .sty
%% files that must be put in a path searched by latex.
%%
%% Note all of the required packages are included in the teTeX distribution,
%% and a current version of latex2html can be obtained from:
%% http://saftsack.fs.uni-bayreuth.de/~latex2ht/

%% Delete "[a4paper]" if you don't want this formatted for A4 paper.
\documentclass[a4paper]{report}
\usepackage{xspace}
\usepackage{alltt}
\usepackage{index}
\usepackage{ifpdf}
\usepackage{ifthen}
\usepackage{calc}
\usepackage{sectsty}
\usepackage{varioref}
\usepackage[hyperindex=false,colorlinks=false,urlcolor=blue]{hyperref}
%% \usepackage{html}
\usepackage{typehtml}

% macro.tex
%
% LaTeX macros for CMUCL User's Manual
%
% by Raymond Toy

% use Palatino
\renewcommand{\rmdefault}{ppl}
\ifpdf
\usepackage{palatino}
\fi

%% Define the indices.  We need one for Types, Variables, Functions,
%% and a general concept index.
\makeindex
\newindex{types}{tdx}{tnd}{Type Index}
\newindex{vars}{vdx}{vnd}{Variable Index}
\newindex{funs}{fdx}{fnd}{Function Index}
\newindex{concept}{cdx}{cnd}{Concept Index}

\newcommand{\tindexed}[1]{\index[types]{#1}\code{#1}}
\newcommand{\findexed}[1]{\index[funs]{#1}\code{#1}}
\newcommand{\vindexed}[1]{\index[vars]{#1}\code{*#1*}}
\newcommand{\cindex}[1]{\index[concept]{#1}}
\newcommand{\cpsubindex}[2]{\index[concept]{#1!#2}}


%% This code taken from the LaTeX companion.  It's meant as a
%% replacement for the description environment.  We want one that
%% prints description items in a fixed size box and puts the
%% description itself on the same line or the next depending on the
%% size of the item.
\newcommand{\entrylabel}[1]{\mbox{#1}\hfil}
\newenvironment{entry}{%
  \begin{list}{}%
    {\renewcommand{\makelabel}{\entrylabel}%
      \setlength{\labelwidth}{45pt}%
      \setlength{\leftmargin}{\labelwidth+\labelsep}}}%
  {\end{list}}

\newlength{\Mylen}
\newcommand{\Lentrylabel}[1]{%
  \settowidth{\Mylen}{#1}%
  \ifthenelse{\lengthtest{\Mylen > \labelwidth}}%
  {\parbox[b]{\labelwidth}%  term > labelwidth
    {\makebox[0pt][l]{#1}\\}}%
  {#1}%
  \hfil\relax}
\newenvironment{Lentry}{%
  \renewcommand{\entrylabel}{\Lentrylabel}
  \begin{entry}}%
  {\end{entry}}

\newcommand{\fcntype}[1]{\textit{#1}}
\newcommand{\argtype}[1]{\textit{#1}}
\newcommand{\fcnname}[1]{\textsf{#1}}

\newlength{\formnamelen}        % length of a name of a form
\newlength{\pboxargslen}        % length of parbox for arguments
\newlength{\typelen}            % length of the type label for the form

\newcommand{\args}[1]{#1}
\newcommand{\keys}[1]{\code{\&key} \= #1}
\newcommand{\morekeys}[1]{\\ \> #1}
\newcommand{\yetmorekeys}[1]{\\ \> #1}

\newcommand{\defunvspace}{\ifhmode\unskip \par\fi\addvspace{18pt plus 12pt minus 6pt}}


%% \layout[pkg]{name}{param list}{type}
%%
%% This lays out a entry like so:
%%
%% pkg:name arg1 arg2                             [Function]
%%
%% where [Function] is flush right.
%%
\newcommand{\layout}[4][\mbox{}]{%
  \par\noindent
  \fcnname{#1#2\hspace{1em}}%
  \settowidth{\formnamelen}{\fcnname{#1#2\hspace{1em}}}%
  \settowidth{\typelen}{[\argtype{#4}]}%
  \setlength{\pboxargslen}{\linewidth}%
  \addtolength{\pboxargslen}{-1\formnamelen}%
  \addtolength{\pboxargslen}{-1\typelen}%
  \begin{minipage}[t]{\pboxargslen}
    \begin{tabbing}
      #3
    \end{tabbing}
  \end{minipage}
  \hfill[\fcntype{#4}]%
  \par\addvspace{2pt plus 2pt minus 2pt}}

\newcommand{\vrindexbold}[1]{\index[vars]{#1|textbf}}
\newcommand{\fnindexbold}[1]{\index[funs]{#1|textbf}}

%% Define a new type
%%
%% \begin{deftp}{typeclass}{typename}{args}
%%    some description
%% \end{deftp}
\newenvironment{deftp}[3]{%
  \par\bigskip\index[types]{#2|textbf}%
  \layout{#2}{\var{#3}}{#1}
  }{}

%% Define a function with name NAME and given parameters PARAM.  The
%% function is in the package PKG.  If the optional arg SUFFIX is
%% given, this is used as a suffix for the label.  (Useful when you
%% have functions of the same name, such as methods, but want
%% different labels for each version.)
%%
%% The defunx is for additional functions that are related to this one
%% in some way, and we want to group them all together.
%%  
%% \begin{defun}[suffix]{pkg}{name}{params}
%%   \defunx[pkg]{name}{params}
%%   description of function
%% \end{defun}
\newenvironment{defun}[4][]{%
  \par\defunvspace\fnindexbold{#3}\label{FN:#3#1}%
  \layout[#2]{#3}{#4}{Function}
  }{}
\newcommand{\defunx}[3][\mbox{}]{%
  \par\fnindexbold{#2}\label{FN:#2}%
  \layout[#1]{#2}{#3}{Function}}

%% Define a generic function.  Like defun, but for defgeneric.
%%
%% \begin{defgeneric}[suffix]{pkg}{name}{params}
%%   \defgenericx[pkg]{name}{params}
%%   description of function
%% \end{defgeneric}
\newenvironment{defgeneric}[4][]{%
  \par\defunvspace\fnindexbold{#3}\label{FN:#3-generic#1}%
  \layout[#2]{#3}{#4}{Generic Function}
  }{}
\newcommand{\defgenericx}[3][\mbox{}]{%
  \par\fnindexbold{#2}\label{FN:#2}%
  \layout[#1]{#2}{#3}{Generic Function}}

%% Define a method.  Like defgeneric, but for methods.
%%
%% \begin{defmethod}[suffix]{pkg}{name}{params}
%%   \defmethod[pkg]{name}{params}
%%   description of function
%% \end{defmethod}
\newenvironment{defmethod}[4][]{%
  \par\defunvspace\fnindexbold{#3}\label{FN:#3-method#1}%
  \layout[#2]{#3}{#4}{Method}
  }{}
\newcommand{\defmethodx}[3][\mbox{}]{%
  \par\fnindexbold{#2}\label{FN:#2}%
  \layout[#1]{#2}{#3}{Method}}

%% Define a macro
%%
%% \begin{defmac}[suffix]{pkg}{name}{params}
%%   \defmacx[pkg]{name}{params}
%%   description of macro
%% \end{defmac}
\newenvironment{defmac}[4][]{%
  \par\defunvspace\fnindexbold{#3}\label{FN:#3#1}%
  \layout[#2]{#3}{#4}{Macro}}{}
\newcommand{\defmacx}[3][\mbox{}]{%
  \par\fnindexbold{#2}\label{FN:#2}%
  \layout[#1]{#2}{#3}{Macro}}

%% Define a variable
%%
%% \begin{defvar}{pkg}{name}
%%   \defvarx[pkg]{name}
%%   description of defvar
%% \end{defvar}
\newenvironment{defvar}[2]{%
  \par\defunvspace\vrindexbold{#2}\label{VR:#2}
  \layout[#1]{*#2*}{}{Variable}}{}
\newcommand{\defvarx}[2][\mbox{}]{%
  \par\vrindexbold{#2}\label{VR:#2}
  \layout[#1]{*#2*}{}{Variable}}

%% Define a constant
%%
%% \begin{defconst}{pkg}{name}
%%   \ddefconstx[pkg]{name}
%%   description of defconst
%% \end{defconst}
\newcommand{\defconstx}[2][\mbox{}]{%
  \layout[#1]{#2}{}{Constant}}
\newenvironment{defconst}[2]{%
  \defunvspace\defconstx[#1]{#2}}{}

\newcommand{\credits}[1]{%
  \begin{center}
  \textbf{#1}
  \end{center}}

\newenvironment{example}{\begin{quote}\begin{alltt}}{\end{alltt}\end{quote}}
\newenvironment{lisp}{\begin{example}}{\end{example}}

\newcommand{\hide}[1]{}
\newcommand{\trnumber}[1]{#1}
\newcommand{\citationinfo}[1]{#1}
\newcommand{\var}[1]{{\textsf{\textsl{#1}}\xspace}}
\newcommand{\code}[1]{\textnormal{{\sffamily #1}}}
\newcommand{\file}[1]{`\texttt{#1}'}
\newcommand{\kwd}[1]{\code{:#1}}
\newcommand{\F}[1]{\code{#1}}
\newcommand{\w}[1]{\hbox{#1}}
\newcommand{\ctrl}[1]{$\uparrow$\textsf{#1}}
\newcommand{\result}{$\Rightarrow$}
\newcommand{\myequiv}{$\equiv$}
\newcommand{\back}[1]{\(\backslash\)#1}
\newcommand{\pxlref}[1]{see section~\ref{#1}, page~\pageref{#1}}
\newcommand{\xlref}[1]{See section~\ref{#1}, page~\pageref{#1}}
\newcommand{\funref}[1]{\findexed{#1} (page~\pageref{FN:#1})}
\newcommand{\specref}[1]{\findexed{#1} (page~\pageref{FN:#1})}
\newcommand{\macref}[1]{\findexed{#1} (page~\pageref{FN:#1})}
\newcommand{\varref}[1]{\vindexed{#1} (page~\pageref{VR:#1})}
\newcommand{\conref}[1]{\conindexed{#1} (page~\pageref{VR:#1})}

\newcommand{\false}{\code{nil}}
\newcommand{\true}{\code{t}}
\newcommand{\nil}{\false{}}
%% Printed lisp character #\foo
\newcommand{\lispchar}[1]{\code{\#\back{#1}}}

\newcommand{\ampoptional}{\code{\&optional}}
\newcommand{\amprest}{\code{\&rest}}
\newcommand{\ampbody}{\code{\&body}}

\newcommand{\mopt}[1]{{$\,\{$}\textnormal{\textsf{\textsl{#1\/}}}{$\}\,$}}
\newcommand{\mstar}[1]{{$\,\{$}\textnormal{\textsf{\textsl{#1\/}}}{$\}^*\,$}}
\newcommand{\mplus}[1]{{$\,\{$}\textnormal{\textsf{\textsl{#1\/}}}{$\}^+\,$}}
\newcommand{\mgroup}[1]{{$\,\{$}\textnormal{\textsf{\textsl{#1\/}}}{$\}\,$}}
\newcommand{\mor}{$|$}


%% Some common abbreviations
\newcommand{\dash}{---}
\newcommand{\alien}{Alien}
\newcommand{\aliens}{Aliens}
\newcommand{\hemlock}{Hemlock}
\newcommand{\python}{Python}
\newcommand{\cmucl}{\textsc{cmucl}}
\newcommand{\clisp}{Common Lisp}
\newcommand{\llisp}{Common Lisp}
\newcommand{\cltl}{\textit{Common Lisp: The Language}}
\newcommand{\cltltwo}{\textit{Common Lisp: The Language II}}


%% Set up margins
\setlength{\oddsidemargin}{-10pt}
\setlength{\evensidemargin}{-10pt}
\setlength{\topmargin}{-40pt}
\setlength{\headheight}{12pt}
\setlength{\headsep}{25pt}
\setlength{\footskip}{30pt}
\setlength{\textheight}{9.25in}
\setlength{\textwidth}{6.75in}
\setlength{\columnsep}{0.375in}
\setlength{\columnseprule}{0pt}


\setcounter{tocdepth}{2}
\setcounter{secnumdepth}{3}
\def\textfraction{.1}
\def\bottomfraction{.9}         % was .3
\def\topfraction{.9}

%% Allow TeX some stretching space to avoid overfull and underfull
%% boxes.
\setlength{\emergencystretch}{5pt}

%% requires the sectsty package
\allsectionsfont{\bfseries\sffamily}
\chapterfont{\fontfamily{pag}\selectfont}

%% section numbers in the left margin
\makeatletter
\def\@seccntformat#1{\protect\makebox[0pt][r]{\csname
    the#1\endcsname\quad}}
\makeatother



\title{CMUCL User's Manual}
\author{Robert A. MacLachlan, \textit{Editor}}
\newcommand{\keywords}{lisp, Common Lisp, manual, compiler, programming
language implementation, programming environment}

\date{October 2017 \\ 21c}


\begin{document}

\begin{titlepage}
  \makeatletter
  \vspace{60pt}
  \begin{center}
    \rule{\linewidth}{0.7mm}
    \vspace{3em}
    {\Huge \@title \par}
    \vspace{4em}
     {\large
       \begin{tabular}[t]{c}
         \@author
       \end{tabular}\par}
       \vspace{2em}
     {\large \@date \par}
     \vspace{2em}
    \rule{\linewidth}{0.7mm}
  \end{center}
  \vfill

  \begin{quotation}
    \cmucl{} is a free, high-performance implementation of the Common Lisp
    programming language, which runs on most major Unix platforms. It
    mainly conforms to the ANSI Common Lisp Standard. \cmucl{} features a
    sophisticated native-code compiler, a foreign function interface, a
    graphical source-level debugger, an interface to the X11 Window
    System, and an Emacs-like editor.

    \medskip \textbf{Keywords}: \keywords
  \end{quotation}

  \vspace{5cm}
  
  This manual is based on CMU Technical Report CMU-CS-92-161, edited by
  Robert A. MacLachlan, dated July 1992.

  \thispagestyle{empty}
  \makeatother
\end{titlepage}

\ifpdf
\pdfinfo{
/Author (Robert A. MacLachlan, ed)
/Title (CMUCL User's Manual)
/Keywords (\keywords)
}
% Add section numbers to the bookmarks, and open 2 levels by default.
\hypersetup{bookmarksnumbered=true,
            bookmarksopen=true,
            bookmarksopenlevel=2}
\fi

% \maketitle

\pagestyle{headings}
\pagenumbering{roman}
\tableofcontents

\clearpage
\pagenumbering{arabic}

\chapter{Introduction}

\cmucl{} is a free, high-performance implementation of the Common Lisp
programming language which runs on most major Unix platforms. It
mainly conforms to the ANSI Common Lisp standard. Here is a summary of
its main features:

\begin{itemize}
\item a {\em sophisticated native-code compiler} which is capable of
powerful type inferences, and generates code competitive in speed with
C compilers.

\item generational garbage collection and multiprocessing
capability on the x86 ports.

\item a foreign function interface which allows interfacing with C code and
system libraries, including shared libraries on most platforms, and
direct access to Unix system calls.

\item support for interprocess communication and remote procedure
calls.
     
\item an implementation of CLOS, the Common Lisp Object System, which
includes multimethods and a metaobject protocol.

\item a graphical source-level debugger using a Motif interface, and a
code profiler.

\item an interface to the X11 Window System (CLX), and a sophisticated
graphical widget library (Garnet).

\item programmer-extensible input and output streams.
                        
\item an Emacs-like editor implemented in Common Lisp.

\item public domain: free, with full source code and no
strings attached (and no warranty).  Like GNU/Linux and the *BSD
operating systems, \cmucl{} is maintained and improved by a team of
volunteers collaborating over the Internet.
\end{itemize}


This user's manual contains only implementation-specific information
about \cmucl. Users will also need a separate manual describing the
\clisp{} standard, for example, the
\ifpdf
\href{http://www.lispworks.com/documentation/HyperSpec/Front/index.htm}
{Hyperspec}.
\else
\emph{Hyperspec} at \url{http://www.lispworks.com/documentation/HyperSpec/Front/index.htm}
\fi


In addition to the language itself, this document describes a number
of useful library modules that run in \cmucl. \hemlock, an Emacs-like
text editor, is included as an integral part of the \cmucl{}
environment. Two documents describe \hemlock{}: the {\it Hemlock
User's Manual}, and the {\it Hemlock Command Implementor's Manual}.


\section{Distribution and Support}

\cmucl{} is developed and maintained by a group of volunteers who
collaborate over the internet. Sources and binary releases for the
various supported platforms can be obtained from
\href{http://www.cons.org/cmucl/}{www.cons.org/cmucl}. These pages
describe how to download by FTP or CVS.

A number of mailing lists are available for users and developers;
please see the web site for more information. 


\section{Command Line Options}
\cindex{command line options}
\label{command-line-options}

The command line syntax and environment is described in the
\verb|lisp(1)| man page in the man/man1 directory of the distribution.
See also \verb|cmucl(1)|. Currently \cmucl{} accepts the following
switches:

\begin{Lentry}
\item[\code{--help}] Same as \code{-help}.

\item[\code{-help}] Print ou the command line options and exit.
  
\item[\code{-batch}] specifies batch mode, where all input is
  directed from standard-input.  An error code of 0 is returned upon
  encountering an EOF and 1 otherwise.

\item[\code{-quiet}] enters quiet mode. This implies setting the
  variables \code{*load-verbose*}, \code{*compile-verbose*},
  \code{*compile-print*}, \code{*compile-progress*},
  \code{*require-verbose*} and \code{*gc-verbose*} to NIL, and
  disables the printing of the startup banner.

\item[\code{-core}] requires an argument that should be the name of a
  core file.  Rather than using the default core file, which is searched
  in a number of places, according to the initial value of the
  \code{library:} search-list, the specified core file is loaded.  This
  switch overrides the value of the \code{CMUCLCORE} environment variable,
  if present.
  
\item[\code{-lib}] requires an argument that should be the path to the
  CMUCL library directory, which is going to be used to initialize the
  \code{library:} search-list, among other things.  This switch overrides
  the value of the \code{CMUCLLIB} environment variable, if present.

\item[\code{-dynamic-space-size}] requires an argument that should be
  the number of megabytes (1048576 bytes) that should be allocated to
  the heap.  If not specified, a platform-specific default is used.
  The actual maximum allowed heap size is platform-specific.

  Currently, this option is only available for the x86 and sparc
  platforms. 

\item[\code{-edit}] specifies to enter Hemlock.  A file to edit may be
  specified by placing the name of the file between the program name
  (usually \file{lisp}) and the first switch.
  
\item[\code{-eval}] accepts one argument which should be a Lisp form
  to evaluate during the start up sequence.  The value of the form
  will not be printed unless it is wrapped in a form that does output.
  
\item[\code{-hinit}] accepts an argument that should be the name of
  the hemlock init file to load the first time the function
  \findexed{ed} is invoked.  The default is to load
  \file{hemlock-init.\var{object-type}}, or if that does not exist,
  \file{hemlock-init.lisp} from the user's home directory.  If the
  file is not in the user's home directory, the full path must be
  specified.
  
\item[\code{-init}] accepts an argument that should be the name of an
  init file to load during the normal start up sequence.  The default
  is to load \file{init.\var{object-type}} or, if that does not exist,
  \file{init.lisp} from the user's home directory.  If neither exists,
  \cmucl tries \file{.cmucl-init.\var{object-type}} and then
  \file{.cmucl-init.lisp}.  If the file is not
  in the user's home directory, the full path must be specified.  If
  the file does not exist, \cmucl silently ignores it.
  
\item[\code{-noinit}] accepts no arguments and specifies that an init
  file should not be loaded during the normal start up sequence.
  Also, this switch suppresses the loading of a hemlock init file when
  Hemlock is started up with the \code{-edit} switch.

\item[\code{-nositeinit}] accepts no arguments and specifies that the
  site init file should not be loaded during the normal start up
  sequence. 

\item[\code{-load}] accepts an argument which should be the name of a
  file to load into Lisp before entering Lisp's read-eval-print loop.
  
\item[\code{-slave}] specifies that Lisp should start up as a
  \i{slave} Lisp and try to connect to an editor Lisp.  The name of
  the editor to connect to must be specified\dash{}to find the
  editor's name, use the \hemlock{} ``\code{Accept Slave
    Connections}'' command.  The name for the editor Lisp is of the
  form:
  \begin{example}
    \var{machine-name}\code{:}\var{socket}
  \end{example}
  where \var{machine-name} is the internet host name for the machine
  and \var{socket} is the decimal number of the socket to connect to.

\item[\code{-fpu}] specifies what fpu should be used for x87 machines.
  The possible values are ``\code{x87}'', ``\code{sse2}'', or
  ``\code{auto}'', which is the default.  By default, \cmucl will
  detect if the chip supports the SSE2 instruction set or not.  If so
  or if \code{-fpu sse2} is specified, the SSE2 core will be loaded
  that uses SSE2 for floating-point arithmetic.  If SSE2 is not
  available or if \code{-fpu x87} is given, the legacy x87 core is
  loaded.

\item[\code{--}] indicates that everything after ``\code{--}'' is not
  subject to \cmucl's command line parsing.  Everything after
  ``\code{--}'' is placed in the variable
  \code{ext:*command-line-application-arguments*}.
\end{Lentry}

For more details on the use of the \code{-edit} and \code{-slave}
switches, see the {\it Hemlock User's Manual}.

Arguments to the above switches can be specified in one of two ways:
\w{\var{switch}\code{=}\var{value}} or
\w{\var{switch}<\var{space}>\var{value}}.  For example, to start up
the saved core file mylisp.core use either of the following two
commands:

\begin{example}
   lisp -core=mylisp.core
   lisp -core mylisp.core
\end{example}


\section{Credits}

\cmucl{} was developed at the Computer Science Department of Carnegie
Mellon University. The work was a small autonomous part within the
Mach microkernel-based operating system project, and started more as a
tool development effort than a research project. The project started
out as Spice Lisp, which provided a modern Lisp implementation for use
in the CMU community. \cmucl{} has been under continual development since
the early 1980's (concurrent with the \clisp{} standardization
effort). Most of the CMU Common Lisp implementors went on to work on
the Gwydion environment for Dylan. The CMU team was lead by Scott E.
Fahlman, the \python{} compiler was written by Robert MacLachlan.

\cmucl{}'s CLOS implementation is derived from the PCL reference
implementation written at Xerox PARC:
\begin{quotation}
\noindent Copyright (c) 1985, 1986, 1987, 1988, 1989, 1990 Xerox
Corporation.\\
All rights reserved.

\vspace{1ex}
\noindent Use and copying of this software and preparation of
derivative works based upon this software are permitted.  Any
distribution of this software or derivative works must comply with all
applicable United States export control laws.

\vspace{1ex}
\noindent This software is made available AS IS, and Xerox Corporation
makes no warranty about the software, its performance or its
conformity to any specification.
\end{quotation}
Its implementation of the LOOP macro was derived from code from
Symbolics, which was derived from code written at MIT:
\begin{quotation}
\noindent Portions of LOOP are Copyright (c) 1986 by the Massachusetts
Institute of Technology.\\
All Rights Reserved.

\vspace{1ex}
\noindent Permission to use, copy, modify and distribute this software
and its documentation for any purpose and without fee is hereby granted,
provided that the M.I.T. copyright notice appear in all copies and that
both that copyright notice and this permission notice appear in
supporting documentation.  The names "M.I.T." and "Massachusetts
Institute of Technology" may not be used in advertising or publicity
pertaining to distribution of the software without specific, written
prior permission.  Notice must be given in supporting documentation that
copying distribution is by permission of M.I.T.  M.I.T. makes no
representations about the suitability of this software for any purpose.
It is provided "as is" without express or implied warranty.

\vspace{3ex}
\noindent Portions of LOOP are Copyright (c) 1989, 1990, 1991, 1992 by
Symbolics, Inc.\\
All Rights Reserved.

\vspace{1ex}
\noindent Permission to use, copy, modify and distribute this software
and its documentation for any purpose and without fee is hereby
granted, provided that the Symbolics copyright notice appear in all
copies and that both that copyright notice and this permission notice
appear in supporting documentation.  The name "Symbolics" may not be
used in advertising or publicity pertaining to distribution of the
software without specific, written prior permission.  Notice must be
given in supporting documentation that copying distribution is by
permission of Symbolics.  Symbolics makes no representations about the
suitability of this software for any purpose.  It is provided "as is"
without express or implied warranty.

\vspace{1ex}
\noindent Symbolics, CLOE Runtime, and Minima are trademarks, and
CLOE, Genera, and Zetalisp are registered trademarks of Symbolics,
Inc.
\end{quotation}
The CLX code is copyrighted by Texas Instruments Incorporated:
\begin{quotation}
\noindent Copyright (C) 1987 Texas Instruments Incorporated.

\vspace{1ex}
\noindent Permission is granted to any individual or institution to
use, copy, modify, and distribute this software, provided that this
complete copyright and permission notice is maintained, intact, in all
copies and supporting documentation.

\vspace{1ex}
\noindent Texas Instruments Incorporated provides this software "as
is" without express or implied warranty.
\end{quotation}

\cmucl{} was funded by DARPA under CMU's "Research on Parallel Computing"
contract. Rather than doing pure research on programming languages and
environments, the emphasis was on developing practical programming
tools. Sometimes this required new technology, but much of the work
was in creating a \clisp{} environment that incorporates
state-of-the-art features from existing systems (both Lisp and
non-Lisp). Archives of the project are available online.

The project funding stopped in 1994, so support at Carnegie Mellon
University has been discontinued. All code and documentation developed
at CMU was released into the public domain. The project continues as a
group of users and developers collaborating over the Internet. The
current and previous maintainers include:

\begin{itemize}
\item Marco Antoniotti
\item Martin Cracauer
\item Fred Gilham
\item Alex Goncharov
\item Rob MacLachlan
\item Pierre Mai
\item Eric Marsden
\item Gerd Moellman
\item Tim Moore
\item Carl Shapiro  
\item Robert Swindells
\item Raymond Toy
\item Peter Van Eynde
\item Paul Werkowski
\end{itemize}

In particular, Paul Werkowski and Douglas Crosher completed the port
for the x86 architecture for FreeBSD. Peter VanEnyde took the FreeBSD
port and created a Linux version. Other people who have contributed to
the development of \cmucl{} since 1981 are

\begin{itemize}
\item David Axmark
\item Miles Bader
\item Rick Busdiecker
\item Bill Chiles
\item Douglas Thomas Crosher
\item Casper Dik
\item Ted Dunning
\item Scott Fahlman
\item Mike Garland
\item Paul Gleichauf
\item Sean Hallgren
\item Richard Harris
\item Joerg-Cyril Hoehl
\item Chris Hoover
\item John Kolojejchick
\item Todd Kaufmann
\item Simon Leinen
\item Sandra Loosemore
\item William Lott
\item Dave McDonald
\item Tim Moore
\item Skef Wholey
\item Paul Foley
\item Helmut Eller
\item Jan Rychter
\end{itemize}

Countless others have contributed to the project by sending in bug
reports, bug fixes, and new features.

This manual is based on CMU Technical Report CMU-CS-92-161, edited by
Robert A. MacLachlan, dated July 1992. Other contributors include
Raymond Toy, Paul Werkowski and Eric Marsden. The Hierarchical
Packages chapter is based on documentation written by Franz. Inc, and
is used with permission. The remainder of the document is in the
public domain.

\chapter{Design Choices and Extensions}

Several design choices in \clisp{} are left to the individual
implementation, and some essential parts of the programming environment
are left undefined.  This chapter discusses the most important design
choices and extensions.

\section{Data Types}

\subsection{Integers}

The \tindexed{fixnum} type is equivalent to \code{(signed-byte 30)}.
Integers outside this range are represented as a \tindexed{bignum} or
a word integer (\pxlref{word-integers}.)  Almost all integers that
appear in programs can be represented as a \code{fixnum}, so integer
number consing is rare.


\subsection{Floats}
\label{ieee-float}

\cmucl{} supports three floating point formats:
\tindexed{single-float}, \tindexed{double-float} and
\tindexed{double-double-float}.  The first two are implemented with
IEEE single and double float arithmetic, respectively.  The last is an
extension; \pxlref{extended-float} for more information.
\code{short-float} is a synonym for \code{single-float}, and
\code{long-float} is a synonym for \code{double-float}.  The initial
value of \vindexed{read-default-float-format} is \code{single-float}.

Both \code{single-float} and \code{double-float} are represented with
a pointer descriptor, so float operations can cause number consing.
Number consing is greatly reduced if programs are written to allow the
use of non-descriptor representations (\pxlref{numeric-types}.)


\subsubsection{IEEE Special Values}

\cmucl{} supports the IEEE infinity and NaN special values.  These
non-numeric values will only be generated when trapping is disabled
for some floating point exception (\pxlref{float-traps}), so users of
the default configuration need not concern themselves with special
values.

\begin{defconst}{extensions:}{short-float-positive-infinity}
  \defconstx[extensions:]{short-float-negative-infinity}
  \defconstx[extensions:]{single-float-positive-infinity}
  \defconstx[extensions:]{single-float-negative-infinity}
  \defconstx[extensions:]{double-float-positive-infinity}
  \defconstx[extensions:]{double-float-negative-infinity}
  \defconstx[extensions:]{long-float-positive-infinity}
  \defconstx[extensions:]{long-float-negative-infinity}
  
  The values of these constants are the IEEE positive and negative
  infinity objects for each float format.
\end{defconst}

\begin{defun}{extensions:}{float-infinity-p}{\args{\var{x}}}
  
  This function returns true if \var{x} is an IEEE float infinity (of
  either sign.)  \var{x} must be a float.
\end{defun}

\begin{defun}{extensions:}{float-nan-p}{\args{\var{x}}}
  \defunx[extensions:]{float-signaling-nan-p}{\args{\var{x}}}
  \defunx[extensions:]{float-trapping-nan-p}{\args{\var{x}}}
  
  \code{float-nan-p} returns true if \var{x} is an IEEE NaN (Not A
  Number) object.  \code{float-signaling-nan-p} returns true only if
  \var{x} is a trapping NaN.  With either function, \var{x} must be a
  float. \code{float-trapping-nan-p} is the former name of
  \code{float-signaling-nan-p} and is deprecated.
\end{defun}

\subsubsection{Negative Zero}

The IEEE float format provides for distinct positive and negative
zeros.  To test the sign on zero (or any other float), use the
\clisp{} \findexed{float-sign} function.  Negative zero prints as
\code{-0.0f0} or \code{-0.0d0}.

\subsubsection{Denormalized Floats}

\cmucl{} supports IEEE denormalized floats.  Denormalized floats
provide a mechanism for gradual underflow.  The \clisp{}
\findexed{float-precision} function returns the actual precision of a
denormalized float, which will be less than \findexed{float-digits}.
Note that in order to generate (or even print) denormalized floats,
trapping must be disabled for the underflow exception
(\pxlref{float-traps}.)  The \clisp{}
\w{\code{least-positive-}\var{format}-\code{float}} constants are
denormalized.

\begin{defun}{extensions:}{float-denormalized-p}{\args{\var{x}}}
  
  This function returns true if \var{x} is a denormalized float.
  \var{x} must be a float.
\end{defun}


\subsubsection{Floating Point Exceptions}
\label{float-traps}

The IEEE floating point standard defines several exceptions that occur
when the result of a floating point operation is unclear or
undesirable.  Exceptions can be ignored, in which case some default
action is taken, such as returning a special value.  When trapping is
enabled for an exception, a error is signalled whenever that exception
occurs.  These are the possible floating point exceptions:
\begin{Lentry}
  
\item[\kwd{underflow}] This exception occurs when the result of an
  operation is too small to be represented as a normalized float in
  its format.  If trapping is enabled, the
  \tindexed{floating-point-underflow} condition is signalled.
  Otherwise, the operation results in a denormalized float or zero.
  
\item[\kwd{overflow}] This exception occurs when the result of an
  operation is too large to be represented as a float in its format.
  If trapping is enabled, the \tindexed{floating-point-overflow}
  exception is signalled.  Otherwise, the operation results in the
  appropriate infinity.
  
\item[\kwd{inexact}] This exception occurs when the result of a
  floating point operation is not exact, i.e. the result was rounded.
  If trapping is enabled, the \code{extensions:floating-point-inexact}
  condition is signalled.  Otherwise, the rounded result is returned.
  
\item[\kwd{invalid}] This exception occurs when the result of an
  operation is ill-defined, such as \code{\w{(/ 0.0 0.0)}}.  If
  trapping is enabled, the \code{extensions:floating-point-invalid}
  condition is signalled.  Otherwise, a quiet NaN is returned.
  
\item[\kwd{divide-by-zero}] This exception occurs when a float is
  divided by zero.  If trapping is enabled, the
  \tindexed{divide-by-zero} condition is signalled.  Otherwise, the
  appropriate infinity is returned.
\end{Lentry}

\subsubsection{Floating Point Rounding Mode}
\label{float-rounding-modes}

IEEE floating point specifies four possible rounding modes:
\begin{Lentry}
  
\item[\kwd{nearest}] In this mode, the inexact results are rounded to
  the nearer of the two possible result values.  If the neither
  possibility is nearer, then the even alternative is chosen.  This
  form of rounding is also called ``round to even'', and is the form
  of rounding specified for the \clisp{} \findexed{round} function.
  
\item[\kwd{positive-infinity}] This mode rounds inexact results to the
  possible value closer to positive infinity.  This is analogous to
  the \clisp{} \findexed{ceiling} function.
  
\item[\kwd{negative-infinity}] This mode rounds inexact results to the
  possible value closer to negative infinity.  This is analogous to
  the \clisp{} \findexed{floor} function.
  
\item[\kwd{zero}] This mode rounds inexact results to the possible
  value closer to zero.  This is analogous to the \clisp{}
  \findexed{truncate} function.
\end{Lentry}

Warning: Although the rounding mode can be changed with
\code{set-floating-point-modes}, use of any value other than the
default (\kwd{nearest}) can cause unusual behavior, since it will
affect rounding done by \llisp{} system code as well as rounding in
user code.  In particular, the unary \code{round} function will stop
doing round-to-nearest on floats, and instead do the selected form of
rounding.

%% \subsubsection{Precision Control}
%% \label{precision-control}
%% 
%% The floating-point unit for the Intel IA-32 architecture supports a
%% precision control mechanism.  The floating-point unit consists of an
%% IEEE extended double-float unit and all operations are always done
%% using his format, and this includes rounding.  However, by setting the
%% precision control mode, the user can control how rounding is done for
%% each basic arithmetic operation like addition, subtraction,
%% multiplication, and division.  The extra instructions for
%% trigonometric, exponential, and logarithmic operations are not
%% affected.  We refer the reader to Intel documentation for more
%% information. 
%% 
%% The possible modes are:
%% \begin{Lentry}
%%   
%% \item[\kwd{24-bit}] In this mode, all basic arithmetic operations like
%%   addition, subtraction, multiplication, and division, are rounded
%%   after each operation as if both the operands were IEEE single
%%   precision numbers.  
%%   
%% \item[\kwd{53-bit}] In this mode, rounding is performed as if the
%%   operands and results were IEEE double precision numbers.
%%   
%% \item[\kwd{64-bit}] In this mode, the default, rounding is performed
%%   on the full IEEE extended double precision format.
%%   
%% \end{Lentry}
%% 
%% \subsubsection{Warning:}
%% 
%% Although the precision mode can be changed with
%% \code{set-floating-point-modes}, use of anything other than
%% \kwd{64-bit} or \kwd{53-bit} can cause unexpected results, especially
%% if external functions or libraries are called.  A setting of
%% \kwd{64-bit} also causes \code{(= 1d0 (+ 1d0 double-float-epsilon))}
%% to return \true{} instead of \false.
%% 
%% 
\subsubsection{Accessing the Floating Point Modes}

These functions can be used to modify or read the floating point modes:

\begin{defun}{extensions:}{set-floating-point-modes}{%
    \keys{\kwd{traps} \kwd{rounding-mode}}
    \morekeys{\kwd{fast-mode} \kwd{accrued-exceptions}}
    \yetmorekeys{\kwd{current-exceptions}}}
  \defunx[extensions:]{get-floating-point-modes}{}
  
  The keyword arguments to \code{set-floating-point-modes} set various
  modes controlling how floating point arithmetic is done:
  \begin{Lentry}
  
  \item[\kwd{traps}] A list of the exception conditions that should
    cause traps.  Possible exceptions are \kwd{underflow},
    \kwd{overflow}, \kwd{inexact}, \kwd{invalid} and
    \kwd{divide-by-zero}.  Initially all traps except \kwd{inexact}
    are enabled.  \xlref{float-traps}.
    
  \item[\kwd{rounding-mode}] The rounding mode to use when the result
    is not exact. Possible values are \kwd{nearest},
    \kwd{positive-infinity}, \kwd{negative-infinity} and \kwd{zero}.
    Initially, the rounding mode is \kwd{nearest}. See the warning in
    section \ref{float-rounding-modes} about use of other rounding
    modes.
  
  \item[\kwd{current-exceptions}, \kwd{accrued-exceptions}] Lists of
    exception keywords used to set the exception flags.  The
    \var{current-exceptions} are the exceptions for the previous
    operation, so setting it is not very useful.  The
    \var{accrued-exceptions} are a cumulative record of the exceptions
    that occurred since the last time these flags were cleared.
    Specifying \code{()} will clear any accrued exceptions.
  
  \item[\kwd{fast-mode}] Set the hardware's ``fast mode'' flag, if
    any.  When set, IEEE conformance or debuggability may be impaired.
    Some machines may not have this feature, in which case the value
    is always \false.  Sparc platforms support a fast mode where
    denormal numbers are silently truncated to zero.
  \end{Lentry}
  If a keyword argument is not supplied, then the associated state is
  not changed.
  
  \code{get-floating-point-modes} returns a list representing the
  state of the floating point modes.  The list is in the same format
  as the keyword arguments to \code{set-floating-point-modes}, so
  \code{apply} could be used with \code{set-floating-point-modes} to
  restore the modes in effect at the time of the call to
  \code{get-floating-point-modes}.
\end{defun}

To make handling control of floating-point exceptions, the following
macro is useful.

\begin{defmac}{ext:}{with-float-traps-masked}{\var{traps} \ampbody\ \var{body}}
  \code{body} is executed with the selected floating-point exceptions
  given by \code{traps} masked out (disabled).  \code{traps} should be
  a list of possible floating-point exceptions that should be ignored.
  Possible values are \kwd{underflow}, \kwd{overflow}, \kwd{inexact},
  \kwd{invalid} and \kwd{divide-by-zero}.
  
  This is equivalent to saving the current traps from
  \code{get-floating-point-modes}, setting the floating-point modes to
  the desired exceptions, running the \code{body}, and restoring the
  saved floating-point modes.  The advantage of this macro is that it
  causes less consing to occur.

  Some points about the with-float-traps-masked:

  \begin{itemize}
  \item Two approaches are available for detecting FP exceptions:
    \begin{enumerate}
    \item enabling the traps and handling the exceptions
    \item disabling the traps and either handling the return values or
      checking the accrued exceptions.
    \end{enumerate}
    Of these the latter is the most portable because on the alpha port
    it is not possible to enable some traps at run-time.
    
  \item To assist the checking of the exceptions within the body any
    accrued exceptions matching the given traps are cleared at the
    start of the body when the traps are masked.
    
  \item To allow the macros to be nested these accrued exceptions are
    restored at the end of the body to their values at the start of
    the body. Thus any exceptions that occurred within the body will
    not affect the accrued exceptions outside the macro.
    
  \item Note that only the given exceptions are restored at the end of
    the body so other exception will be visible in the accrued
    exceptions outside the body.
    
  \item On the x86, setting the accrued exceptions of an unmasked
    exception would cause a FP trap. The macro behaviour of restoring
    the accrued exceptions ensures than if an accrued exception is
    initially not flagged and occurs within the body it will be
    restored/cleared at the exit of the body and thus not cause a
    trap.
    
  \item On the x86, and, perhaps, the hppa, the FP exceptions may be
    delivered at the next FP instruction which requires a FP
    \code{wait} instruction (\code{x86::float-wait}) if using the lisp
    conditions to catch trap within a \code{handler-bind}.  The
    \code{handler-bind} macro does the right thing and inserts a
    float-wait (at the end of its body on the x86).  The masking and
    noting of exceptions is also safe here.
    
  \item The setting of the FP flags uses the
    \code{(floating-point-modes)} and the \code{(set
      (floating-point-modes)\ldots)} VOPs. These VOPs blindly update
    the flags which may include other state.  We assume this state
    hasn't changed in between getting and setting the state. For
    example, if you used the FP unit between the above calls, the
    state may be incorrectly restored! The
    \code{with-float-traps-masked} macro keeps the intervening code to
    a minimum and uses only integer operations.
    %% Safe byte-compiled?
    %% Perhaps the VOPs (x86) should be smarter and only update some of
    %% the flags, the trap masks and exceptions?
  \end{itemize}

\end{defmac}

\subsection{Extended Floats}
\label{extended-float}

\cmucl{} also has an extension to support \code{double-double-float}
type.  This float format provides extended precision of about 31
decimal digits, with the same exponent range as \code{double-float}.
It is completely integrated into \cmucl{}, and can be used just like
any other floating-point object, including arrays, complex
\code{double-double-float}'s, and special functions.  With appropriate
declarations, no boxing is needed, just like \code{single-float} and
\code{double-float}. 

The exponent marker for a double-double float number is ``W'', so
``1.234w0'' is a double-double float number.


Note that there are a few shortcomings with
\code{double-double-float}'s:
\begin{itemize}
 \item There are no equivalents to \code{most-positive-double-float},
   \code{double-float-positive-infinity}, \textit{etc}.  This is because
   these are not really well defined for \code{double-double-float}'s.
 \item Underflow and overflow may be prematurely signaled.  This is
   due to how \code{double-double-float}'s are implemented.
 \item Basic arithmetic operations are inlined, so the code size is
   fairly large.
 \item \code{double-double-float} arithmetic is quite a bit slower
   than \code{double-float} since there is no hardware support for
   this type.
 \item The constant \code{pi} is still a \code{double-float} instead
   of a \code{double-double-float}.  Use \code{ext:dd-pi} if you
   want a \code{double-double-float} value for $\pi$.
\end{itemize}

\begin{deftp}{float}{extensions:double-double-float}{}
  The \code{double-double-float} type.  It is in the \code{EXTENSIONS}
  package.
\end{deftp}

\begin{defconst}{extensions:}{dd-pi}
  A \code{double-double-float} approximation to $\pi$.
\end{defconst}

\subsection{Characters}

\cmucl{} implements characters according to \cltltwo{}. The
main difference from the first version is that character bits and font
have been eliminated, and the names of the types have been changed.
\tindexed{base-character} is the new equivalent of the old
\tindexed{string-char}. In this implementation, all characters are
base characters (there are no extended characters.) Character codes
range between \code{0} and \code{255}, using the ASCII encoding.
Table~\ref{tbl:chars}~\vpageref{tbl:chars} shows characters recognized
by \cmucl.

\begin{table}[tbhp]
  \begin{center}
    \begin{tabular}{|c|c|l|l|l|l|}
      \hline
      \multicolumn{2}{|c|}{ASCII} & \multicolumn{1}{|c}{Lisp} &
      \multicolumn{3}{|c|}{} \\
      \cline{1-2}
      Name & Code & \multicolumn{1}{|c|}{Name} & \multicolumn{3}{|c|}{\raisebox{1.5ex}{Alternatives}}\\
      \hline
      \hline
      \code{nul} & 0 & \code{\#\back{NULL}} & \code{\#\back{NUL}} & &\\
      \code{bel} & 7 & \code{\#\back{BELL}} & & &\\
      \code{bs} &  8 & \code{\#\back{BACKSPACE}} & \code{\#\back{BS}} & &\\
      \code{tab} & 9 & \code{\#\back{TAB}} & & &\\
      \code{lf} & 10 & \code{\#\back{NEWLINE}} & \code{\#\back{NL}} & \code{\#\back{LINEFEED}} & \code{\#\back{LF}}\\
      \code{ff} & 11 & \code{\#\back{VT}} & \code{\#\back{PAGE}} & \code{\#\back{FORM}} &\\
      \code{cr} & 13 & \code{\#\back{RETURN}} & \code{\#\back{CR}} & &\\
      \code{esc} & 27 & \code{\#\back{ESCAPE}} & \code{\#\back{ESC}} & \code{\#\back{ALTMODE}} & \code{\#\back{ALT}}\\
      \code{sp} & 32 & \code{\#\back{SPACE}} & \code{\#\back{SP}} & &\\
      \code{del} & 127 & \code{\#\back{DELETE}} & \code{\#\back{RUBOUT}} & &\\
      \hline
    \end{tabular}
    \caption{Characters recognized by \cmucl}
    \label{tbl:chars}
  \end{center}
\end{table}


\subsection{Array Initialization}

If no \kwd{initial-value} is specified, arrays are initialized to zero.


\subsection{Hash tables}

The \tindexed{hash-tables} defined by \clisp{} have limited utility because they
are limited to testing their keys using the equality predicates
provided by (pre-CLOS) \clisp{}.  \cmucl{} overcomes this limitation
by allowing its users to specify new hash table tests and hashing
methods.  The hashing method must also be specified, since the
compiler is unable to determine a good hashing function for an
arbitrary equality (equivalence) predicate.

\begin{defun}{extensions:}{define-hash-table-test}%
  {\args{\var{hash-table-test-name} \var{test-function} \var{hash-function}}}
      
      The \var{hash-table-test-name} must be a symbol.
      % I just assumed the above. [2002/10/10:rpg]
      The \var{test-function} takes two objects and returns true
      iff they are the same.  The \var{hash-function} takes one object and
      returns two values: the (positive fixnum) hash value and true if
      the hashing depends on pointer values and will have to be redone
      if the object moves.
      
      To create a hash-table using this new ``test'' (really, a
      test/hash-function pair), use
      \code{(\index[funs]{make-hash-table}make-hash-table :test
        \var{hash-table-test-name} \ldots)}.

      Note that it is the \var{hash-table-test-name} that will be
      returned by the function \findexed{hash-table-test}, when applied to
      a hash-table created using this function.

      This function updates \vindexed{hash-table-tests}, which is now
      internal.  
\end{defun}

\cmucl{} also supports a number of weak hash tables.  These weak
tables are created using the \kwd{weak-p} argument to
\code{make-hash-table}.  Normally, a reference to an object as either
the key or value of the hash-table will prevent that object from being
garbage-collected.  However, in a weak table, if the only reference is
the hash-table, the object can be collected.

The possible values for \kwd{weak-p} are listed below.  An entry in
the table remains if the condition holds
\begin{Lentry}
\item[\kwd{key}] The key is referenced elsewhere
\item[\kwd{value}] The value is referenced elsewhere
\item[\kwd{key-and-value}] Both the key and value are referenced elsewhere
\item[\kwd{key-or-value}] Either the key or value are referenced elsewhere
\item[T] For backward compatibility, this means the same as \kwd{key}.
\end{Lentry}
If the condition does not hold, the object can be removed from the
hash table.  

Weak hash tables can only be created if the test is \code{eq} or
\code{eql}.  An error is signaled if this is not the case.

\begin{defun}{}{make-hash-table}%
  {\args{\keys{\kwd{test} \kwd{size} \kwd{rehash-size} \kwd{rehash-threshold} \kwd{weak-p}}}}
  Creates a hash-table with the specified properties.
\end{defun}
\section{Default Interrupts for Lisp}

\cmucl{} has several interrupt handlers defined when it starts up,
as follows:
\begin{Lentry}
  
\item[\code{SIGINT} (\ctrl{c})] causes Lisp to enter a break loop.
  This puts you into the debugger which allows you to look at the
  current state of the computation.  If you proceed from the break
  loop, the computation will proceed from where it was interrupted.
  
\item[\code{SIGQUIT} (\ctrl{L})] causes Lisp to do a throw to the
  top-level.  This causes the current computation to be aborted, and
  control returned to the top-level read-eval-print loop.
  
\item[\code{SIGTSTP} (\ctrl{z})] causes Lisp to suspend execution and
  return to the Unix shell.  If control is returned to Lisp, the
  computation will proceed from where it was interrupted.
  
\item[\code{SIGILL}, \code{SIGBUS}, \code{SIGSEGV}, and \code{SIGFPE}]
  cause Lisp to signal an error.
\end{Lentry}
For keyboard interrupt signals, the standard interrupt character is in
parentheses.  Your \file{.login} may set up different interrupt
characters.  When a signal is generated, there may be some delay before
it is processed since Lisp cannot be interrupted safely in an arbitrary
place.  The computation will continue until a safe point is reached and
then the interrupt will be processed.  \xlref{signal-handlers} to define
your own signal handlers.


\section{Implementation-Specific Packages}

When \cmucl{} is first started up, the default package is the
\code{common-lisp-user} package. The \code{common-lisp-user} package
uses the \code{common-lisp} and \code{extensions} packages. The
symbols exported from these three packages can be referenced without
package qualifiers. This section describes packages which have
exported interfaces that may concern users. The numerous internal
packages which implement parts of the system are not described here.
Package nicknames are in parenthesis after the full name.

\begin{Lentry}
\item[\code{alien}, \code{c-call}] Export the features of the Alien
  foreign data structure facility (\pxlref{aliens}.)
  
\item[\code{pcl}] This package contains PCL (Portable CommonLoops),
  which is a portable implementation of CLOS (the Common Lisp Object
  System.)  This implements most (but not all) of the features in the
  CLOS chapter of \cltltwo{}.

\item[\code{clos-mop (mop)}] This package contains an implementation
  of the CLOS Metaobject Protocol, as per the book \textit{The Art of
  the Metaobject Protocol}.
  
\item[\code{debug}] The \code{debug} package contains the command-line
  oriented debugger.  It exports utility various functions and
  switches.
  
\item[\code{debug-internals}] The \code{debug-internals} package
  exports the primitives used to write debuggers.
  \xlref{debug-internals}.
  
\item[\code{extensions (ext)}] The \code{extensions} packages exports
  local extensions to \clisp{} that are documented in this manual.
  Examples include the \code{save-lisp} function and time parsing.
  
\item[\code{hemlock (ed)}] The \code{hemlock} package contains all the
  code to implement Hemlock commands.  The \code{hemlock} package
  currently exports no symbols.
  
\item[\code{hemlock-internals (hi)}] The \code{hemlock-internals}
  package contains code that implements low level primitives and
  exports those symbols used to write Hemlock commands.
  
\item[\code{keyword}] The \code{keyword} package contains keywords
  (e.g., \kwd{start}).  All symbols in the \code{keyword} package are
  exported and evaluate to themselves (i.e., the value of the symbol
  is the symbol itself).
  
\item[\code{profile}] The \code{profile} package exports a simple
  run-time profiling facility (\pxlref{profiling}).
  
\item[\code{common-lisp (cl)}] The \code{common-lisp} package
  exports all the symbols defined by \cltl{} and only those symbols.
  Strictly portable Lisp code will depend only on the symbols exported
  from the \code{common-lisp} package.
  
\item[\code{unix}] This package exports system call
  interfaces to Unix (\pxlref{unix-interface}).
  
\item[\code{system (sys)}] The \code{system} package contains
  functions and information necessary for system interfacing.  This
  package is used by the \code{lisp} package and exports several
  symbols that are necessary to interface to system code.
  
\item[\code{xlib}] The \code{xlib} package contains the Common Lisp X
  interface (CLX) to the X11 protocol.  This is mostly Lisp code with
  a couple of functions that are defined in C to connect to the
  server.
  
\item[\code{wire}] The \code{wire} package exports a remote procedure
  call facility (\pxlref{remote}).

\item[\code{stream}] The \code{stream} package exports the public
  interface to the simple-streams implementation (\pxlref{simple-streams}).

\item[\code{xref}] The \code{xref} package exports the public
  interface to the cross-referencing utility (\pxlref{xref}).

\end{Lentry}


\input{hierarchical-packages}

\input{package-locks}



\section{The Editor}

The \code{ed} function invokes the Hemlock editor which is described
in {\it Hemlock User's Manual} and {\it Hemlock Command Implementor's
Manual}. Most users at CMU prefer to use Hemlock's slave \llisp{}
mechanism which provides an interactive buffer for the
\code{read-eval-print} loop and editor commands for evaluating and
compiling text from a buffer into the slave \llisp.  Since the editor
runs in the \llisp, using slaves keeps users from trashing their
editor by developing in the same \llisp{} with \hemlock{}.


\section{Garbage Collection}

\cmucl{} uses either a stop-and-copy garbage collector or a
generational, mostly copying garbage collector.  Which collector is
available depends on the platform and the features of the platform.
The stop-and-copy GC is available on all RISC platforms.  The x86
platform supports a conservative stop-and-copy collector, which is now
rarely used, and a generational conservative collector.  On the Sparc
platform, both the stop-and-copy GC and the generational GC are
available, but the stop-and-copy GC is deprecated in favor of the
generational GC.  

The generational GC is available if \var{*features*} contains
\code{:gencgc}.

%% The stop-and-copy GC compacts the items in dynamic space every time it
%% runs. Most users cause the system to garbage collect (GC) frequently,
%% long before space is exhausted. With 16 or 24 megabytes of memory,
%% causing GC's more frequently on less garbage allows the system to GC
%% without much (if any) paging.

The following functions invoke the garbage collector or control whether
automatic garbage collection is in effect:

\begin{defun}[-cheney]{extensions:}{gc}{\args{\ampoptional{} \var{verbose-p}}}
  
  This function runs the garbage collector.  If
  \code{ext:*gc-verbose*} is non-\nil, then it invokes
  \code{ext:*gc-notify-before*} before GC'ing and
  \code{ext:*gc-notify-after*} afterwards.
  
  \code{verbose-p} indicates whether GC statistics are printed or
  not. 

\end{defun}

\begin{defun}{extensions:}{gc-off}{}
  
  This function inhibits automatic garbage collection.  After calling
  it, the system will not GC unless you call \code{ext:gc} or
  \code{ext:gc-on}.
\end{defun}

\begin{defun}{extensions:}{gc-on}{}
  
  This function reinstates automatic garbage collection.  If the
  system would have GC'ed while automatic GC was inhibited, then this
  will call \code{ext:gc}.
\end{defun}

\subsection{GC Parameters}

The following variables control the behavior of the garbage collector:

\begin{defvar}{extensions:}{bytes-consed-between-gcs}
  
  \cmucl{} automatically GC's whenever the amount of memory
  allocated to dynamic objects exceeds the value of an internal
  variable.  After each GC, the system sets this internal variable to
  the amount of dynamic space in use at that point plus the value of
  the variable \code{ext:*bytes-consed-between-gcs*}.  The default
  value is 2000000.
\end{defvar}

\begin{defvar}{extensions:}{gc-verbose}
  
  This variable controls whether \code{ext:gc} invokes the functions
  in \code{ext:*gc-notify-before*} and
  \code{ext:*gc-notify-after*}.  If \code{*gc-verbose*} is \nil,
  \code{ext:gc} foregoes printing any messages.  The default value is
  \code{T}.
\end{defvar}

\begin{defvar}{extensions:}{gc-notify-before}
  
  This variable's value is a function that should notify the user that
  the system is about to GC.  It takes one argument, the amount of
  dynamic space in use before the GC measured in bytes.  The default
  value of this variable is a function that prints a message similar
  to the following:
\begin{verbatim}
   [GC threshold exceeded with 2,107,124 bytes in use.  Commencing GC.]
\end{verbatim}
\end{defvar}

\begin{defvar}{extensions:}{gc-notify-after}
  
  This variable's value is a function that should notify the user when
  a GC finishes.  The function must take three arguments, the amount
  of dynamic spaced retained by the GC, the amount of dynamic space
  freed, and the new threshold which is the minimum amount of space in
  use before the next GC will occur.  All values are byte quantities.
  The default value of this variable is a function that prints a
  message similar to the following:
  \begin{verbatim}
    [GC completed with 25,680 bytes retained and 2,096,808 bytes freed.]
    [GC will next occur when at least 2,025,680 bytes are in use.]
  \end{verbatim}
\end{defvar}

Note that a garbage collection will not happen at exactly the new
threshold printed by the default \code{ext:*gc-notify-after*}
function.  The system periodically checks whether this threshold has
been exceeded, and only then does a garbage collection.

\begin{defvar}{extensions:}{gc-inhibit-hook}
  
  This variable's value is either a function of one argument or \nil.
  When the system has triggered an automatic GC, if this variable is a
  function, then the system calls the function with the amount of
  dynamic space currently in use (measured in bytes).  If the function
  returns \nil, then the GC occurs; otherwise, the system inhibits
  automatic GC as if you had called \code{ext:gc-off}.  The writer of
  this hook is responsible for knowing when automatic GC has been
  turned off and for calling or providing a way to call
  \code{ext:gc-on}.  The default value of this variable is \nil.
\end{defvar}

\begin{defvar}{extensions:}{before-gc-hooks}
  \defvarx[extensions:]{after-gc-hooks}
  
  These variables' values are lists of functions to call before or
  after any GC occurs.  The system provides these purely for
  side-effect, and the functions take no arguments.
\end{defvar}

\subsection{Generational GC}
Generational GC also supports some additional functions and variables
to control it.

\begin{defun}[-gencgc]{extensions:}{gc}{\args{\keys{\kwd{verbose} \kwd{gen} \kwd{full}}}}
  
  This function runs the garbage collector.  If
  \code{ext:*gc-verbose*} is non-\nil, then it invokes
  \code{ext:*gc-notify-before*} before GC'ing and
  \code{ext:*gc-notify-after*} afterwards.

  \begin{Lentry}
  \item[\code{verbose}] Print GC statistics if non-\code{NIL}.
  \item[\code{gen}] The number of generations to be collected.
  \item[\code{full}] If non-\code{NIL}, a full collection of all
    generations is performed.
  \end{Lentry}
\end{defun}

\begin{defun}{lisp::}{gencgc-stats}{\args{\var{generation}}}
  Returns statistics about the generation, as multiple values:
  \begin{enumerate}
  \item Bytes allocated in this generation
  \item The GC trigger for this generation.  When this many bytes have
    been allocated, a GC is started automatically.
  \item The number of bytes consed between GCs.
  \item The number of GCs that have been done on this generation.
    This is reset to zero when the generation is raised.
  \item The trigger age, which is the maximum number of GCs to perform
    before this generation is raised.
  \item The total number of bytes allocated to this generation.
  \item Average age of the objects in this generations.  The average
    age is the cumulative bytes allocated divided by current number of
    bytes allocated.
  \end{enumerate}
\end{defun}

\begin{defun}{lisp::}{set-gc-trigger}{\args{\var{gen} \var{trigger}}}
  Sets the GC trigger value for the specified generation.
\end{defun}

\begin{defun}{lisp::}{set-trigger-age}{\args{\var{gen} \var{trigger-age}}}
  Sets the GC trigger age for the specified generation.
\end{defun}

\begin{defun}{lisp::}{set-min-mem-age}{\args{\var{gen} \var{min-mem-age}}}
  Sets the minimum average memory age for the specified generation.
  If the computed memory age is below this, GC is not performed, which
  helps prevent a GC when a large number of new live objects have been
  added in which case a GC would usually be a waste of time.
\end{defun}

\subsection{Weak Pointers}

A weak pointer provides a way to maintain a reference to an object
without preventing an object from being garbage collected.  If the
garbage collector discovers that the only pointers to an object are
weak pointers, then it breaks the weak pointers and deallocates the
object.

\begin{defun}{extensions:}{make-weak-pointer}{\args{\var{object}}}
  \defunx[extensions:]{weak-pointer-value}{\args{\var{weak-pointer}}}
  
  \code{make-weak-pointer} returns a weak pointer to an object.
  \code{weak-pointer-value} follows a weak pointer, returning the two
  values: the object pointed to (or \false{} if broken) and a boolean
  value which is \false{} if the pointer has been broken, and true
  otherwise.
\end{defun}


\subsection{Finalization}

Finalization provides a ``hook'' that is triggered when the garbage
collector reclaims an object.  It is usually used to recover non-Lisp
resources that were allocated to implement the finalized Lisp object.
For example, when a unix file-descriptor stream is collected,
finalization is used to close the underlying file descriptor.

\begin{defun}{extensions:}{finalize}{\args{\var{object} \var{function}}}
  
  This function registers \var{object} for finalization.
  \var{function} is called with no arguments when \var{object} is
  reclaimed.  Normally \var{function} will be a closure over the
  underlying state that needs to be freed, e.g. the unix file
  descriptor in the fd-stream case.  Note that \var{function} must not
  close over \var{object} itself, as this prevents the object from
  ever becoming garbage.
\end{defun}

\begin{defun}{extensions:}{cancel-finalization}{\args{\var{object}}}
  
  This function cancel any finalization request for \var{object}.
\end{defun}


\section{Describe}

\begin{defun}{}{describe}{ \args{\var{object} \ampoptional{} \var{stream}}}
  
  The \code{describe} function prints useful information about
  \var{object} on \var{stream}, which defaults to
  \code{*standard-output*}.  For any object, \code{describe} will
  print out the type.  Then it prints other information based on the
  type of \var{object}.  The types which are presently handled are:

  \begin{Lentry}
  
  \item[\tindexed{hash-table}] \code{describe} prints the number of
    entries currently in the hash table and the number of buckets
    currently allocated.
  
  \item[\tindexed{function}] \code{describe} prints a list of the
    function's name (if any) and its formal parameters.  If the name
    has function documentation, then it will be printed.  If the
    function is compiled, then the file where it is defined will be
    printed as well.
  
  \item[\tindexed{fixnum}] \code{describe} prints whether the integer
    is prime or not.
  
  \item[\tindexed{symbol}] The symbol's value, properties, and
    documentation are printed.  If the symbol has a function
    definition, then the function is described.
  \end{Lentry}
  If there is anything interesting to be said about some component of
  the object, describe will invoke itself recursively to describe that
  object.  The level of recursion is indicated by indenting output.
\end{defun}

A number of switches can be used to control \code{describe}'s behavior.

\begin{defvar}{extensions:}{describe-level}

  The maximum level of recursive description allowed.  Initially two.
\end{defvar}

\begin{defvar}{extensions:}{describe-indentation}

The number of spaces to indent for each level of recursive
description, initially three.
\end{defvar}

\begin{defvar}{extensions:}{describe-print-level}
  \defvarx[extensions:]{describe-print-length}
  
  The values of \code{*print-level*} and \code{*print-length*} during
  description.  Initially two and five.
\end{defvar}


\section{The Inspector}

\cmucl{} has both a graphical inspector that uses the X Window System,
and a simple terminal-based inspector.

\begin{defun}{}{inspect}{ \args{\ampoptional{} \var{object}}}
  
  \code{inspect} calls the inspector on the optional argument
  \var{object}.  If \var{object} is unsupplied, \code{inspect}
  immediately returns \false.  Otherwise, the behavior of inspect
  depends on whether Lisp is running under X.  When \code{inspect} is
  eventually exited, it returns some selected Lisp object.
\end{defun}


\subsection{The Graphical Interface}
\label{motif-interface}

\cmucl{} has an interface to Motif which is functionally similar to
CLM, but works better in \cmucl{}.  This interface is documented in
separate manuals \textit{CMUCL Motif Toolkit} and \textit{Design Notes
on the Motif Toolkit}, which are distributed with \cmucl{}.

This motif interface has been used to write the inspector and graphical
debugger.  There is also a Lisp control panel with a simple file management
facility, apropos and inspector dialogs, and controls for setting global
options.  See the \code{interface} and \code{toolkit} packages.

\begin{defun}{interface:}{lisp-control-panel}{}
  
  This function creates a control panel for the Lisp process.
\end{defun}

\begin{defvar}{interface:}{interface-style}
  
  When the graphical interface is loaded, this variable controls
  whether it is used by \code{inspect} and the error system.  If the
  value is \kwd{graphics} (the default) and the \code{DISPLAY}
  environment variable is defined, the graphical inspector and
  debugger will be invoked by \findexed{inspect} or when an error is
  signalled.  Possible values are \kwd{graphics} and {tty}.  If the
  value is \kwd{graphics}, but there is no X display, then we quietly
  use the TTY interface.
\end{defvar}


\subsection{The TTY Inspector}

If X is unavailable, a terminal inspector is invoked.  The TTY inspector
is a crude interface to \code{describe} which allows objects to be
traversed and maintains a history.  This inspector prints information
about and object and a numbered list of the components of the object.
The command-line based interface is a normal
\code{read}--\code{eval}--\code{print} loop, but an integer \var{n}
descends into the \var{n}'th component of the current object, and
symbols with these special names are interpreted as commands:

\begin{Lentry}
\item[U] Move back to the enclosing object.  As you descend into the
components of an object, a stack of all the objects previously seen is
kept.  This command pops you up one level of this stack.

\item[Q, E] Return the current object from \code{inspect}.

\item[R] Recompute object display, and print again.  Useful if the
object may have changed.

\item[D] Display again without recomputing.

\item[H, ?] Show help message.
\end{Lentry}


\section{Load}

\begin{defun}{}{load}{%
    \args{\var{filename}
      \keys{\kwd{verbose} \kwd{print} \kwd{if-does-not-exist}}
      \morekeys{\kwd{if-source-newer} \kwd{contents}}}}
  
  As in standard \clisp{}, this function loads a file containing
  source or object code into the running Lisp.  Several CMU extensions
  have been made to \code{load} to conveniently support a variety of
  program file organizations.  \var{filename} may be a wildcard
  pathname such as \file{*.lisp}, in which case all matching files are
  loaded.
  
  If \var{filename} has a \code{pathname-type} (or extension), then
  that exact file is loaded.  If the file has no extension, then this
  tells \code{load} to use a heuristic to load the ``right'' file.
  The \code{*load-source-types*} and \code{*load-object-types*}
  variables below are used to determine the default source and object
  file types.  If only the source or the object file exists (but not
  both), then that file is quietly loaded.  Similarly, if both the
  source and object file exist, and the object file is newer than the
  source file, then the object file is loaded.  The value of the
  \var{if-source-newer} argument is used to determine what action to
  take when both the source and object files exist, but the object
  file is out of date:
  \begin{Lentry}
  \item[\kwd{load-object}] The object file is loaded even though the
    source file is newer.
    
  \item[\kwd{load-source}] The source file is loaded instead of the
    older object file.
    
  \item[\kwd{compile}] The source file is compiled and then the new
    object file is loaded.
    
  \item[\kwd{query}] The user is asked a yes or no question to
    determine whether the source or object file is loaded.
  \end{Lentry}
  This argument defaults to the value of
  \code{ext:*load-if-source-newer*} (initially \kwd{load-object}.)
  
  The \var{contents} argument can be used to override the heuristic
  (based on the file extension) that normally determines whether to
  load the file as a source file or an object file.  If non-null, this
  argument must be either \kwd{source} or \kwd{binary}, which forces
  loading in source and binary mode, respectively. You really
  shouldn't ever need to use this argument.
\end{defun}

\begin{defvar}{extensions:}{load-source-types}
  \defvarx[extensions:]{load-object-types}
  
  These variables are lists of possible \code{pathname-type} values
  for source and object files to be passed to \code{load}.  These
  variables are only used when the file passed to \code{load} has no
  type; in this case, the possible source and object types are used to
  default the type in order to determine the names of the source and
  object files.
\end{defvar}

\begin{defvar}{extensions:}{load-if-source-newer}
  
  This variable determines the default value of the
  \var{if-source-newer} argument to \code{load}.  Its initial value is
  \kwd{load-object}.
\end{defvar}


\section{The Reader}

\subsection{Reader Extensions}
\cmucl{} supports an ANSI-compatible extension to enable reading of
specialized arrays.  Thus
\begin{example}
  * (setf *print-readably* nil)
  NIL
  * (make-array '(2 2) :element-type '(signed-byte 8))
  #2A((0 0) (0 0))
  * (setf *print-readably* t)
  T
  * (make-array '(2 2) :element-type '(signed-byte 8))
  #A((SIGNED-BYTE 8) (2 2) ((0 0) (0 0)))
  * (type-of (read-from-string "#A((SIGNED-BYTE 8) (2 2) ((0 0) (0 0)))"))
  (SIMPLE-ARRAY (SIGNED-BYTE 8) (2 2))
  * (setf *print-readably* nil)
  NIL
  * (type-of (read-from-string "#A((SIGNED-BYTE 8) (2 2) ((0 0) (0 0)))"))
  (SIMPLE-ARRAY (SIGNED-BYTE 8) (2 2))
\end{example}

\subsection{Reader Parameters}
\begin{defvar}{extensions:}{ignore-extra-close-parentheses}
  
  If this variable is \true{} (the default), then the reader merely
  prints a warning when an extra close parenthesis is detected
  (instead of signalling an error.)
\end{defvar}

\section{Stream Extensions}
\begin{defun}{sys:}{read-n-bytes}{%
    \args{\var{stream buffer start numbytes} 
      \ampoptional{} \var{eof-error-p}}}
  
  On streams that support it, this function reads multiple bytes of
  data into a buffer.  The buffer must be a \code{simple-string} or
  \code{(simple-array (unsigned-byte 8) (*))}.  The argument
  \var{nbytes} specifies the desired number of bytes, and the return
  value is the number of bytes actually read.
  \begin{itemize}
  \item If \var{eof-error-p} is true, an \tindexed{end-of-file}
    condition is signalled if end-of-file is encountered before
    \var{count} bytes have been read.
    
  \item If \var{eof-error-p} is false, \code{read-n-bytes reads} as
    much data is currently available (up to count bytes.)  On pipes or
    similar devices, this function returns as soon as any data is
    available, even if the amount read is less than \var{count} and
    eof has not been hit.  See also \funref{make-fd-stream}.
  \end{itemize}
\end{defun}

\input{simple-streams}

\section{Running Programs from Lisp}

It is possible to run programs from Lisp by using the following function.

\begin{defun}{extensions:}{run-program}{%
    \args{\var{program} \var{args}
      \keys{\kwd{env} \kwd{wait} \kwd{pty} \kwd{input}}
      \morekeys{\kwd{if-input-does-not-exist}}
      \yetmorekeys{\kwd{output} \kwd{if-output-exists}}
      \yetmorekeys{\kwd{error} \kwd{if-error-exists}}
      \yetmorekeys{\kwd{status-hook} \kwd{external-format}}
      \yetmorekeys{\kwd{element-type}}}}
     
  \code{run-program} runs \var{program} in a child process.
  \var{Program} should be a pathname or string naming the program.
  \var{Args} should be a list of strings which this passes to
  \var{program} as normal Unix parameters.  For no arguments, specify
  \var{args} as \nil.  The value returned is either a process
  structure or \nil.  The process interface follows the description of
  \code{run-program}.  If \code{run-program} fails to fork the child
  process, it returns \nil.
  
  Except for sharing file descriptors as explained in keyword argument
  descriptions, \code{run-program} closes all file descriptors in the
  child process before running the program.  When you are done using a
  process, call \code{process-close} to reclaim system resources.  You
  only need to do this when you supply \kwd{stream} for one of
  \kwd{input}, \kwd{output}, or \kwd{error}, or you supply \kwd{pty}
  non-\nil.  You can call \code{process-close} regardless of whether
  you must to reclaim resources without penalty if you feel safer.

  \code{run-program} accepts the following keyword arguments:

  \begin{Lentry}   
  \item[\kwd{env}] This is an a-list mapping keywords and
    simple-strings.  The default is \code{ext:*environment-list*}.  If
    \kwd{env} is specified, \code{run-program} uses the value given
    and does not combine the environment passed to Lisp with the one
    specified.
    
  \item[\kwd{wait}] If non-\nil{} (the default), wait until the child
    process terminates.  If \nil, continue running Lisp while the
    child process runs.
    
  \item[\kwd{pty}] This should be one of \true, \nil, or a stream.  If
    specified non-\nil, the subprocess executes under a Unix PTY.
    If specified as a stream, the system collects all output to this
    pty and writes it to this stream.  If specified as \true, the
    \code{process-pty} slot contains a stream from which you can read
    the program's output and to which you can write input for the
    program.  The default is \nil.
    
  \item[\kwd{input}] This specifies how the program gets its input.
    If specified as a string, it is the name of a file that contains
    input for the child process.  \code{run-program} opens the file as
    standard input.  If specified as \nil{} (the default), then
    standard input is the file \file{/dev/null}.  If specified as
    \true, the program uses the current standard input.  This may
    cause some confusion if \kwd{wait} is \nil{} since two processes
    may use the terminal at the same time.  If specified as
    \kwd{stream}, then the \code{process-input} slot contains an
    output stream.  Anything written to this stream goes to the
    program as input.  \kwd{input} may also be an input stream that
    already contains all the input for the process.  In this case
    \code{run-program} reads all the input from this stream before
    returning, so this cannot be used to interact with the process.
    If \kwd{input} is a string stream, it is up to the caller to call
    \code{string-encode} or other function to convert the string to
    the appropriate encoding.  In either case, the least significant 8
    bits of the \code{char-code} of each \code{character} is
    sent to the program.
    
  \item[\kwd{if-input-does-not-exist}] This specifies what to do if
    the input file does not exist.  The following values are valid:
    \nil{} (the default) causes \code{run-program} to return \nil{}
    without doing anything; \kwd{create} creates the named file; and
    \kwd{error} signals an error.
    
  \item[\kwd{output}] This specifies what happens with the program's
    output.  If specified as a pathname, it is the name of a file that
    contains output the program writes to its standard output.  If
    specified as \nil{} (the default), all output goes to
    \file{/dev/null}.  If specified as \true, the program writes to
    the Lisp process's standard output.  This may cause confusion if
    \kwd{wait} is \nil{} since two processes may write to the terminal
    at the same time.  If specified as \kwd{stream}, then the
    \code{process-output} slot contains an input stream from which you
    can read the program's output.  \kwd{output} can also be a stream
    in which case all output from the process is written to this
    stream.  If \kwd{output} is a string-stream, each octet read from
    the program is converted to a character using \code{code-char}.
    It is up to the caller to convert this using the appropriate
    external format to create the desired encoded string.
    
  \item[\kwd{if-output-exists}] This specifies what to do if the
    output file already exists.  The following values are valid:
    \nil{} causes \code{run-program} to return \nil{} without doing
    anything; \kwd{error} (the default) signals an error;
    \kwd{supersede} overwrites the current file; and \kwd{append}
    appends all output to the file.
    
  \item[\kwd{error}] This is similar to \kwd{output}, except the file
    becomes the program's standard error.  Additionally, \kwd{error}
    can be \kwd{output} in which case the program's error output is
    routed to the same place specified for \kwd{output}.  If specified
    as \kwd{stream}, the \code{process-error} contains a stream
    similar to the \code{process-output} slot when specifying the
    \kwd{output} argument.
    
  \item[\kwd{if-error-exists}] This specifies what to do if the error
    output file already exists.  It accepts the same values as
    \kwd{if-output-exists}.
    
  \item[\kwd{status-hook}] This specifies a function to call whenever
    the process changes status.  This is especially useful when
    specifying \kwd{wait} as \nil.  The function takes the process as
    a required argument.

  \item[\kwd{external-format}] This specifies the external format to
    use for streams created for \code{run-program}.  This does not
    apply to string streams passed in as \kwd{input} or \kwd{output}
    parameters.

  \item[\kwd{element-type}] If streams are created \code{run-program},
    use this as the \kwd{element-type} for the stream.  Defaults to
    \code{BASE-CHAR}. 
    
%   \item[\kwd{before-execve}] This specifies a function to run in the
%     child process before it becomes the program to run.  This is
%     useful for actions such as authenticating the child process
%     without modifying the parent Lisp process.
  \end{Lentry}
\end{defun}


\subsection{Process Accessors}

The following functions interface the process returned by \code{run-program}:

\begin{defun}{extensions:}{process-p}{\args{\var{thing}}}
  
  This function returns \true{} if \var{thing} is a process.
  Otherwise it returns \nil{}
\end{defun}

\begin{defun}{extensions:}{process-pid}{\args{\var{process}}}
  
  This function returns the process ID, an integer, for the
  \var{process}.
\end{defun}

\begin{defun}{extensions:}{process-status}{\args{\var{process}}}
  
  This function returns the current status of \var{process}, which is
  one of \kwd{running}, \kwd{stopped}, \kwd{exited}, or
  \kwd{signaled}.
\end{defun}

\begin{defun}{extensions:}{process-exit-code}{\args{\var{process}}}
  
  This function returns either the exit code for \var{process}, if it
  is \kwd{exited}, or the termination signal \var{process} if it is
  \kwd{signaled}.  The result is undefined for processes that are
  still alive.
\end{defun}

\begin{defun}{extensions:}{process-core-dumped}{\args{\var{process}}}
  
  This function returns \true{} if someone used a Unix signal to
  terminate the \var{process} and caused it to dump a Unix core image.
\end{defun}

\begin{defun}{extensions:}{process-pty}{\args{\var{process}}}
  
  This function returns either the two-way stream connected to
  \var{process}'s Unix PTY connection or \nil{} if there is none.
\end{defun}

\begin{defun}{extensions:}{process-input}{\args{\var{process}}}
  \defunx[extensions:]{process-output}{\args{\var{process}}}
  \defunx[extensions:]{process-error}{\args{\var{process}}}
  
  If the corresponding stream was created, these functions return the
  input, output or error fd-stream.  \nil{} is returned if there
  is no stream.
\end{defun}

\begin{defun}{extensions:}{process-status-hook}{\args{\var{process}}}
  
  This function returns the current function to call whenever
  \var{process}'s status changes.  This function takes the
  \var{process} as a required argument.  \code{process-status-hook} is
  \code{setf}'able.
\end{defun}

\begin{defun}{extensions:}{process-plist}{\args{\var{process}}}
  
  This function returns annotations supplied by users, and it is
  \code{setf}'able.  This is available solely for users to associate
  information with \var{process} without having to build a-lists or
  hash tables of process structures.
\end{defun}

\begin{defun}{extensions:}{process-wait}{
    \args{\var{process} \ampoptional{} \var{check-for-stopped}}}
  
  This function waits for \var{process} to finish.  If
  \var{check-for-stopped} is non-\nil, this also returns when
  \var{process} stops.
\end{defun}

\begin{defun}{extensions:}{process-kill}{%
    \args{\var{process} \var{signal} \ampoptional{} \var{whom}}}
  
  This function sends the Unix \var{signal} to \var{process}.
  \var{Signal} should be the number of the signal or a keyword with
  the Unix name (for example, \kwd{sigsegv}).  \var{Whom} should be
  one of the following:
  \begin{Lentry}
    
  \item[\kwd{pid}] This is the default, and it indicates sending the
    signal to \var{process} only.
    
  \item[\kwd{process-group}] This indicates sending the signal to
    \var{process}'s group.
    
  \item[\kwd{pty-process-group}] This indicates sending the signal to
    the process group currently in the foreground on the Unix PTY
    connected to \var{process}.  This last option is useful if the
    running program is a shell, and you wish to signal the program
    running under the shell, not the shell itself.  If
    \code{process-pty} of \var{process} is \nil, using this option is
    an error.
  \end{Lentry}
\end{defun}

\begin{defun}{extensions:}{process-alive-p}{\args{\var{process}}}
  
  This function returns \true{} if \var{process}'s status is either
  \kwd{running} or \kwd{stopped}.
\end{defun}

\begin{defun}{extensions:}{process-close}{\args{\var{process}}}
  
  This function closes all the streams associated with \var{process}.
  When you are done using a process, call this to reclaim system
  resources.
\end{defun}


\section{Saving a Core Image}

A mechanism has been provided to save a running Lisp core image and to
later restore it.  This is convenient if you don't want to load several files
into a Lisp when you first start it up.  The main problem is the large
size of each saved Lisp image, typically at least 20 megabytes.

\begin{defun}{extensions:}{save-lisp}{%
    \args{\var{file}
      \keys{\kwd{purify} \kwd{root-structures} \kwd{init-function}}
      \morekeys{\kwd{load-init-file} \kwd{print-herald} \kwd{site-init}}
      \yetmorekeys{\kwd{process-command-line} \kwd{batch-mode} \kwd{executable}}}}
  
  The \code{save-lisp} function saves the state of the currently
  running Lisp core image in \var{file}.  The keyword arguments have
  the following meaning:
  \begin{Lentry}
    
  \item[\kwd{purify}] If non-\nil{} (the default), the core image is
    purified before it is saved (see \funref{purify}.)  This reduces
    the amount of work the garbage collector must do when the
    resulting core image is being run.  Also, if more than one Lisp is
    running on the same machine, this maximizes the amount of memory
    that can be shared between the two processes.
    
  \item[\kwd{root-structures}]
      This should be a list of the main entry points in any newly
      loaded systems.  This need not be supplied, but locality and/or
      GC performance will be better if they are.  Meaningless if
      \kwd{purify} is \nil.  See \funref{purify}.

  \item[\kwd{init-function}] This is the function that starts running
    when the created core file is resumed.  The default function
    simply invokes the top level read-eval-print loop.  If the
    function returns the lisp will exit.
    
  \item[\kwd{load-init-file}] If non-NIL, then load an init file;
    either the one specified on the command line or
    ``\w{\file{init.}\var{fasl-type}}'', or, if
    ``\w{\file{init.}\var{fasl-type}}'' does not exist,
    \code{init.lisp} from the user's home directory.  If the init file
    is found, it is loaded into the resumed core file before the
    read-eval-print loop is entered.
    
  \item[\kwd{site-init}] If non-NIL, the name of the site init file to
    quietly load.  The default is \file{library:site-init}.  No error
    is signalled if the file does not exist.
    
  \item[\kwd{print-herald}] If non-NIL (the default), then print out
    the standard Lisp herald when starting.
    
  \item[\kwd{process-command-line}] If non-NIL (the default),
    processes the command line switches and performs the appropriate
    actions.

  \item[\kwd{batch-mode}] If NIL (the default), then the presence of
    the -batch command-line switch will invoke batch-mode processing
    upon resuming the saved core.  If non-NIL, the produced core will
    always be in batch-mode, regardless of any command-line switches.

  \item[\kwd{executable}] If non-NIL, an executable image is created.
    Normally, \cmucl{} consists of the C runtime along with a core
    file image.  When \kwd{executable} is non-NIL, the core file is
    incorporated into the C runtime, so one (large) executable is
    created instead of a new separate core file.

    This feature is only available on some platforms, as indicated by
    having the feature \kwd{executable}.  Currently only x86 ports and
    the solaris/sparc port have this feature.
  \end{Lentry}
\end{defun}

To resume a saved file, type:
\begin{example}
lisp -core file
\end{example}
However, if the \kwd{executable} option was specified, you can just
use
\begin{example}
  file
\end{example}
since the executable contains the core file within the executable.

\begin{defun}{extensions:}{purify}{
    \args{\var{file}
      \keys{\kwd{root-structures} \kwd{environment-name}}}}
  
  This function optimizes garbage collection by moving all currently
  live objects into non-collected storage.  Once statically allocated,
  the objects can never be reclaimed, even if all pointers to them are
  dropped.  This function should generally be called after a large
  system has been loaded and initialized.

  \begin{Lentry}
  \item[\kwd{root-structures}] is an optional list of objects which
    should be copied first to maximize locality.  This should be a
    list of the main entry points for the resulting core image.  The
    purification process tries to localize symbols, functions, etc.,
    in the core image so that paging performance is improved.  The
    default value is NIL which means that Lisp objects will still be
    localized but probably not as optimally as they could be.
  
    \var{defstruct} structures defined with the \code{(:pure t)}
    option are moved into read-only storage, further reducing GC cost.
    List and vector slots of pure structures are also moved into
    read-only storage.
  
  \item[\kwd{environment-name}] is gratuitous documentation for the
    compacted version of the current global environment (as seen in
    \code{c::*info-environment*}.)  If \false{} is supplied, then
    environment compaction is inhibited.
  \end{Lentry}
\end{defun}


\section{Pathnames}

In \clisp{} quite a few aspects of \tindexed{pathname} semantics are left to
the implementation.  


\subsection{Unix Pathnames}
\cpsubindex{unix}{pathnames}

Unix pathnames are always parsed with a \code{unix-host} object as the host and
\code{nil} as the device.  The last two dots (\code{.}) in the namestring mark
the type and version, however if the first character is a dot, it is considered
part of the name.  If the last character is a dot, then the pathname has the
empty-string as its type.  The type defaults to \code{nil} and the version
defaults to \kwd{newest}.

\begin{example}
(defun parse (x)
  (values (pathname-name x) (pathname-type x) (pathname-version x)))

(parse "foo") \result "foo", NIL, NIL
(parse "foo.bar") \result "foo", "bar", NIL
(parse ".foo") \result ".foo", NIL, NIL
(parse ".foo.bar") \result ".foo", "bar", NIL
(parse "..") \result NIL, NIL, NIL
(parse "foo.") \result "foo", "", NIL
(parse "foo.bar.~1~") \result "foo", "bar", 1
(parse "foo.bar.baz") \result "foo.bar", "baz", NIL
\end{example}

The directory of pathnames beginning with a slash (or a search-list,
\pxlref{search-lists}) is starts \kwd{absolute}, others start with
\kwd{relative}.  The \code{..} directory is parsed as \kwd{up}; there is no
namestring for \kwd{back}:

\begin{example}
(pathname-directory "/usr/foo/bar.baz") \result (:ABSOLUTE "usr" "foo")
(pathname-directory "../foo/bar.baz") \result (:RELATIVE :UP "foo")
\end{example}


\subsection{Wildcard Pathnames}

Wildcards are supported in Unix pathnames.  If `\code{*}' is specified for a
part of a pathname, that is parsed as \kwd{wild}.  `\code{**}' can be used as a
directory name to indicate \kwd{wild-inferiors}.  Filesystem operations
treat \kwd{wild-inferiors} the same as\ \kwd{wild}, but pathname pattern
matching (e.g. for logical pathname translation, \pxlref{logical-pathnames})
matches any number of directory parts with `\code{**}' (see
\pxlref{wildcard-matching}.)

`\code{*}' embedded in a pathname part matches any number of characters.
Similarly, `\code{?}' matches exactly one character, and `\code{[a,b]}'
matches the characters `\code{a}' or `\code{b}'.  These pathname parts are
parsed as \code{pattern} objects.

Backslash can be used as an escape character in namestring
parsing to prevent the next character from being treated as a wildcard.  Note
that if typed in a string constant, the backslash must be doubled, since the
string reader also uses backslash as a quote:

\begin{example}
(pathname-name "foo\(\backslash\backslash\)*bar") => "foo*bar"
\end{example}


\subsection{Logical Pathnames}
\cindex{logical pathnames}
\label{logical-pathnames}

If a namestring begins with the name of a defined logical pathname
host followed by a colon, then it will be parsed as a logical
pathname.  Both `\code{*}' and `\code{**}' wildcards are implemented.
\findexed{load-logical-pathname-translations} on \var{name} looks for a
logical host definition file in
\w{\file{library:\var{name}.translations}}. Note that \file{library:}
designates the search list (\pxlref{search-lists}) initialized to the
\cmucl{} \file{lib/} directory, not a logical pathname.  The format of
the file is a single list of two-lists of the from and to patterns:

\begin{example}
(("foo;*.text" "/usr/ram/foo/*.txt")
 ("foo;*.lisp" "/usr/ram/foo/*.l"))
\end{example}


\subsection{Search Lists}
\cindex{search lists}
\label{search-lists}

Search lists are an extension to \clisp{} pathnames.  They serve a function
somewhat similar to \clisp{} logical pathnames, but work more like Unix PATH
variables.  Search lists are used for two purposes:
\begin{itemize}
\item They provide a convenient shorthand for commonly used directory names,
and

\item They allow the abstract (directory structure independent) specification
of file locations in program pathname constants (similar to logical pathnames.)
\end{itemize}
Each search list has an associated list of directories (represented as
pathnames with no name or type component.)  The namestring for any relative
pathname may be prefixed with ``\var{slist}\code{:}'', indicating that the
pathname is relative to the search list \var{slist} (instead of to the current
working directory.)  Once qualified with a search list, the pathname is no
longer considered to be relative.

When a search list qualified pathname is passed to a file-system operation such
as \code{open}, \code{load} or \code{truename}, each directory in the search
list is successively used as the root of the pathname until the file is
located.  When a file is written to a search list directory, the file is always
written to the first directory in the list.


\subsection{Predefined Search-Lists}

These search-lists are initialized from the Unix environment or when Lisp was
built:
\begin{Lentry}
\item[\code{default:}] The current directory at startup.

\item[\code{home:}] The user's home directory.

\item[\code{library:}] The \cmucl{} \file{lib/} directory (\code{CMUCLLIB} environment
variable).

\item[\code{path:}] The Unix command path (\code{PATH} environment variable).
\item[\code{ld-library-path:}] The Unix \code{LD\_LIBRARY\_PATH}
  environment variable.
\item[\code{target:}] The root of the tree where \cmucl{} was compiled.
\item[\code{modules:}] The list of directories where \cmucl{}'s
  modules can be found.
\item[\code{ext-formats:}] The list of directories where \cmucl{} can
  find the implementation of external formats.  
\end{Lentry}
It can be useful to redefine these search-lists, for example, \file{library:}
can be augmented to allow logical pathname translations to be located, and
\file{target:} can be redefined to point to where \cmucl{} system sources are
locally installed. 


\subsection{Search-List Operations}

These operations define and access search-list definitions.  A search-list name
may be parsed into a pathname before the search-list is actually defined, but
the search-list must be defined before it can actually be used in a filesystem
operation.

\begin{defun}{extensions:}{search-list}{\var{name}}
  
  This function returns the list of directories associated with the
  search list \var{name}.  If \var{name} is not a defined search list,
  then an error is signaled.  When set with \code{setf}, the list of
  directories is changed to the new value.  If the new value is just a
  namestring or pathname, then it is interpreted as a one-element
  list.  Note that (unlike Unix pathnames), search list names are
  case-insensitive.
\end{defun}

\begin{defun}{extensions:}{search-list-defined-p}{\var{name}}
  \defunx[extensions:]{clear-search-list}{\var{name}}
  
  \code{search-list-defined-p} returns \true{} if \var{name} is a
  defined search list name, \false{} otherwise.
  \code{clear-search-list} make the search list \var{name} undefined.
\end{defun}

\begin{defmac}{extensions:}{enumerate-search-list}{%
    \args{(\var{var} \var{pathname} \mopt{result}) \mstar{form}}}
  
  This macro provides an interface to search list resolution.  The
  body \var{forms} are executed with \var{var} bound to each
  successive possible expansion for \var{name}.  If \var{name} does
  not contain a search-list, then the body is executed exactly once.
  Everything is wrapped in a block named \nil, so \code{return} can be
  used to terminate early.  The \var{result} form (default \nil) is
  evaluated to determine the result of the iteration.
\end{defmac}


\subsection{Search List Example}

The search list \code{code:} can be defined as follows:
\begin{example}
(setf (ext:search-list "code:") '("/usr/lisp/code/"))
\end{example}
It is now possible to use \code{code:} as an abbreviation for the directory
\file{/usr/lisp/code/} in all file operations.  For example, you can now specify
\code{code:eval.lisp} to refer to the file \file{/usr/lisp/code/eval.lisp}.

To obtain the value of a search-list name, use the function search-list
as follows:
\begin{example}
(ext:search-list \var{name})
\end{example}
Where \var{name} is the name of a search list as described above.  For example,
calling \code{ext:search-list} on \code{code:} as follows:
\begin{example}
(ext:search-list "code:")
\end{example}
returns the list \code{("/usr/lisp/code/")}.


\section{Filesystem Operations}

\cmucl{} provides a number of extensions and optional features beyond those
required by the \clisp{} specification.


\subsection{Wildcard Matching}
\label{wildcard-matching}

Unix filesystem operations such as \code{open} will accept wildcard pathnames
that match a single file (of course, \code{directory} allows any number of
matches.)  Filesystem operations treat \kwd{wild-inferiors} the same as\
\kwd{wild}.

\begin{defun}{}{directory}{\var{wildname} \keys{\kwd{all} \kwd{check-for-subdirs}}
    \kwd{truenamep} \morekeys{\kwd{follow-links}}}
  
  The keyword arguments to this \clisp{} function are a \cmucl{} extension.
  The arguments (all default to \code{t}) have the following
  functions:
  \begin{Lentry}
  \item[\kwd{all}] Include files beginning with dot such as
    \file{.login}, similar to ``\code{ls -a}''.
    
  \item[\kwd{check-for-subdirs}] Test whether files are directories,
    similar to ``\code{ls -F}''.
    
  \item[\kwd{truenamep}] Call \code{truename} on each file, which
    expands out all symbolic links.  Note that this option can easily
    result in pathnames being returned which have a different
    directory from the one in the \var{wildname} argument.

  \item[\kwd{follow-links}] Follow symbolic links when searching for
    matching directories.
  \end{Lentry}
\end{defun}

\begin{defun}{extensions:}{print-directory}{%
    \args{\var{wildname}
      \ampoptional{} \var{stream}
      \keys{\kwd{all} \kwd{verbose}}
      \morekeys{\kwd{return-list}}}}
  
  Print a directory of \var{wildname} listing to \var{stream} (default
  \code{*standard-output*}.)  \kwd{all} and \kwd{verbose} both default
  to \false{} and correspond to the ``\code{-a}'' and ``\code{-l}''
  options of \file{ls}.  Normally this function returns \false{}, but
  if \kwd{return-list} is true, a list of the matched pathnames are
  returned.
\end{defun}


\subsection{File Name Completion}

\begin{defun}{extensions:}{complete-file}{%
    \args{\var{pathname}
      \keys{\kwd{defaults} \kwd{ignore-types}}}}
  
  Attempt to complete a file name to the longest unambiguous prefix.
  If supplied, directory from \kwd{defaults} is used as the ``working
  directory'' when doing completion.  \kwd{ignore-types} is a list of
  strings of the pathname types (a.k.a. extensions) that should be
  disregarded as possible matches (binary file names, etc.)
\end{defun}

\begin{defun}{extensions:}{ambiguous-files}{%
    \args{\var{pathname}
      \ampoptional{} \var{defaults}}}
  
  Return a list of pathnames for all the possible completions of
  \var{pathname} with respect to \var{defaults}.
\end{defun}


\subsection{Miscellaneous Filesystem Operations}

\begin{defun}{extensions:}{default-directory}{}
  
  Return the current working directory as a pathname.  If set with
  \code{setf}, set the working directory.
\end{defun}

\begin{defun}{extensions:}{file-writable}{\var{name}}
  
  This function accepts a pathname and returns \true{} if the current
  process can write it, and \false{} otherwise.
\end{defun}

\begin{defun}{extensions:}{unix-namestring}{%
    \args{\var{pathname}
      \ampoptional{} \var{for-input}}}
  
  This function converts \var{pathname} into a string that can be used
  with UNIX system calls.  Search-lists and wildcards are expanded.
  \var{for-input} controls the treatment of search-lists: when true
  (the default) and the file exists anywhere on the search-list, then
  that absolute pathname is returned; otherwise the first element of
  the search-list is used as the directory.
\end{defun}


\section{Time Parsing and Formatting}

\cindex{time parsing} \cindex{time formatting}
Functions are provided to allow parsing strings containing time information
and printing time in various formats are available.

\begin{defun}{extensions:}{parse-time}{%
    \args{\var{time-string}
      \keys{\kwd{error-on-mismatch} \kwd{default-seconds}}
      \morekeys{\kwd{default-minutes} \kwd{default-hours}}
      \yetmorekeys{\kwd{default-day} \kwd{default-month}}
      \yetmorekeys{\kwd{default-year} \kwd{default-zone}}
      \yetmorekeys{\kwd{default-weekday}}}}
  
  \code{parse-time} accepts a string containing a time (e.g.,
  \w{"\code{Jan 12, 1952}"}) and returns the universal time if it is
  successful.  If it is unsuccessful and the keyword argument
  \kwd{error-on-mismatch} is non-\nil{}, it signals an error.
  Otherwise it returns \nil{}.  The other keyword arguments have the
  following meaning:

  \begin{Lentry}
  \item[\kwd{default-seconds}] specifies the default value for the
    seconds value if one is not provided by \var{time-string}.  The
    default value is 0.
    
  \item[\kwd{default-minutes}] specifies the default value for the
    minutes value if one is not provided by \var{time-string}.  The
    default value is 0.
    
  \item[\kwd{default-hours}] specifies the default value for the hours
    value if one is not provided by \var{time-string}.  The default
    value is 0.
    
  \item[\kwd{default-day}] specifies the default value for the day
    value if one is not provided by \var{time-string}.  The default
    value is the current day.
    
  \item[\kwd{default-month}] specifies the default value for the month
    value if one is not provided by \var{time-string}.  The default
    value is the current month.
    
  \item[\kwd{default-year}] specifies the default value for the year
    value if one is not provided by \var{time-string}.  The default
    value is the current year.
    
  \item[\kwd{default-zone}] specifies the default value for the time
    zone value if one is not provided by \var{time-string}.  The
    default value is the current time zone.
    
  \item[\kwd{default-weekday}] specifies the default value for the day
    of the week if one is not provided by \var{time-string}.  The
    default value is the current day of the week.
  \end{Lentry}
  Any of the above keywords can be given the value \kwd{current} which
  means to use the current value as determined by a call to the
  operating system.
\end{defun}

\begin{defun}{extensions:}{format-universal-time}{
    \args{\var{dest} \var{universal-time}
       \\
       \keys{\kwd{timezone}}
       \morekeys{\kwd{style} \kwd{date-first}}
       \yetmorekeys{\kwd{print-seconds} \kwd{print-meridian}}
       \yetmorekeys{\kwd{print-timezone} \kwd{print-weekday}}}}
   \defunx[extensions:]{format-decoded-time}{
     \args{\var{dest} \var{seconds} \var{minutes} \var{hours} \var{day} \var{month} \var{year}
       \\
       \keys{\kwd{timezone}}
       \morekeys{\kwd{style} \kwd{date-first}}
       \yetmorekeys{\kwd{print-seconds} \kwd{print-meridian}}
       \yetmorekeys{\kwd{print-timezone} \kwd{print-weekday}}}}
   
   \code{format-universal-time} formats the time specified by
   \var{universal-time}.  \code{format-decoded-time} formats the time
   specified by \var{seconds}, \var{minutes}, \var{hours}, \var{day},
   \var{month}, and \var{year}.  \var{Dest} is any destination
   accepted by the \code{format} function.  The keyword arguments have
   the following meaning:
   \begin{Lentry}
     
   \item[\kwd{timezone}] is an integer specifying the hours west of
     Greenwich.  \kwd{timezone} defaults to the current time zone.
     
   \item[\kwd{style}] specifies the style to use in formatting the
     time.  The legal values are:
     \begin{Lentry}
  
     \item[\kwd{short}] specifies to use a numeric date.
  
     \item[\kwd{long}] specifies to format months and weekdays as
       words instead of numbers.
  
     \item[\kwd{abbreviated}] is similar to long except the words are
       abbreviated.
  
     \item[\kwd{government}] is similar to abbreviated, except the
       date is of the form ``day month year'' instead of ``month day,
       year''.
     \end{Lentry}
     
   \item[\kwd{date-first}] if non-\false{} (default) will place the
     date first.  Otherwise, the time is placed first.
  
   \item[\kwd{print-seconds}] if non-\false{} (default) will format
     the seconds as part of the time.  Otherwise, the seconds will be
     omitted.
  
   \item[\kwd{print-meridian}] if non-\false{} (default) will format
     ``AM'' or ``PM'' as part of the time.  Otherwise, the ``AM'' or
     ``PM'' will be omitted.
  
   \item[\kwd{print-timezone}] if non-\false{} (default) will format
     the time zone as part of the time.  Otherwise, the time zone will
     be omitted.

     %%\item[\kwd{print-seconds}]
     %%if non-\false{} (default) will format the seconds as part of
     %%the time.  Otherwise, the seconds will be omitted.
  
   \item[\kwd{print-weekday}] if non-\false{} (default) will format
     the weekday as part of date.  Otherwise, the weekday will be
     omitted.
   \end{Lentry}
\end{defun}


\section{Random Number Generation}
\cindex{random number generation}

\clisp{} includes a random number generator as a standard part of the
language; however, the implementation of the generator is not
specified.

\subsection{MT-19937 Generator}
\cpsubindex{random number generation}{MT-19937 generator}
On all platforms, the random number is \code{MT-19937} generator as indicated by
\kwd{rand-mt19937} being in \code{*features*}.  This is a Lisp
implementation of the MT-19937 generator of Makoto Matsumoto and
T. Nishimura.  We refer the reader to their paper\footnote{``Mersenne
  Twister: A 623-Dimensionally Equidistributed Uniform Pseudorandom
  Number Generator,'' ACM Trans. on Modeling and Computer Simulation,
  Vol. 8, No. 1, January 1998, pp.3--30} or to
their
\ifpdf
\href{http://www.math.sci.hiroshima-u.ac.jp/~m-mat/MT/emt.html}{website}.
\else
website at
\href{http://www.math.keio.ac.jp/~matumoto/emt.html}{\texttt{http://www.math.keio.ac.jp/~matsumoto/emt.html}}.
\fi

When \cmucl{} starts up, \code{*random-state*} is initialized by
reading 627 words from \code{/dev/urandom}, when available.  If
\code{/dev/urandom} is not available, the universal time is used to
initialize \code{*random-state*}.  The initialization is done as given
in Matsumoto's paper.

\section{Lisp Threads}
\cindex{lisp threads}

\cmucl{} supports Lisp threads for the x86 platform.

\section{Lisp Library}
\label{lisp-lib}

The \cmucl{} project maintains a collection of useful or interesting
programs written by users of our system.  The library is in
\file{lib/contrib/}.  Two files there that users should read are:
\begin{Lentry}

\item[CATALOG.TXT]
This file contains a page for each entry in the library.  It
contains information such as the author, portability or dependency issues, how
to load the entry, etc.

\item[READ-ME.TXT]
This file describes the library's organization and all the
possible pieces of information an entry's catalog description could contain.
\end{Lentry}

Hemlock has a command \F{Library Entry} that displays a list of the current
library entries in an editor buffer.  There are mode specific commands that
display catalog descriptions and load entries.  This is a simple and convenient
way to browse the library.


\section{Generalized Function Names}

\begin{defmac}{ext:}{define-function-name-syntax}{%
    \var{name} (\var{var}) \ampbody\ \var{body}}
  Define lists starting with the symbol \code{name} as a new extended
  function name syntax.
  
  \code{body} is executed with \code{var} bound to an actual function
  name of that form, and should return two values:

  \begin{itemize}
  \item A generalized boolean that is true if \code{var} is a valid
    function name.
  \item A symbol that can be used as a \code{block} name in functions
    whose name is \code{var}.  (For some sorts of function names it
    might make sense to return \code{nil} for the block name, or just
    return one value.)
  \end{itemize}
  
  Users should not define function names starting with a symbol that
  \cmucl{} might be using internally.  It is therefore advisable to
  only define new function names starting with a symbol from a
  user-defined package.
\end{defmac}

\begin{defun}{ext:}{valid-function-name-p}{\var{name}}
  Returns two values:

  \begin{itemize}
  \item True if \code{name} is a valid function name.
  \item A symbol that can be used as a \code{block} name in
    functions whose name is \code{name}.  This can be \code{nil}
    for some function names.
  \end{itemize}
\end{defun}



\section{CLOS}

\subsection{Primary Method Errors}
\cindex{primary method}

The standard requires that an error is signaled when a generic
function is called and

\begin{itemize}
\item no primary method is applicable to the generic function's actual
  arguments, and
\item the generic function's method combination is either the standard
  method combination or a method combination defined with the short
  form of \code{define-method-combination}.  The latter includes the
  standardized method combinations like \code{progn}, \code{and}, etc.
\end{itemize}

\begin{defgeneric}[-generic]{pcl:}{no-primary-method}{\var{gf} \amprest{} \var{args}}
  In \cmucl, this generic function is called in the above erroneous
  cases.  The parameter \code{gf} is the generic function being
  called, and \code{args} is a list of actual arguments in the generic
  function call.
\end{defgeneric}

\begin{defmethod}[-standard]{pcl:}{no-primary-method}{%
    (\var{gf} \argtype{standard-generic-function}) \amprest{} \var{args}}
  This method signals a continuable error of type
  \code{pcl:no-primary-method-error}.
\end{defmethod}


\subsection{Slot Type Checking}
\cindex{slot type checking}

Declared slot types are used when 

\begin{itemize}
\item reading slot values with \code{slot-value} in methods, or

\item setting slots with \code{(setf slot-value)} in methods, or 
  
\item creating instances with \code{make-instance}, when slots are
  initialized from initforms.  This currently depends on PCL being
  able to use its internal \code{make-instance} optimization, which it
  usually can.
\end{itemize}

Example:

\begin{example}
(defclass foo ()
  ((a :type fixnum)))

(defmethod bar ((object foo) value)
  (with-slots (a) object
    (setf a value)))

(defmethod baz ((object foo))
  (< (slot-value object 'a) 10))
\end{example}

In method \code{bar}, and with a suitable safety setting, a type error
will occur if \code{value} is not a \code{fixnum}.  In method
\code{baz}, a \code{fixnum} comparison can be used by the compiler.
  
\begin{defvar}{pcl::}{use-slot-types-p}
  Slot type checking can be turned off by setting this variable to
  \false, which can be useful for compiling code containing incorrect
  slot type declarations.
\end{defvar}


\subsection{Slot Access Optimization}
\cindex{slot access optimization}
\cindex{slot declarations}

The declaration \code{ext:slots} is used for optimizing slot access in
methods.

\begin{example}
declare (ext:slots specifier*)

specifier   ::= (quality class-entry*)
quality     ::= SLOT-BOUNDP | INLINE
class-entry ::= class | (class slot-name*)
class       ::= the name of a class
slot-name   ::= the name of a slot
\end{example}

The \code{slot-boundp} quality specifies that all or some slots of a
class are always bound.

The \code{inline} quality specifies that access to all or some slots
of a class should be inlined, using compile-time knowledge of class
layouts.



\subsubsection{\code{slot-boundp} Declaration}
\cpsubindex{slot declaration}{slot-boundp}

Example:

\begin{example}
(defclass foo ()
  (a b))

(defmethod bar ((x foo))
  (declare (ext:slots (slot-boundp foo)))
  (list (slot-value x 'a) (slot-value x 'b)))
\end{example}

The \code{slot-boundp} declaration in method \code{bar} specifies that
the slots \code{a} and \code{b} accessed through parameter \code{x} in
the scope of the declaration are always bound, because parameter
\code{x} is specialized on class \code{foo} to which the
\code{slot-boundp} declaration applies.  The PCL-generated code for
the \code{slot-value} forms will thus not contain tests for the slots
being bound or not.  The consequences are undefined should one of the
accessed slots not be bound.



\subsubsection{\code{inline} Declaration}
\cpsubindex{slot declaration}{inline}

Example:

\begin{example}
(defclass foo ()
  (a b))

(defmethod bar ((x foo))
  (declare (ext:slots (inline (foo a))))
  (list (slot-value x 'a) (slot-value x 'b)))
\end{example}

The \code{inline} declaration in method \code{bar} tells PCL to use
compile-time knowledge of slot locations for accessing slot \code{a}
of class \code{foo}, in the scope of the declaration.

Class \code{foo} must be known at compile time for this optimization
to be possible.  PCL prints a warning and uses normal slot access If
the class is not defined at compile time.

If a class is \code{proclaim}ed to use inline slot access before it is
defined, the class is defined at compile time.  Example:

\begin{example}
(declaim (ext:slots (inline (foo slot-a))))
(defclass foo () ...)
(defclass bar (foo) ...)
\end{example}
  
Class \code{foo} will be defined at compile time because it is
declared to use inline slot access; methods accessing slot
\code{slot-a} of \code{foo} will use inline slot access if otherwise
possible.  Class \code{bar} will be defined at compile time because
its superclass \code{foo} is declared to use inline slot access.  PCL
uses compile-time information from subclasses to warn about situations
where using inline slot access is not possible.

Normal slot access will be used if PCL finds, at method compilation
time, that

\begin{itemize}
\item class \code{foo} has a subclass in which slot \code{a} is at a
  different location, or

\item there exists a \code{slot-value-using-class} method for
  \code{foo} or a subclass of \code{foo}.
\end{itemize}
  
When the declaration is used to optimize calls to slot accessor
generic functions in methods, as opposed to \code{slot-value} or
\code{(setf slot-value)}, the optimization is additionally not used if

\begin{itemize}
\item there exist, at compile time, applicable methods on the
  reader/writer generic function that are not standard accessor
  methods (for instance, there exist around-methods), or
  
\item applicable reader/writer methods access different slots in a
  class accessed inline, and one of its subclasses.
\end{itemize}

The consequences are undefined if the compile-time environment is not
the same as the run-time environment in these respects, or if the
definition of class \code{foo} or any subclass of \code{foo} is
changed in an incompatible way, that is, if slot locations change.

The effect of the \code{inline} optimization combined with the
\code{slot-boundp} optimization is that CLOS slot access becomes as
fast as structure slot access, which is an order of magnitude faster
than normal CLOS slot access.

\begin{defvar}{pcl::}{optimize-inline-slot-access-p}
  This variable controls if inline slot access optimizations are
  performed.  It is true by default.
\end{defvar}



\subsubsection{Automatic Method Recompilation}
\cindex{methods}
\cpsubindex{methods}{auto-compilation}
\cpsubindex{slot declaration}{method recompilation}
  
Methods using inline slot access can be automatically recompiled after
class changes.  Two declarations control which methods are
automatically recompiled.

\begin{example}
declaim (ext:auto-compile specifier*)
declaim (ext:not-auto-compile specifier*)

specifier   ::= gf-name | (gf-name qualifier* (specializer*))
gf-name     ::= the name of a generic function
qualifier   ::= a method qualifier
specializer ::= a method specializer
\end{example}

If no specifier is given, auto-compilation is by default done/not done
for all methods of all generic functions using inline slot access;
current default is that it is not done.  This global policy can be
overridden on a generic function and method basis.  If
\code{specifier} is a generic function name, it applies to all methods
of that generic function.

Examples:

\begin{example}
(declaim (ext:auto-compile foo))
(defmethod foo :around ((x bar)) ...)
\end{example}

The around-method \code{foo} will be automatically recompiled because
the declamation applies to all methods with name \code{foo}.

\begin{example}
(declaim (ext:auto-compile (foo (bar))))
(defmethod foo :around ((x bar)) ...)
(defmethod foo ((x bar)) ...)
\end{example}

The around-method will not be automatically recompiled, but the
primary method will.

\begin{example}
(declaim (ext:auto-compile foo))
(declaim (ext:not-auto-compile (foo :around (bar)))  
(defmethod foo :around ((x bar)) ...)
(defmethod foo ((x bar)) ...)
\end{example}

The around-method will not be automatically recompiled, because it
is explicitly declaimed not to be.  The primary method will be
automatically recompiled because the first declamation applies to
it.

Auto-recompilation works by recording method bodies using inline slot
access.  When PCL determines that a recompilation is necessary, a
\code{defmethod} form is constructed and evaluated.

Auto-compilation can only be done for methods defined in a null
lexical environment.  PCL prints a warning and doesn't record the
method body if a method using inline slot access is defined in a
non-null lexical environment.  Instead of doing a recompilation on
itself, PCL will then print a warning that the method must be
recompiled manually when classes are changed.



\subsection{Inlining Methods in Effective Methods}
\cindex{effective method}
\cpsubindex{methods}{inlining in effective methods}
\cpsubindex{effective method}{inlining of methods}
\cindex{inline}

When a generic function is called, an effective method is constructed
from applicable methods.  The effective method is called with the
original arguments, and itself calls applicable methods according to
the generic function's method combination.  Some of the function call
overhead in effective methods can be removed by inlining methods in
effective methods, at the expense of increased code size.

Inlining of methods is controlled by the usual \code{inline}
declaration.  In the following example, both \code{foo} methods shown
will be inlined in effective methods:

\begin{example}
(declaim (inline (method foo (foo))
                 (method foo :before (foo))))
(defmethod foo ((x foo)) ...)
(defmethod foo :before ((x foo)) ...)
\end{example}

Please note that this form of inlining has no noticeable effect for
effective methods that consist of a primary method only, which doesn't
have keyword arguments.  In such cases, PCL uses the primary method
directly for the effective method.

When the definition of an inlined method is changed, effective methods
are \textbf{not} automatically updated to reflect the change.  This is
just as it is when inlining normal functions.  Different from the
normal case is that users do not have direct access to effective
methods, as it would be the case when a function is inlined somewhere
else.  Because of this, the function \code{pcl:flush-emf-cache} is
provided for forcing such an update of effective methods.

\begin{defun}{pcl:}{flush-emf-cache}{\ampoptional{} \var{gf}}
  Flush cached effective method functions.  If \code{gf} is supplied,
  it should be a generic function metaobject or the name of a generic
  function, and this function flushes all cached effective methods for
  the given generic function.  If \code{gf} is not supplied, all
  cached effective methods are flushed.
\end{defun}

\begin{defvar}{pcl::}{inline-methods-in-emfs}
  If true, the default, perform method inlining as described above.
  If false, don't.
\end{defvar}



\subsection{Effective Method Precomputation}
\cpsubindex{effective method}{precomputation}
\cpsubindex{methods}{load time}
\cpsubindex{methods}{emf precomputation}

When a generic function is called, the generic function's
discriminating function computes the set of methods applicable to
actual arguments and constructs an effective method function from
applicable methods, using the generic function's method combination.

Effective methods can be precomputed at method load time instead of
when the generic function is called depending on the value of
\code{pcl:*max-emf-precomputation-methods*}.

\begin{defvar}{pcl:}{*max-emf-precomputation-methods*}
  If nonzero, the default value is 100, precompute effective methods
  when methods are loaded, and the method's generic function has less
  than the specified number of methods.
  
  If zero, compute effective methods only when the generic function is
  called.
\end{defvar}



\subsection{Sealing}
\cindex{sealing}
\cpsubindex{sealing}{subclasses}
\cpsubindex{sealing}{methods}
\cpsubindex{methods}{sealing}

Support for sealing classes and generic functions have been
implemented.  Please note that this interface is subject to change.

\begin{defmac}{pcl:}{seal}{\var{name} (\var{var}) \amprest{} \var{specifiers}}
  Seal \code{name} with respect to the given specifiers; \code{name}
  can be the name of a class or generic-function.

  Supported specifiers are \kwd{subclasses} for classes,
  which prevents changing subclasses of a class, and \kwd{methods}
  which prevents changing the methods of a generic function.
  
  Sealing violations signal an error of type \code{pcl:sealed-error}.
\end{defmac}

\begin{defun}{pcl:}{unseal}{\var{name-or-object}}
  Remove seals from \code{name-or-object}.
\end{defun}



\subsection{Method Tracing and Profiling}
\label{sec:method-tracing}
\cindex{tracing}
\cpsubindex{tracing}{methods}
\cindex{profiling}
\cpsubindex{profiling}{methods}
\cpsubindex{methods}{tracing}
\cpsubindex{methods}{profiling}

Methods can be traced with \code{trace}, using function names of the
form \code{(method <name> <qualifiers> <specializers>)}.  Example:

\begin{example}
(defmethod foo ((x integer)) x)
(defmethod foo :before ((x integer)) x)

(trace (method foo (integer)))
(trace (method foo :before (integer)))
(untrace (method foo :before (integer)))
\end{example}
  
\code{trace} and \code{untrace} also allow a name specifier
\code{:methods gf-form} for tracing all methods of a generic function:

\begin{example}
(trace :methods 'foo)
(untrace :methods 'foo)
\end{example}

Methods can also be specified for the \kwd{wherein} option to
\code{trace}.  Because this option is a name or a list of names,
methods must be specified as a list.  Thus, to trace all calls of
\code{foo} from the method \code{bar} specialized on integer argument,
use
\begin{example}
  (trace foo :wherein ((method bar (integer))))
\end{example}
Before and after methods are supported as well:
\begin{example}
  (trace foo :wherein ((method bar :before (integer))))
\end{example}

Method profiling is done analogously to \code{trace}:

\begin{example}
(defmethod foo ((x integer)) x)
(defmethod foo :before ((x integer)) x)

(profile:profile (method foo (integer)))
(profile:profile (method foo :before (integer)))
(profile:unprofile (method foo :before (integer)))

(profile:profile :methods 'foo)
(profile:unprofile :methods 'foo)

(profile:profile-all :methods t)
\end{example}



\subsection{Misc}
\cpsubindex{methods}{interpreted}

\begin{defvar}{pcl::}{compile-interpreted-methods-p}
  This variable controls compilation of interpreted method functions,
  e.g. for methods defined interactively at the REPL.  Default is
  true, that is, method functions are compiled.
\end{defvar}





\section{Differences from ANSI Common Lisp}
This section describes some of the known differences between \cmucl{}
and ANSI \clisp{}.  Some may be non-compliance issues; same may be
extensions.

\subsection{Extensions}

\begin{defun}{}{constantly}{%
    \var{value} \ampoptional{} \var{val1} \var{val2} \amprest{} \var{more-values}}
  As an extension, \cmucl{} allows \code{constantly} to accept more
  than one value which are returned as multiple values.
\end{defun}




\section{Function Wrappers}
\cindex{function wrappers}
\cindex{fwrappers}

Function wrappers, fwrappers for short, are a facility for efficiently
encapsulating functions\footnote{This feature was independently
developed, but the interface is modelled after a similar feature in
Allegro.  Some names, however, have been changed.}.

Functions in \cmucl{} are represented by \code{kernel:fdefn}
objects.  Each \code{fdefn} object contains a reference to its
function's actual code, which we call the function's primary function.

A function wrapper replaces the primary function in the \code{fdefn}
object with a function of its own, and records the original function
in an fwrapper object, a funcallable instance.  Thus, when the
function is called, the fwrapper gets called, which in turn might call
the primary function, or a previously installed fwrapper that was
found in the \code{fdefn} object when the second fwrapper was
installed.

Example:

\begin{lisp}
(use-package :fwrappers)

(define-fwrapper foo (x y)
  (format t "x = ~s, y = ~s, user-data = ~s~%"
          x y (fwrapper-user-data fwrapper))
  (let ((value (call-next-function)))
    (format t "value = ~s~%" value)
    value))

(defun bar (x y)
  (+ x y))

(fwrap 'bar #'foo :type 'foo :user-data 42)

(bar 1 2)
 =>
 x = 1, y = 2, user-data = 42
 value = 3
 3   
\end{lisp}

Fwrappers are used in the implementation of \code{trace} and
\code{profile}.

Please note that \code{fdefinition} always returns the primary
definition of a function; if a function is fwrapped,
\code{fdefinition} returns the primary function stored in the
innermost fwrapper object.  Likewise, if a function is fwrapped,
\code{(setf fdefinition)} will set the primary function in the
innermost fwrapper.

\begin{defmac}{fwrappers:}{define-fwrapper}{\var{name} \var{lambda-list} \ampbody{} \var{body}}
  This macro is like \code{defun}, but defines a function named
  \var{name} that can be used as an fwrapper definition.
  
  In \var{body}, the symbol \code{fwrapper} is bound to the current
  fwrapper object.
  
  The macro \code{call-next-function} can be used to invoke the next
  fwrapper, or the primary function that is being fwrapped.  When
  called with no arguments, \code{call-next-function} invokes the next
  function with the original arguments passed to the fwrapper, unless
  you modify one of the parameters.  When called with arguments,
  \code{call-next-function} invokes the next function with the given
  arguments.
\end{defmac}

\begin{defun}{fwrappers:}{fwrap}{\var{function-name} \var{fwrapper} %
    \keys{\kwd{type} \kwd{user-data}}}
  This function wraps function \code{function-name} in an fwrapper
  \var{fwrapper} which was defined with \code{define-fwrapper}.

  The value of \var{type}, if supplied, is used as an identifying
  tag that can be used in various other operations.
  
  The value of \var{user-data} is stored as user-supplied data in the
  fwrapper object that is created for the function encapsulation.
  User-data is accessible in the body of fwrappers defined with
  \code{define-fwrapper} as \code{(fwrapper-user-data fwrapper)}.

  Value is the fwrapper object created.
\end{defun}

\begin{defun}{fwrappers:}{funwrap}{\var{function-name} \keys{\kwd{type} \kwd{test}}}
  Remove fwrappers from the function named \var{function-name}.  If
  \var{type} is supplied, remove fwrappers whose type is \code{equal}
  to \var{type}.  If \var{test} is supplied, remove fwrappers
  satisfying \var{test}.
\end{defun}

\begin{defun}{fwrappers:}{find-fwrapper}{\var{function-name} \keys{\kwd{type} \kwd{test}}}
  Find an fwrapper of \var{function-name}.  If \var{type} is supplied,
  find an fwrapper whose type is \code{equal} to \var{type}.  If
  \var{test} is supplied, find an fwrapper satisfying \var{test}.
\end{defun}

\begin{defun}{fwrappers:}{update-fwrapper}{\var{fwrapper}}
  Update the funcallable instance function of the fwrapper object
  \var{fwrapper} from the definition of its function that was 
  defined with \code{define-fwrapper}.  This can be used to update
  fwrappers after changing a \code{define-fwrapper}.
\end{defun}

\begin{defun}{fwrappers:}{update-fwrappers}{\var{function-name} \keys{\kwd{type} \kwd{test}}}
  Update fwrappers of \var{function-name}; see \code{update-fwrapper}.
  If \var{type} is supplied, update fwrappers whose type is
  \code{equal} to \var{type}.  If \var{test} is supplied, update fwrappers
  satisfying \var{test}.
\end{defun}

\begin{defun}{fwrappers:}{set-fwrappers}{\var{function-name} \var{fwrappers}}
  Set \var{function-names}'s fwrappers to elements of the list
  \var{fwrappers}, which is assumed to be ordered from outermost to
  innermost.  \var{fwrappers} null means remove all fwrappers.
\end{defun}

\begin{defun}{fwrappers:}{list-fwrappers}{\var{function-name}}
  Return a list of all fwrappers of \var{function-name}, ordered
  from outermost to innermost.
\end{defun}

\begin{defun}{fwrappers:}{push-fwrapper}{\var{fwrapper} \var{function-name}}
  Prepend fwrapper \var{fwrapper} to the definition of
  \var{function-name}.  Signal an error if \var{function-name} is an
  undefined function.
\end{defun}

\begin{defun}{fwrappers:}{delete-fwrapper}{\var{fwrapper} \var{function-name}}
  Remove fwrapper \var{fwrapper} from the definition of
  \var{function-name}.  Signal an error if \var{function-name} is an
  undefined function.
\end{defun}

\begin{defmac}{fwrappers:}{do-fwrappers}{(\var{var} \var{fdefn} \ampoptional{}
  \var{result}) \ampbody{} \var{body}}
  Evaluate \var{body} with \var{var} bound to consecutive fwrappers of
  \var{fdefn}.  Return \var{result} at the end.  Note that \var{fdefn}
  must be an \code{fdefn} object.  You can use
  \code{kernel:fdefn-or-lose}, for instance, to get the \code{fdefn}
  object from a function name.
\end{defmac}

\section{Dynamic-Extent Declarations}
\cindex{dynamic-extent}

\emph{Note:  As of the 19a release, \code{dynamic-extent} is
  unfortunately disabled by default.  It is known to cause some issues
  with CLX and Hemlock.  The cause is not known, but causes random
  errors and brokeness.  Enable at your own risk.  However, it is safe
  enough to build all of CMUCL without problems.}

On x86 and sparc, \cmucl{} can exploit \code{dynamic-extent}
declarations by allocating objects on the stack instead of the heap.

You can tell \cmucl{} to trust or not trust \code{dynamic-extent}
declarations by setting the variable
\var{*trust-dynamic-extent-declarations*}.

\begin{defvar}{ext:}{trust-dynamic-extent-declarations}
  If the value of \var{*trust-dynamic-extent-declarations*} is 
  \code{NIL}, \code{dynamic-extent} declarations are effectively
  ignored.

  If the value of this variable is a function, the function is called
  with four arguments to determine if a \code{dynamic-extent} 
  declaration should be trusted.  The arguments are the safety,
  space, speed, and debug settings at the point where the 
  \code{dynamic-extent} declaration is used.  If the function
  returns true, the declaration is trusted, otherwise it is not
  trusted.

  In all other cases, \code{dynamic-extent} declarations are
  trusted.
\end{defvar}

Please note that stack-allocation is inherently unsafe.  If you make a
mistake, and a stack-allocated object or part of it escapes, \cmucl{}
is likely to crash, or format your hard disk.

\subsection{\code{\&rest} argument lists}
\cpsubindex{dynamic-extent}{rest lists}

Rest argument lists can be allocated on the stack by declaring the
rest argument variable \code{dynamic-extent}.  Examples:

\begin{lisp}
(defun foo (x &rest rest)
  (declare (dynamic-extent rest))
  ...)

(defun bar ()
  (lambda (&rest rest)
    (declare (dynamic-extent rest))
    ...))
\end{lisp}

\subsection{Closures}
\cpsubindex{dynamic-extent}{closures}

Closures for local functions can be allocated on the stack if the
local function is declared \code{dynamic-extent}, and the closure
appears as an argument in the call of a named function.  In the
example:

\begin{lisp}
(defun foo (x)
  (flet ((bar () x))
    (declare (dynamic-extent #'bar))
    (baz #'bar)))
\end{lisp}

the closure passed to function \code{baz} is allocated on the stack.
Likewise in the example:

\begin{lisp}
(defun foo (x)
  (flet ((bar () x))
    (baz #'bar)
    (locally (declare (dynamic-extent #'bar))
      (baz #'bar))))
\end{lisp}

\cpsubindex{dynamic-extent}{known CL functions}

Stack-allocation of closures can also automatically take place when
calling certain known CL functions taking function arguments, for
example \code{some} or \code{find-if}.

\subsection{\code{list}, \code{list*}, and \code{cons}}
\cpsubindex{dynamic-extent}{list, list*, cons}

New conses allocated by \code{list}, \code{list*}, or \code{cons}
which are used to initialize variables can be allocated from the stack
if the variables are declared \code{dynamic-extent}.  In the case of
\code{cons}, only the outermost cons cell is allocated from the stack;
this is an arbitrary restriction.

\begin{lisp}
(let ((x (list 1 2))
      (y (list* 1 2 x))
      (z (cons 1 (cons 2 nil))))
  (declare (dynamic-extent x y z))
  ...
  (setq x (list 2 3))
  ...)
\end{lisp}

Please note that the \code{setq} of \code{x} in the example program
assigns to \code{x} a list that is allocated from the heap.  This is
another arbitrary restriction that exists because other Lisps behave
that way.

\section{Modular Arithmetic}
\cindex{modular-arith}

This section is mostly taken, with permission,  from the documentation
for SBCL.

Some numeric functions have a property: \code{N} lower bits of
the result depend only on \code{N} lower bits of (all or some)
arguments. If the compiler sees an expression of form \code{(logand
exp mask)}, where \code{exp} is a tree of such ``good'' functions
and \code{mask} is known to be of type \code{(unsigned-byte
w)}, where \code{w} is a "good" width, all intermediate results
will be cut to \code{w} bits (but it is not done for variables
and constants!). This often results in an ability to use simple
machine instructions for the functions.

Consider an example.
\begin{lisp}
(defun i (x y)
  (declare (type (unsigned-byte 32) x y))
  (ldb (byte 32 0) (logxor x (lognot y))))
\end{lisp}
The result of \code{(lognot y)} will be negative and of
type \code{(signed-byte 33)}, so a naive implementation on a 32-bit
platform is unable to use 32-bit arithmetic here. But modular
arithmetic optimizer is able to do it: because the result is cut down
to 32 bits, the compiler will replace \code{logxor}
and \code{lognot} with versions cutting results to 32 bits, and
because terminals (here---expressions \code{x} and \code{y})
are also of type \code{(unsigned-byte 32)}, 32-bit machine
arithmetic can be used.


Currently ``good'' functions
are \code{+}, \code{-}, \code{*}; \code{logand}, \code{logior},
\code{logxor}, \code{lognot} and their combinations;
and \code{ash} with the positive second argument. ``Good'' widths
are 32 on HPPA, MIPS, PPC, Sparc and X86 and 64 on Alpha. While it is
possible to support smaller widths as well, currently it is not
implemented.

A more extensive description of modular arithmetic can be found in the
paper ``Efficient Hardware Arithmetic in Common Lisp'' by Alexey
Dejneka, and Christophe Rhodes, to be published.

\section{Extension to REQUIRE}
\cindex{require}

The behavior of \code{require} when called with only one argument is
implementation-defined.  In \cmucl, functions from the list
\var{*module-provider-functions*} are called in order with the
stringified module name as the argument.  The first function to return
non-\var{NIL} is assumed to have loaded the module.

By default the functions \code{module-provide-cmucl-defmodule} and
\code{module-provide- cmucl-library} are on this list of functions, in
that order.

\begin{defvar}{ext:}{module-provider-functions}
  This is a list of functions taking a single argument.
  \code{require} calls each function in turn with the stringified
  module name.  The first function to return non-\var{NIL} indicates
  that the module has been loaded.  The remaining functions, if any,
  are not called.

  To add new providers, push the new provider function onto the
  beginning of this list.
\end{defvar}

\begin{defmac}{ext:}{defmodule}{\var{name} \amprest{} \var{files}}
  Defines a module by registering the files that need to be loaded
  when the module is required.  If \var{name} is a symbol, its print
  name is used after downcasing it.
\end{defmac}

\begin{defun}{ext:}{module-provide-cmucl-defmodule}{\var{module-name}}
  This function is the module-provider for modules registered by a
  \code{ext:defmodule} form.  
\end{defun}

\begin{defun}{ext:}{module-provide-cmucl-library}{\var{module-name}}
  This function is the module-provider for \cmucl's libraries,
  including Gray streams, simple streams, CLX, CLM, Hemlock,
  \emph{etc}.
  
  This function causes a file to be loaded whose name is formed by
  merging the search-list ``modules:'' and the concatenation of
  module-name with the suffix ``-LIBRARY''.  Note that both the
  module-name and the suffix are each, separately, converted from
  :case :common to :case :local.  This merged name will be probed with
  both a .lisp and .fasl extensions, calling \code{LOAD} if it exists.
\end{defun}


\section{Localization}
\label{sec:localization}

\cmucl{} support localization where messages can be presented in the
native language.  This is done in the style of \code{gettext} which
marks strings that are to be translated and provides the lookup to
convert the string to the specified language.

All messages from \cmucl{} can be translated but as of this writing,
the only complete translation is a Pig Latin translation done by
machine.  There are a few messages translated to Korean.

In general, translatable strings are marked as such by using the
functions \code{intl:gettext} and \code{intl:ngettext} or by using the
reader macros \verb+_+ or \verb+_N+.  When loading or compiling, such
strings are recorded for translation.  At runtime, such strings are
looked in and the translation is returned.  Doc strings do not need to
be noted in any way; the are automatically noted for translation.

By default, recording of translatable strings is disabled.  To enable
recording of strings, call \code{intl:translation-enable}.

\subsection{Dictionary}
\label{sec:localization-dictionary}

\begin{defun}{intl:}{translation-enable}{}
  Enable recording of translatable strings.
\end{defun}

\begin{defun}{intl:}{translation-disable}{}
  Disablle recording of translatable strings.
\end{defun}

\begin{defun}{intl:}{setlocale}{\ampoptional{} \var{locale}}
  Sets the locale to the locale specified by \var{locale}.  If
  \var{locale} is not give or is \nil, the locale is determined by
  look at the environment variables \code{LANGUAGE}, \code{LC\_ALL},
  \code{LC\_MESSAGES}, or \code{LANG}.  If none of these are set, the
  locale is unchanged.

  The default locale is ``C''.
\end{defun}

\begin{defun}{intl:}{textdomain}{\var{domain}}
  Set the default domain to the domain specified by \var{domain}.
  Typically,  this only needs to be done at the top of each source
  file.  This is used to \code{gettext} and \code{ngettext} to set the
  domain for the message string.
\end{defun}

\begin{defmac}{intl:}{gettext}{\var{string}}
  Look up the specified string, \var{string}, in the current message
  domain and return its translation.
\end{defmac}

\begin{defun}{intl:}{dgettext}{\var{domain} \var{string}}
  Look up the specified string, \var{string}, in the message domain,
  \var{domain}.  The translation is returned.

  When compiled, this also function also records the string so that an
  appropriate message template file can be created.  (See
  \code{intl::dump-pot-files}.) 
\end{defun}

\begin{defmac}{intl:}{ngettext}{\var{singular} \var{plural} \var{n}}
  Look up the singular or plural form of a message in the default
  domain.  The singular form is \var{singular}; the plural is
  \var{plural}.  The number of items is specified by \var{n} in case
  the correct translation depends on the actual number of items.
\end{defmac}

\begin{defun}{intl:}{dngettext}{\var{domain} \var{singular} \var{plural} \var{n}}
  Look up the singular or plural form of a message in the specified
  domain, \var{domain}.  The singular form is \var{singular}; the
  plural is \var{plural}.  The number of items is specified by \var{n}
  in case the correct translation depends on the actual number of
  items.

  When compiled, this also function also records the singular and
  plural forms so that an appropriate message template file can be
  created.  (See \code{intl::dump-pot-files}.)
\end{defun}

\begin{defun}{intl::}{dump-pot-files}{\keys{\kwd{copyright} \kwd{output-directory}}}
  Dumps the translatable strings recorded by \code{dgettext} and
  \code{dngettext}.  The message template file (pot file) is written
  to a file in the directory specified by \var{output-directory}, and
  the name of the file is the domain of the string.

  If \var{copyright} is specified, this is placed in the output file
  as the copyright message.
\end{defun}

\begin{defvar}{intl:}{locale-directories}
  This is a list of directory pathnames where the translations can be found.
\end{defvar}  

\begin{defun}{intl:}{install}{\ampoptional{} (\var{rt} \var{*readtable*})}
  Installs reader macros and comment reader into the specified
  readtable as explained below.  The readtable defaults to
  \var{*readtable*}.
\end{defun}

Two reader macros are also provided: \code{\_''} and \code{\_N''}.  The
first is equivalent to wrapping \code{dgettext} around the string.
The second returns the string, but also records the string.  This is
needed when we want to record a docstring for translation or any other
string in a place where a macro or function call would be incorrect.

Also, the standard comment reader is extended to allow translator
comments to be saved and written to the messages template file so that
the translator may not need to look at the original source to
understand the string.  Any comment line that begins with exactly
\verb|"TRANSLATORS: "| is saved.  This means each translator comment
must be preceded by this string to be saved; the translator comment
ends at the end of each line.


\subsection{Example Usage}
\label{sec:localization-usage}

Here is a simple example of how to localize your code.  Let the file
\code{intl-ex.lisp} contain:

\begin{example}

(intl:textdomain "example")  

(defun foo (x y)
  "Cool function foo of x and y"
  (let ((result (bar x y)))
    ;; TRANSLATORS:  One line comment about bar.
    (format t _"bar of ~A and ~A = ~A~%" x y result)
    #| TRANSLATORS:  Multiline comment about
    how many Xs there are
    |#
    (format t (intl:ngettext "There is one X"
                             "There are many Xs"
                             x))
    result))
\end{example}

The call to \code{textdomain} sets the default domain for all
translatable strings following the call.

Here is a sample session for creating a template file:

\begin{example}
* (intl:install)

T
* (intl:translation-enable)

T
* (compile-file "intl-ex")

#P"/Volumes/share/cmucl/cvs/intl-ex.sse2f"
NIL
NIL
* (intl::dump-pot-files :output-directory "./")

Dumping 3 messages for domain "example"
NIL
*
\end{example}

When this file is compiled, all of the translatable strings are
recorded.  This includes the docstring for \code{foo}, the string for
the first \code{format}, and the string marked by the call to
\code{intl:ngettext}.

A file named ``example.pot'' in the directory ``./'' is created.
The contents of this file are:
\begin{example}
#@ example

# SOME DESCRIPTIVE TITLE
# FIRST AUTHOR <EMAIL@ADDRESS>, YEAR
#
#, fuzzy
msgid ""
msgstr ""
"Project-Id-Version: PACKAGE VERSION"
"Report-Msgid-Bugs-To: "
"PO-Revision-Date: YEAR-MO-DA HO:MI +ZONE"
"Last-Translator: FULL NAME <EMAIL@ADDRESS>"
"Language-Team: LANGUAGE <LL@li.org>"
"MIME-Version: 1.0"
"Content-Type: text/plain; charset=UTF-8"
"Content-Transfer-Encoding: 8bit"

#.  One line comment about bar.
#: intl-ex.lisp
msgid "bar of ~A and ~A = ~A~%"
msgstr ""

#.  Multiline comment about
    how many Xs there are
#: intl-ex.lisp
msgid "Cool function foo of x and y"
msgstr ""

#: intl-ex.lisp
msgid "There is one X"
msgid_plural "There are many Xs"
msgstr[0] ""

\end{example}

To finish the translation, a corresponding ``example.po'' file needs
to be created with the appropriate translations for the given
strings.  This file must be placed in some directory that is included
in \code{intl:*locale-directories*}.

Suppose the translation is done for Korean.  Then the user can set the
environment variables appropriately or call \code{(intl:setlocale
  "ko")}.  Note that the external format for the standard streams
needs to be set up appropriately too.  It is up to the user to set
this correctly.  Once this is all done, the output from the function
\code{foo} will now be in Korean instead of English as in the original
source file.

For further information, we refer the reader to documentation on
\ifpdf
\href{http://www.gnu.org/software/gettext/manual/gettext.html}{gettext}.
\else
gettext at
\href{http://www.gnu.org/software/gettext/manual/gettext.html}{\texttt{http://www.gnu.org/software/gettext/manual/gettext.html}}.
\fi

\section{Static Arrays}
\label{sec:static-arrays}

\cmucl{} supports static arrays which are arrays that are not moved by
the garbage collector.  To create such an array, use the
\kwd{allocation} option to \code{make-array} with a value of
\kwd{malloc}.  These arrays appear as normal Lisp arrays, but are
actually allocated from the \code{C} heap (hence the \kwd{malloc}).
Thus, the number and size of such arrays are limited by the available
\code{C} heap.

Also, only certain types of arrays can be allocated.  The static array
cannot be adjustable and cannot be displaced to.  The array must also
be a \code{simple-array} of one dimension.  The element type is also
constrained to be one of the types in
Table~\ref{tbl:static-array-types}.

\begin{table}[tbhp]
  \begin{center}
    \begin{tabular}{|c|}
      \hline
      \code{(unsigned-byte 8)} \\
      \hline
      \code{(unsigned-byte 16)} \\
      \hline
      \code{(unsigned-byte 32)} \\
      \hline
      \code{(signed-byte 8)} \\
      \hline
      \code{(signed-byte 16)} \\
      \hline
      \code{(signed-byte 32)} \\
      \hline
      \code{single-float} \\
      \hline
      \code{double-float} \\
      \hline
      \code{(complex single-float)} \\
      \hline
      \code{(complex double-float)} \\
      \hline
    \end{tabular}
    \caption{Allowed element types for static arrays}
    \label{tbl:static-array-types}
  \end{center}
\end{table}

The arrays are properly handled by GC.  GC will not move the arrays,
but they will be properly removed up if they become garbage.

\chapter{The Debugger}
\cindex{debugger}
\label{debugger}

\credits{by Robert MacLachlan}


\section{Debugger Introduction}

The \cmucl{} debugger is unique in its level of support for source-level
debugging of compiled code.  Although some other debuggers allow access of
variables by name, this seems to be the first \llisp{} debugger that:
\begin{itemize}

\item
Tells you when a variable doesn't have a value because it hasn't been
initialized yet or has already been deallocated, or

\item
Can display the precise source location corresponding to a code
location in the debugged program.
\end{itemize}
These features allow the debugging of compiled code to be made almost
indistinguishable from interpreted code debugging.

The debugger is an interactive command loop that allows a user to examine
the function call stack.  The debugger is invoked when:
\begin{itemize}

\item
A \tindexed{serious-condition} is signaled, and it is not handled, or

\item
\findexed{error} is called, and the condition it signals is not handled, or

\item
The debugger is explicitly invoked with the \clisp{} \findexed{break}
or \findexed{debug} functions.
\end{itemize}

{\it Note: there are two debugger interfaces in \cmucl{}: the TTY
debugger (described below) and the Motif debugger. Since the
difference is only in the user interface, much of this chapter also
applies to the Motif version. \xlref{motif-interface} for a very brief
discussion of the graphical interface.}

When you enter the TTY debugger, it looks something like this:

\begin{example}
Error in function CAR.
Wrong type argument, 3, should have been of type LIST.

Restarts:
  0: Return to Top-Level.

Debug  (type H for help)

(CAR 3)
0]
\end{example}

The first group of lines describe what the error was that put us in the
debugger.  In this case \code{car} was called on \code{3}.  After \code{Restarts:}
is a list of all the ways that we can restart execution after this error.  In
this case, the only option is to return to top-level.  After printing its
banner, the debugger prints the current frame and the debugger prompt.


\section{The Command Loop}

The debugger is an interactive read-eval-print loop much like the normal
top-level, but some symbols are interpreted as debugger commands instead
of being evaluated.  A debugger command starts with the symbol name of
the command, possibly followed by some arguments on the same line.  Some
commands prompt for additional input.  Debugger commands can be
abbreviated by any unambiguous prefix: \code{help} can be typed as
\code{h}, \code{he}, etc.  For convenience, some commands have
ambiguous one-letter abbreviations: \code{f} for \code{frame}.

The package is not significant in debugger commands; any symbol with the
name of a debugger command will work.  If you want to show the value of
a variable that happens also to be the name of a debugger command, you
can use the \code{list-locals} command or the \code{debug:var}
function, or you can wrap the variable in a \code{progn} to hide it from
the command loop.

The debugger prompt is ``\var{frame}\code{]}'', where \var{frame} is the number
of the current frame.  Frames are numbered starting from zero at the top (most
recent call), increasing down to the bottom.  The current frame is the frame
that commands refer to.  The current frame also provides the lexical
environment for evaluation of non-command forms.

\cpsubindex{evaluation}{debugger} The debugger evaluates forms in the lexical
environment of the functions being debugged.  The debugger can only
access variables.  You can't \code{go} or \code{return-from} into a
function, and you can't call local functions.  Special variable
references are evaluated with their current value (the innermost binding
around the debugger invocation)\dash{}you don't get the value that the
special had in the current frame.  \xlref{debug-vars} for more
information on debugger variable access.


\section{Stack Frames}
\cindex{stack frames} \cpsubindex{frames}{stack}

A stack frame is the run-time representation of a call to a function;
the frame stores the state that a function needs to remember what it is
doing.  Frames have:
\begin{itemize}

\item
Variables (\pxlref{debug-vars}), which are the values being operated
on, and

\item
Arguments to the call (which are really just particularly interesting
variables), and

\item
A current location (\pxlref{source-locations}), which is the place in
the program where the function was running when it stopped to call another
function, or because of an interrupt or error.
\end{itemize}


\subsection{Stack Motion}

These commands move to a new stack frame and print the name of the function
and the values of its arguments in the style of a Lisp function call:
\begin{Lentry}

\item[\code{up}]
Move up to the next higher frame.  More recent function calls are considered
to be higher on the stack.

\item[\code{down}]
Move down to the next lower frame.

\item[\code{top}]
Move to the highest frame.

\item[\code{bottom}]
Move to the lowest frame.

\item[\code{frame} [\textit{n}]]
Move to the frame with the specified number.  Prompts for the number if not
supplied.

% \key{S} [\var{function-name} [\var{n}]]
% 
% \item
% Search down the stack for function.  Prompts for the function name if not
% supplied.  Searches an optional number of times, but doesn't prompt for
% this number; enter it following the function.
% 
% \item[\key{R} [\var{function-name} [\var{n}]]]
% Search up the stack for function.  Prompts for the function name if not
% supplied.  Searches an optional number of times, but doesn't prompt for
% this number; enter it following the function.
\end{Lentry}


\subsection{How Arguments are Printed}

A frame is printed to look like a function call, but with the actual argument
values in the argument positions.  So the frame for this call in the source:

\begin{lisp}
(myfun (+ 3 4) 'a)
\end{lisp}

would look like this:

\begin{example}
(MYFUN 7 A)
\end{example}

All keyword and optional arguments are displayed with their actual
values; if the corresponding argument was not supplied, the value will
be the default.  So this call:

\begin{lisp}
(subseq "foo" 1)
\end{lisp}

would look like this:

\begin{example}
(SUBSEQ "foo" 1 3)
\end{example}

And this call:

\begin{lisp}
(string-upcase "test case")
\end{lisp}

would look like this:

\begin{example}
(STRING-UPCASE "test case" :START 0 :END NIL)
\end{example}

The arguments to a function call are displayed by accessing the argument
variables.  Although those variables are initialized to the actual argument
values, they can be set inside the function; in this case the new value will be
displayed.

\code{\amprest} arguments are handled somewhat differently.  The value of
the rest argument variable is displayed as the spread-out arguments to
the call, so:

\begin{lisp}
(format t "~A is a ~A." "This" 'test)
\end{lisp}

would look like this:

\begin{example}
(FORMAT T "~A is a ~A." "This" 'TEST)
\end{example}

Rest arguments cause an exception to the normal display of keyword
arguments in functions that have both \code{\amprest} and \code{\&key}
arguments.  In this case, the keyword argument variables are not
displayed at all; the rest arg is displayed instead.  So for these
functions, only the keywords actually supplied will be shown, and the
values displayed will be the argument values, not values of the
(possibly modified) variables.

If the variable for an argument is never referenced by the function, it will be
deleted.  The variable value is then unavailable, so the debugger prints
\code{\#\textless unused-arg\textgreater} instead of the value.  Similarly, if for any of a number of
reasons (described in more detail in section \ref{debug-vars}) the value of the
variable is unavailable or not known to be available, then
\code{\#\textless unavailable-arg\textgreater} will be printed instead of the argument value.

Printing of argument values is controlled by \code{*debug-print-level*} and
\varref{debug-print-length}.

\subsection{Function Names}
\cpsubindex{function}{names}
\cpsubindex{names}{function}

If a function is defined by \code{defun}, \code{labels}, or \code{flet}, then the
debugger will print the actual function name after the open parenthesis, like:

\begin{example}
(STRING-UPCASE "test case" :START 0 :END NIL)
((SETF AREF) \#\back{a} "for" 1)
\end{example}

Otherwise, the function name is a string, and will be printed in quotes:

\begin{example}
("DEFUN MYFUN" BAR)
("DEFMACRO DO" (DO ((I 0 (1+ I))) ((= I 13))) NIL)
("SETQ *GC-NOTIFY-BEFORE*")
\end{example}

This string name is derived from the \w{\code{def}\var{mumble}} form
that encloses or expanded into the lambda, or the outermost enclosing
form if there is no \w{\code{def}\var{mumble}}.

\subsection{Funny Frames}
\cindex{external entry points}
\cpsubindex{entry points}{external}
\cpsubindex{block compilation}{debugger implications}
\cpsubindex{external}{stack frame kind}
\cpsubindex{optional}{stack frame kind}
\cpsubindex{cleanup}{stack frame kind}

Sometimes the evaluator introduces new functions that are used to implement a
user function, but are not directly specified in the source.  The main place
this is done is for checking argument type and syntax.  Usually these functions
do their thing and then go away, and thus are not seen on the stack in the
debugger.  But when you get some sort of error during lambda-list processing,
you end up in the debugger on one of these funny frames.

These funny frames are flagged by printing ``\code{[}\var{keyword}\code{]}'' after the
parentheses.  For example, this call:

\begin{lisp}
(car 'a 'b)
\end{lisp}

will look like this:

\begin{example}
(CAR 2 A) [:EXTERNAL]
\end{example}

And this call:

\begin{lisp}
(string-upcase "test case" :end)
\end{lisp}

would look like this:

\begin{example}
("DEFUN STRING-UPCASE" "test case" 335544424 1) [:OPTIONAL]
\end{example}

As you can see, these frames have only a vague resemblance to the original
call.  Fortunately, the error message displayed when you enter the debugger
will usually tell you what problem is (in these cases, too many arguments
and odd keyword arguments.)  Also, if you go down the stack to the frame for
the calling function, you can display the original source (\pxlref{source-locations}.)

With recursive or block compiled functions
(\pxlref{block-compilation}), an \kwd{EXTERNAL} frame may appear
before the frame representing the first call to the recursive function
or entry to the compiled block. This is a consequence of the way the
compiler does block compilation: there is nothing odd with your
program. You will also see \kwd{CLEANUP} frames during the execution
of \code{unwind-protect} cleanup code. Note that inline expansion and
open-coding affect what frames are present in the debugger, see
sections \ref{debugger-policy} and \ref{open-coding}.


\subsection{Debug Tail Recursion}
\label{debug-tail-recursion}
\cindex{tail recursion}
\cpsubindex{recursion}{tail}

Both the compiler and the interpreter are ``properly tail recursive.''  If a
function call is in a tail-recursive position, the stack frame will be
deallocated {\em at the time of the call}, rather than after the call returns.
Consider this backtrace:
\begin{example}
(BAR ...) 
(FOO ...)
\end{example}
Because of tail recursion, it is not necessarily the case that
\code{FOO} directly called \code{BAR}.  It may be that \code{FOO} called
some other function \code{FOO2} which then called \code{BAR}
tail-recursively, as in this example:
\begin{example}
(defun foo ()
  ...
  (foo2 ...)
  ...)

(defun foo2 (...)
  ...
  (bar ...))

(defun bar (...)
  ...)
\end{example}

Usually the elimination of tail-recursive frames makes debugging more
pleasant, since theses frames are mostly uninformative.  If there is any
doubt about how one function called another, it can usually be
eliminated by finding the source location in the calling frame (section
\ref{source-locations}.)

The elimination of tail-recursive frames can be prevented by disabling
tail-recursion optimization, which happens when the \code{debug}
optimization quality is greater than \code{2}
(\pxlref{debugger-policy}.)

For a more thorough discussion of tail recursion, \pxlref{tail-recursion}.


\subsection{Unknown Locations and Interrupts}
\label{unknown-locations}
\cindex{unknown code locations}
\cpsubindex{locations}{unknown}
\cindex{interrupts}
\cpsubindex{errors}{run-time}

The debugger operates using special debugging information attached to
the compiled code.  This debug information tells the debugger what it
needs to know about the locations in the code where the debugger can be
invoked.  If the debugger somehow encounters a location not described in
the debug information, then it is said to be \var{unknown}.  If the code
location for a frame is unknown, then some variables may be
inaccessible, and the source location cannot be precisely displayed.

There are three reasons why a code location could be unknown:
\begin{itemize}

\item
There is inadequate debug information due to the value of the \code{debug}
optimization quality.  \xlref{debugger-policy}.

\item
The debugger was entered because of an interrupt such as \code{$\hat{ }C$}.

\item
A hardware error such as ``\code{bus error}'' occurred in code that was
compiled unsafely due to the value of the \code{safety} optimization
quality.  \xlref{optimize-declaration}.
\end{itemize}

In the last two cases, the values of argument variables are accessible,
but may be incorrect.  \xlref{debug-var-validity} for more details on
when variable values are accessible.

It is possible for an interrupt to happen when a function call or return is in
progress.  The debugger may then flame out with some obscure error or insist
that the bottom of the stack has been reached, when the real problem is that
the current stack frame can't be located.  If this happens, return from the
interrupt and try again.

When running interpreted code, all locations should be known.  However,
an interrupt might catch some subfunction of the interpreter at an
unknown location.  In this case, you should be able to go up the stack a
frame or two and reach an interpreted frame which can be debugged.


\section{Variable Access}
\label{debug-vars}
\cpsubindex{variables}{debugger access}
\cindex{debug variables}

There are three ways to access the current frame's local variables in the
debugger.  The simplest is to type the variable's name into the debugger's
read-eval-print loop.  The debugger will evaluate the variable reference as
though it had appeared inside that frame.

The debugger doesn't really understand lexical scoping; it has just one
namespace for all the variables in a function.  If a symbol is the name of
multiple variables in the same function, then the reference appears ambiguous,
even though lexical scoping specifies which value is visible at any given
source location.  If the scopes of the two variables are not nested, then the
debugger can resolve the ambiguity by observing that only one variable is
accessible.

When there are ambiguous variables, the evaluator assigns each one a
small integer identifier.  The \code{debug:var} function and the
\code{list-locals} command use this identifier to distinguish between
ambiguous variables:
\begin{Lentry}

\item[\code{list-locals} \mopt{\var{prefix}}]%%\hfill\\
This command prints the name and value of all variables in the current
frame whose name has the specified \var{prefix}.  \var{prefix} may be a
string or a symbol.  If no \var{prefix} is given, then all available
variables are printed.  If a variable has a potentially ambiguous name,
then the name is printed with a ``\code{\#}\var{identifier}'' suffix, where
\var{identifier} is the small integer used to make the name unique.
\end{Lentry}

\begin{defun}{debug:}{var}{\args{\var{name} \ampoptional{} \var{identifier}}}
  
  This function returns the value of the variable in the current frame
  with the specified \var{name}.  If supplied, \var{identifier}
  determines which value to return when there are ambiguous variables.
  
  When \var{name} is a symbol, it is interpreted as the symbol name of
  the variable, i.e. the package is significant.  If \var{name} is an
  uninterned symbol (gensym), then return the value of the uninterned
  variable with the same name.  If \var{name} is a string,
  \code{debug:var} interprets it as the prefix of a variable name, and
  must unambiguously complete to the name of a valid variable.
  
  This function is useful mainly for accessing the value of uninterned
  or ambiguous variables, since most variables can be evaluated
  directly.
\end{defun}


\subsection{Variable Value Availability}
\label{debug-var-validity}
\cindex{availability of debug variables}
\cindex{validity of debug variables}
\cindex{debug optimization quality}

The value of a variable may be unavailable to the debugger in portions of the
program where \clisp{} says that the variable is defined.  If a variable value is
not available, the debugger will not let you read or write that variable.  With
one exception, the debugger will never display an incorrect value for a
variable.  Rather than displaying incorrect values, the debugger tells you the
value is unavailable.

The one exception is this: if you interrupt (e.g., with \code{$\hat{ }C$}) or if there is
an unexpected hardware error such as ``\code{bus error}'' (which should only happen
in unsafe code), then the values displayed for arguments to the interrupted
frame might be incorrect.\footnote{Since the location of an interrupt or hardware
error will always be an unknown location (\pxlref{unknown-locations}),
non-argument variable values will never be available in the interrupted frame.}
This exception applies only to the interrupted frame: any frame farther down
the stack will be fine.

The value of a variable may be unavailable for these reasons:
\begin{itemize}

\item
The value of the \code{debug} optimization quality may have omitted debug
information needed to determine whether the variable is available.
Unless a variable is an argument, its value will only be available when
\code{debug} is at least \code{2}.

\item
The compiler did lifetime analysis and determined that the value was no longer
needed, even though its scope had not been exited.  Lifetime analysis is
inhibited when the \code{debug} optimization quality is \code{3}.

\item
The variable's name is an uninterned symbol (gensym).  To save space, the
compiler only dumps debug information about uninterned variables when the
\code{debug} optimization quality is \code{3}.

\item
The frame's location is unknown (\pxlref{unknown-locations}) because
the debugger was entered due to an interrupt or unexpected hardware error.
Under these conditions the values of arguments will be available, but might be
incorrect.  This is the exception above.

\item
The variable was optimized out of existence.  Variables with no reads are
always optimized away, even in the interpreter.  The degree to which the
compiler deletes variables will depend on the value of the \code{compile-speed}
optimization quality, but most source-level optimizations are done under all
compilation policies.
\end{itemize}


Since it is especially useful to be able to get the arguments to a function,
argument variables are treated specially when the \code{speed} optimization
quality is less than \code{3} and the \code{debug} quality is at least \code{1}.
With this compilation policy, the values of argument variables are almost
always available everywhere in the function, even at unknown locations.  For
non-argument variables, \code{debug} must be at least \code{2} for values to be
available, and even then, values are only available at known locations.


\subsection{Note On Lexical Variable Access}
\cpsubindex{evaluation}{debugger}
 
When the debugger command loop establishes variable bindings for available
variables, these variable bindings have lexical scope and dynamic
extent.\footnote{The variable bindings are actually created using the \clisp{}
\code{symbol-macrolet} special form.}  You can close over them, but such closures
can't be used as upward funargs.

You can also set local variables using \code{setq}, but if the variable was closed
over in the original source and never set, then setting the variable in the
debugger may not change the value in all the functions the variable is defined
in.  Another risk of setting variables is that you may assign a value of a type
that the compiler proved the variable could never take on.  This may result in
bad things happening.


\section{Source Location Printing}
\label{source-locations}
\cpsubindex{source location printing}{debugger}

One of \cmucl{}'s unique capabilities is source level debugging of compiled
code.  These commands display the source location for the current frame:
\begin{Lentry}

\item[\code{source} \mopt{\var{context}}]%%\hfill\\
This command displays the file that the current frame's function was defined
from (if it was defined from a file), and then the source form responsible for
generating the code that the current frame was executing.  If \var{context} is
specified, then it is an integer specifying the number of enclosing levels of
list structure to print.

\item[\code{vsource} \mopt{\var{context}}]%%\hfill\\
This command is identical to \code{source}, except that it uses the
global values of \code{*print-level*} and \code{*print-length*} instead
of the debugger printing control variables \code{*debug-print-level*}
and \code{*debug-print-length*}.
\end{Lentry}

The source form for a location in the code is the innermost list present
in the original source that encloses the form responsible for generating
that code.  If the actual source form is not a list, then some enclosing
list will be printed.  For example, if the source form was a reference
to the variable \code{*some-random-special*}, then the innermost
enclosing evaluated form will be printed.  Here are some possible
enclosing forms:
\begin{example}
(let ((a *some-random-special*))
  ...)

(+ *some-random-special* ...)
\end{example}

If the code at a location was generated from the expansion of a macro or a
source-level compiler optimization, then the form in the original source that
expanded into that code will be printed.  Suppose the file
\file{/usr/me/mystuff.lisp} looked like this:
\begin{example}
(defmacro mymac ()
  '(myfun))

(defun foo ()
  (mymac)
  ...)
\end{example}
If \code{foo} has called \code{myfun}, and is waiting for it to return, then the
\code{source} command would print:
\begin{example}
; File: /usr/me/mystuff.lisp

(MYMAC)
\end{example}
Note that the macro use was printed, not the actual function call form,
\code{(myfun)}.

If enclosing source is printed by giving an argument to \code{source} or
\code{vsource}, then the actual source form is marked by wrapping it in a list
whose first element is \code{\#:***HERE***}.  In the previous example, 
\w{\code{source 1}} would print:
\begin{example}
; File: /usr/me/mystuff.lisp

(DEFUN FOO ()
  (#:***HERE***
   (MYMAC))
  ...)
\end{example}


\subsection{How the Source is Found}

If the code was defined from \llisp{} by \code{compile} or
\code{eval}, then the source can always be reliably located.  If the
code was defined from a \code{fasl} file created by
\findexed{compile-file}, then the debugger gets the source forms it
prints by reading them from the original source file.  This is a
potential problem, since the source file might have moved or changed
since the time it was compiled.

The source file is opened using the \code{truename} of the source file
pathname originally given to the compiler.  This is an absolute pathname
with all logical names and symbolic links expanded.  If the file can't
be located using this name, then the debugger gives up and signals an
error.

If the source file can be found, but has been modified since the time it was
compiled, the debugger prints this warning:
\begin{example}
; File has been modified since compilation:
;   \var{filename}
; Using form offset instead of character position.
\end{example}
where \var{filename} is the name of the source file.  It then proceeds using a
robust but not foolproof heuristic for locating the source.  This heuristic
works if:
\begin{itemize}

\item
No top-level forms before the top-level form containing the source have been
added or deleted, and

\item
The top-level form containing the source has not been modified much.  (More
precisely, none of the list forms beginning before the source form have been
added or deleted.)
\end{itemize}

If the heuristic doesn't work, the displayed source will be wrong, but will
probably be near the actual source.  If the ``shape'' of the top-level form in
the source file is too different from the original form, then an error will be
signaled.  When the heuristic is used, the the source location commands are
noticeably slowed.

Source location printing can also be confused if (after the source was
compiled) a read-macro you used in the code was redefined to expand into
something different, or if a read-macro ever returns the same \code{eq}
list twice.  If you don't define read macros and don't use \code{\#\#} in
perverted ways, you don't need to worry about this.


\subsection{Source Location Availability}

\cindex{debug optimization quality}
Source location information is only available when the \code{debug}
optimization quality is at least \code{2}.  If source location information is
unavailable, the source commands will give an error message.

If source location information is available, but the source location is
unknown because of an interrupt or unexpected hardware error
(\pxlref{unknown-locations}), then the command will print:

\begin{example}
Unknown location: using block start.
\end{example}

and then proceed to print the source location for the start of the
{\em basic block} enclosing the code location.
\cpsubindex{block}{basic} \cpsubindex{block}{start location} 
It's a bit complicated to explain exactly what a basic block is, but
here are some properties of the block start location:

\begin{itemize}
  
\item The block start location may be the same as the true location.
  
\item The block start location will never be later in the the
  program's flow of control than the true location.
  
\item No conditional control structures (such as \code{if},
  \code{cond}, \code{or}) will intervene between the block start and
  the true location (but note that some conditionals present in the
  original source could be optimized away.)  Function calls {\em do not}
  end basic blocks.
  
\item The head of a loop will be the start of a block.
  
\item The programming language concept of ``block structure'' and the
  \clisp{} \code{block} special form are totally unrelated to the
  compiler's basic block.
\end{itemize}

In other words, the true location lies between the printed location and the
next conditional (but watch out because the compiler may have changed the
program on you.)


\section{Compiler Policy Control}
\label{debugger-policy}
\cpsubindex{policy}{debugger}
\cindex{debug optimization quality}
\cindex{optimize declaration}

The compilation policy specified by \code{optimize} declarations affects the
behavior seen in the debugger.  The \code{debug} quality directly affects the
debugger by controlling the amount of debugger information dumped.  Other
optimization qualities have indirect but observable effects due to changes in
the way compilation is done.

Unlike the other optimization qualities (which are compared in relative value
to evaluate tradeoffs), the \code{debug} optimization quality is directly
translated to a level of debug information.  This absolute interpretation
allows the user to count on a particular amount of debug information being
available even when the values of the other qualities are changed during
compilation.  These are the levels of debug information that correspond to the
values of the \code{debug} quality:
\begin{Lentry}

\item[\code{0}]
Only the function name and enough information to allow the stack to
be parsed.

\item[\code{\w{$>$ 0}}]
Any level greater than \code{0} gives level \code{0} plus all
argument variables.  Values will only be accessible if the argument
variable is never set and
\code{speed} is not \code{3}.  \cmucl{} allows any real value for optimization
qualities.  It may be useful to specify \code{0.5} to get backtrace argument
display without argument documentation.

\item[\code{1}] Level \code{1} provides argument documentation
(printed arglists) and derived argument/result type information.
This makes \findexed{describe} more informative, and allows the
compiler to do compile-time argument count and type checking for any
calls compiled at run-time.

\item[\code{2}]
Level \code{1} plus all interned local variables, source location
information, and lifetime information that tells the debugger when arguments
are available (even when \code{speed} is \code{3} or the argument is set.)  This is
the default.

\item[\code{\w{$>$ 2}}]
Any level greater than \code{2} gives level \code{2} and in addition
disables tail-call optimization, so that the backtrace will contain
frames for all invoked functions, even those in tail positions.

\item[\code{3}]
Level \code{2} plus all uninterned variables.  In addition, lifetime
analysis is disabled (even when \code{speed} is \code{3}), ensuring
that all variable values are available at any known location within
the scope of the binding.  This has a speed penalty in addition to the
obvious space penalty. 
\end{Lentry}

As you can see, if the \code{speed} quality is \code{3}, debugger performance is
degraded.  This effect comes from the elimination of argument variable
special-casing (\pxlref{debug-var-validity}.)  Some degree of
speed/debuggability tradeoff is unavoidable, but the effect is not too drastic
when \code{debug} is at least \code{2}.

\cindex{inline expansion}
\cindex{semi-inline expansion}
In addition to \code{inline} and \code{notinline} declarations, the relative values
of the \code{speed} and \code{space} qualities also change whether functions are
inline expanded (\pxlref{inline-expansion}.)  If a function is inline
expanded, then there will be no frame to represent the call, and the arguments
will be treated like any other local variable.  Functions may also be
``semi-inline'', in which case there is a frame to represent the call, but the
call is to an optimized local version of the function, not to the original
function.


\section{Exiting Commands}

These commands get you out of the debugger.

\begin{Lentry}

\item[\code{quit}]
Throw to top level.

\item[\code{restart} \mopt{\var{n}}]%%\hfill\\
Invokes the \var{n}th restart case as displayed by the \code{error}
command.  If \var{n} is not specified, the available restart cases are
reported.

\item[\code{go}]
Calls \code{continue} on the condition given to \code{debug}.  If there is no
restart case named \var{continue}, then an error is signaled.

\item[\code{abort}]
Calls \code{abort} on the condition given to \code{debug}.  This is
useful for popping debug command loop levels or aborting to top level,
as the case may be.

% (\code{debug:debug-return} \var{expression} \mopt{\var{frame}})
% 
% \item
% From the current or specified frame, return the result of evaluating
% expression.  If multiple values are expected, then this function should be
% called for multiple values.
\end{Lentry}


\section{Information Commands}

Most of these commands print information about the current frame or
function, but a few show general information.

\begin{Lentry}

\item[\code{help}, \code{?}]
Displays a synopsis of debugger commands.

\item[\code{describe}]
Calls \code{describe} on the current function, displays number of local
variables, and indicates whether the function is compiled or interpreted.

\item[\code{print}]
Displays the current function call as it would be displayed by moving to
this frame.

\item[\code{vprint} (or \code{pp}) \mopt{\var{verbosity}}]%%\hfill\\
Displays the current function call using \code{*print-level*} and
\code{*print-length*} instead of \code{*debug-print-level*} and
\code{*debug-print-length*}.  \var{verbosity} is a small integer
(default 2) that controls other dimensions of verbosity.

\item[\code{error}]
Prints the condition given to \code{invoke-debugger} and the active
proceed cases.

\item[\code{backtrace} \mopt{\var{n}}]\hfill\\
Displays all the frames from the current to the bottom.  Only shows
\var{n} frames if specified.  The printing is controlled by
\code{*debug-print-level*} and \code{*debug-print-length*}.

% (\code{debug:debug-function} \mopt{\var{n}})
% 
% \item
% Returns the function from the current or specified frame.
% 
% \item[(\code{debug:function-name} \mopt{\var{n}])]
% Returns the function name from the current or specified frame.
% 
% \item[(\code{debug:pc} \mopt{\var{frame}})]
% Returns the index of the instruction for the function in the current or
% specified frame.  This is useful in conjunction with \code{disassemble}.
% The pc returned points to the instruction after the one that was fatal.
\end{Lentry}


\section{Breakpoint Commands}\cindex{breakpoints}

\cmucl{} supports setting of breakpoints inside compiled functions and
stepping of compiled code.  Breakpoints can only be set at at known
locations (\pxlref{unknown-locations}), so these commands are largely
useless unless the \code{debug} optimize quality is at least \code{2}
(\pxlref{debugger-policy}).  These commands manipulate breakpoints:
\begin{Lentry}
\item[\code{breakpoint} \var{location} \mstar{\var{option} \var{value}}]
%%\hfill\\
Set a breakpoint in some function.  \var{location} may be an integer
code location number (as displayed by \code{list-locations}) or a
keyword.  The keyword can be used to indicate setting a breakpoint at
the function start (\kwd{start}, \kwd{s}) or function end
(\kwd{end}, \kwd{e}).  The \code{breakpoint} command has
\kwd{condition}, \kwd{break}, \kwd{print} and \kwd{function}
options which work similarly to the \code{trace} options.

\item[\code{list-locations} (or \code{ll}) \mopt{\var{function}}]%%\hfill\\
List all the code locations in the current frame's function, or in
\var{function} if it is supplied.  The display format is the code
location number, a colon and then the source form for that location:
\begin{example}
3: (1- N)
\end{example}
If consecutive locations have the same source, then a numeric range like
\code{3-5:} will be printed.  For example, a default function call has a
known location both immediately before and after the call, which would
result in two code locations with the same source.  The listed function
becomes the new default function for breakpoint setting (via the
\code{breakpoint}) command.

\item[\code{list-breakpoints} (or \code{lb})]%%\hfill\\
List all currently active breakpoints with their breakpoint number.

\item[\code{delete-breakpoint} (or \code{db}) \mopt{\var{number}}]%%\hfill\\
Delete a breakpoint specified by its breakpoint number.  If no number is
specified, delete all breakpoints.

\item[\code{step}]%%\hfill\\
Step to the next possible breakpoint location in the current function.
This always steps over function calls, instead of stepping into them
\end{Lentry}


\subsection{Breakpoint Example}

Consider this definition of the factorial function:

\begin{lisp}
(defun ! (n)
  (if (zerop n)
      1
      (* n (! (1- n)))))
\end{lisp}

This debugger session demonstrates the use of breakpoints:

\begin{example}
common-lisp-user> (break) ; Invoke debugger

Break

Restarts:
  0: [CONTINUE] Return from BREAK.
  1: [ABORT   ] Return to Top-Level.

Debug  (type H for help)

(INTERACTIVE-EVAL (BREAK))
0] ll #'!
0: #'(LAMBDA (N) (BLOCK ! (IF # 1 #)))
1: (ZEROP N)
2: (* N (! (1- N)))
3: (1- N)
4: (! (1- N))
5: (* N (! (1- N)))
6: #'(LAMBDA (N) (BLOCK ! (IF # 1 #)))
0] br 2
(* N (! (1- N)))
1: 2 in !
Added.
0] q

common-lisp-user> (! 10) ; Call the function

*Breakpoint hit*

Restarts:
  0: [CONTINUE] Return from BREAK.
  1: [ABORT   ] Return to Top-Level.

Debug  (type H for help)

(! 10) ; We are now in first call (arg 10) before the multiply
Source: (* N (! (1- N)))
3] st

*Step*

(! 10) ; We have finished evaluation of (1- n)
Source: (1- N)
3] st

*Breakpoint hit*

Restarts:
  0: [CONTINUE] Return from BREAK.
  1: [ABORT   ] Return to Top-Level.

Debug  (type H for help)

(! 9) ; We hit the breakpoint in the recursive call
Source: (* N (! (1- N)))
3] 
\end{example}


\section{Function Tracing}
\cindex{tracing}
\cpsubindex{function}{tracing}

The tracer causes selected functions to print their arguments and
their results whenever they are called.  Options allow conditional
printing of the trace information and conditional breakpoints on
function entry or exit.

\begin{defmac}{}{trace}{%
    \args{\mstar{option global-value} \mstar{name \mstar{option value}}}}
  
  \code{trace} is a debugging tool that prints information when
  specified functions are called.  In its simplest form:
  \begin{example}
    (trace \var{name-1} \var{name-2} ...)
  \end{example}
  \code{trace} causes a printout on \vindexed{trace-output} each time
  that one of the named functions is entered or returns (the
  \var{names} are not evaluated.)  Trace output is indented according
  to the number of pending traced calls, and this trace depth is
  printed at the beginning of each line of output.  Printing verbosity
  of arguments and return values is controlled by
  \vindexed{debug-print-level} and \vindexed{debug-print-length}.

  Local functions defined by \code{flet} and \code{labels} can be
  traced using the syntax \code{(flet f f1 f2 ...)} or \code{(labels f
    f1 f2 ...)} where \code{f} is the \code{flet} or \code{labels}
  function we want to trace and \code{f1}, \code{f2}, are the
  functions containing the local function \code{f}.
  Invidiual methods can also be traced using the syntax \code{(method
    \var{name} \var{qualifiers} \var{specializers})}.
  See~\ref{sec:method-tracing} for more information.

  If no \var{names} or \var{options} are are given, \code{trace}
  returns the list of all currently traced functions,
  \code{*traced-function-list*}.
  
  Trace options can cause the normal printout to be suppressed, or
  cause extra information to be printed.  Each option is a pair of an
  option keyword and a value form.  Options may be interspersed with
  function names.  Options only affect tracing of the function whose
  name they appear immediately after.  Global options are specified
  before the first name, and affect all functions traced by a given
  use of \code{trace}.  If an already traced function is traced again,
  any new options replace the old options.  The following options are
  defined:
  \begin{Lentry}
  \item[\kwd{condition} \var{form}, \kwd{condition-after} \var{form},
    \kwd{condition-all} \var{form}] If \kwd{condition} is specified,
    then \code{trace} does nothing unless \var{form} evaluates to true
    at the time of the call.  \kwd{condition-after} is similar, but
    suppresses the initial printout, and is tested when the function
    returns.  \kwd{condition-all} tries both before and after.
    
  \item[\kwd{wherein} \var{names}] If specified, \var{names} is a
    function name or list of names.  \code{trace} does nothing unless
    a call to one of those functions encloses the call to this
    function (i.e. it would appear in a backtrace.)  Anonymous
    functions have string names like \code{"DEFUN FOO"}.  Individual
    methods can also be traced.  See section~\ref{sec:method-tracing}.

  \item[\kwd{wherein-only} \var{names}] If specified, this is just
    like \kwd{wherein}, but trace produces output only if the
    immediate caller of the traced function is one of the functions
    listed in \var{names}.
    
  \item[\kwd{break} \var{form}, \kwd{break-after} \var{form},
    \kwd{break-all} \var{form}] If specified, and \var{form} evaluates
    to true, then the debugger is invoked at the start of the
    function, at the end of the function, or both, according to the
    respective option.
    
  \item[\kwd{print} \var{form}, \kwd{print-after} \var{form},
    \kwd{print-all} \var{form}] In addition to the usual printout, the
    result of evaluating \var{form} is printed at the start of the
    function, at the end of the function, or both, according to the
    respective option.  Multiple print options cause multiple values
    to be printed.
    
  \item[\kwd{function} \var{function-form}] This is a not really an
    option, but rather another way of specifying what function to
    trace.  The \var{function-form} is evaluated immediately, and the
    resulting function is traced.
    
  \item[\kwd{encapsulate \mgroup{:default | t | nil}}] In \cmucl,
    tracing can be done either by temporarily redefining the function
    name (encapsulation), or using breakpoints.  When breakpoints are
    used, the function object itself is destructively modified to
    cause the tracing action.  The advantage of using breakpoints is
    that tracing works even when the function is anonymously called
    via \code{funcall}.
  
    When \kwd{encapsulate} is true, tracing is done via encapsulation.
    \kwd{default} is the default, and means to use encapsulation for
    interpreted functions and funcallable instances, breakpoints
    otherwise.  When encapsulation is used, forms are {\it not}
    evaluated in the function's lexical environment, but
    \code{debug:arg} can still be used.

    Note that if you trace using \kwd{encapsulate}, you will
    only get a trace or breakpoint at the outermost call to the traced
    function, not on recursive calls.

  \end{Lentry}

  In the case of functions where the known return convention is used
  to optimize, encapsulation may be necessary in order to make
  tracing work at all.  The symptom of this occurring is an error
  stating
  \begin{example}
    Error in function \var{foo}: :FUNCTION-END breakpoints are
    currently unsupported for the known return convention.
  \end{example}
  in such cases we recommend using \code{(trace \var{foo} :encapsulate
    t)}
  
  \cpsubindex{tracing}{errors}
  \cpsubindex{breakpoints}{errors}
  \cpsubindex{errors}{breakpoints}
  \cindex{function-end breakpoints}
  \cpsubindex{breakpoints}{function-end}
  
  \kwd{condition}, \kwd{break} and \kwd{print} forms are evaluated in
  the lexical environment of the called function; \code{debug:var} and
  \code{debug:arg} can be used.  The \code{-after} and \code{-all}
  forms are evaluated in the null environment.
\end{defmac}

\begin{defmac}{}{untrace}{ \args{\amprest{} \var{function-names}}}
  
  This macro turns off tracing for the specified functions, and
  removes their names from \code{*traced-function-list*}.  If no
  \var{function-names} are given, then all currently traced functions
  are untraced.
\end{defmac}

\begin{defvar}{extensions:}{traced-function-list}
  
  A list of function names maintained and used by \code{trace},
  \code{untrace}, and \code{untrace-all}.  This list should contain
  the names of all functions currently being traced.
\end{defvar}

\begin{defvar}{extensions:}{max-trace-indentation}
  
  The maximum number of spaces which should be used to indent trace
  printout.  This variable is initially set to 40.
\end{defvar}

\begin{defvar}{debug:}{trace-encapsulate-package-names}
  
  A list of package names.  Functions from these packages are traced
  using encapsulation instead of function-end breakpoints.  This list
  should at least include those packages containing functions used
  directly or indirectly in the implementation of \code{trace}.
\end{defvar}


\subsection{Encapsulation Functions}
\cindex{encapsulation}
\cindex{advising}

The encapsulation functions provide a mechanism for intercepting the
arguments and results of a function.  \code{encapsulate} changes the
function definition of a symbol, and saves it so that it can be
restored later.  The new definition normally calls the original
definition.  The \clisp{} \findexed{fdefinition} function always returns
the original definition, stripping off any encapsulation.

The original definition of the symbol can be restored at any time by
the \code{unencapsulate} function.  \code{encapsulate} and \code{unencapsulate}
allow a symbol to be multiply encapsulated in such a way that different
encapsulations can be completely transparent to each other.

Each encapsulation has a type which may be an arbitrary lisp object.
If a symbol has several encapsulations of different types, then any
one of them can be removed without affecting more recent ones.
A symbol may have more than one encapsulation of the same type, but
only the most recent one can be undone.

\begin{defun}{extensions:}{encapsulate}{%
    \args{\var{symbol} \var{type} \var{body}}}
  
  Saves the current definition of \var{symbol}, and replaces it with a
  function which returns the result of evaluating the form,
  \var{body}.  \var{Type} is an arbitrary lisp object which is the
  type of encapsulation.
  
  When the new function is called, the following variables are bound
  for the evaluation of \var{body}:
  \begin{Lentry}
    
  \item[\code{extensions:argument-list}] A list of the arguments to
    the function.
    
  \item[\code{extensions:basic-definition}] The unencapsulated
    definition of the function.
  \end{Lentry}
  The unencapsulated definition may be called with the original
  arguments by including the form
  \begin{lisp}
    (apply extensions:basic-definition extensions:argument-list)
  \end{lisp}

  \code{encapsulate} always returns \var{symbol}.
\end{defun}

\begin{defun}{extensions:}{unencapsulate}{\args{\var{symbol} \var{type}}}
  
  Undoes \var{symbol}'s most recent encapsulation of type \var{type}.
  \var{Type} is compared with \code{eq}.  Encapsulations of other
  types are left in place.
\end{defun}

\begin{defun}{extensions:}{encapsulated-p}{%
    \args{\var{symbol} \var{type}}}
  
  Returns \true{} if \var{symbol} has an encapsulation of type
  \var{type}.  Returns \nil{} otherwise.  \var{type} is compared with
  \code{eq}.
\end{defun}

\subsection{Tracing Examples}
  Here is an example of tracing with some of the possible options.
  For simplicity, this is the function:
  \begin{example}
    (defun fact (n)
      (declare (double-float n) (optimize speed))
      (if (zerop n)
          1d0
          (* n (fact (1- n)))))
    (compile 'fact)
  \end{example}

  This example shows how to use the :condition option:
  \begin{example}
    (trace fact :condition (= 4d0 (debug:arg 0)))
    (fact 10d0) ->
      0: (FACT 4.0d0)
      0: FACT returned 24.0d0
    3628800.0d0
  \end{example}
  As we can see, we produced output when the condition was satisfied.

  Here's another example:
  \begin{example}
    (untrace)
    (trace fact :break (= 4d0 (debug:arg 0)))
    (fact 10d0) ->
      0: (FACT 5.0d0)
        1: (FACT 4.0d0)


    Breaking before traced call to FACT:
       [Condition of type SIMPLE-CONDITION]

    Restarts:
      0: [CONTINUE] Return from BREAK.
      1: [ABORT   ] Return to Top-Level.

    Debug  (type H for help)
  \end{example}
  In this example, we see that normal tracing occurs until we the
  argument reaches 4d0, at which point, we break into the debugger.


% section{The Single Stepper}
% 
% \begin{defmac}{}{step}{ \args{\var{form}}}
%   
%   Evaluates form with single stepping enabled or if \var{form} is
%   \code{T}, enables stepping until explicitly disabled.  Stepping can
%   be disabled by quitting to the lisp top level, or by evaluating the
%   form \w{\code{(step ())}}.
%   
%   While stepping is enabled, every call to eval will prompt the user
%   for a single character command.  The prompt is the form which is
%   about to be \code{eval}ed.  It is printed with \code{*print-level*}
%   and \code{*print-length*} bound to \code{*step-print-level*} and
%   \code{*step-print-length*}.  All interaction is done through the
%   stream \code{*query-io*}.  Because of this, the stepper can not be
%   used in Hemlock eval mode.  When connected to a slave Lisp, the
%   stepper can be used from Hemlock.
%   
%   The commands are:
%   \begin{Lentry}
%   
%   \item[\key{n} (next)] Evaluate the expression with stepping still
%     enabled.
%   
%   \item[\key{s} (skip)] Evaluate the expression with stepping
%     disabled.
%   
%   \item[\key{q} (quit)] Evaluate the expression, but disable all
%     further stepping inside the current call to \code{step}.
%   
%   \item[\key{p} (print)] Print current form.  (does not use
%     \code{*step-print-level*} or \code{*step-print-length*}.)
%   
%   \item[\key{b} (break)] Enter break loop, and then prompt for the
%     command again when the break loop returns.
%   
%   \item[\key{e} (eval)] Prompt for and evaluate an arbitrary
%     expression.  The expression is evaluated with stepping disabled.
%   
%   \item[\key{?} (help)] Prints a brief list of the commands.
%   
%   \item[\key{r} (return)] Prompt for an arbitrary value to return as
%     result of the current call to eval.
%   
%   \item[\key{g}] Throw to top level.
%   \end{Lentry}
% \end{defmac}
% 
% \begin{defvar}{extensions:}{step-print-level}
%   \defvarx[extensions:]{step-print-length}
%   
%   \code{*print-level*} and \code{*print-length*} are bound to these
%   values while printing the current form.  \code{*step-print-level*}
%   and \code{*step-print-length*} are initially bound to 4 and 5,
%   respectively.
% \end{defvar}
% 
% \begin{defvar}{extensions:}{max-step-indentation}
%   
%   Step indents the prompts to highlight the nesting of the evaluation.
%   This variable contains the maximum number of spaces to use for
%   indenting.  Initially set to 40.
% \end{defvar}


\section{Specials}
These are the special variables that control the debugger action.

\begin{defvar}{debug:}{debug-print-level}
  \defvarx[debug:]{debug-print-length}
  
  \code{*print-level*} and \code{*print-length*} are bound to these
  values during the execution of some debug commands.  When evaluating
  arbitrary expressions in the debugger, the normal values of
  \code{*print-level*} and \code{*print-length*} are in effect.  These
  variables are initially set to 3 and 5, respectively.
\end{defvar}

\input{compiler}
\input{compiler-hint}
\chapter{UNIX Interface}
\label{unix-interface}

\credits{by Robert MacLachlan, Skef Wholey, Bill Chiles and William Lott}


\cmucl{} attempts to make the full power of the underlying
environment available to the Lisp programmer. This is done using
combination of hand-coded interfaces and foreign function calls to C
libraries. Although the techniques differ, the style of interface is
similar. This chapter provides an overview of the facilities available
and general rules for using them, as well as describing specific
features in detail. It is assumed that the reader has a working
familiarity with Unix and X11, as well as access to the standard
system documentation.


\section{Reading the Command Line}

The shell parses the command line with which Lisp is invoked, and
passes a data structure containing the parsed information to Lisp.
This information is then extracted from that data structure and put
into a set of Lisp data structures.

\begin{defvar}{extensions:}{command-line-strings}
  \defvarx[extensions:]{command-line-utility-name}
  \defvarx[extensions:]{command-line-words}
  \defvarx[extensions:]{command-line-switches}
  
  The value of \code{*command-line-words*} is a list of strings that
  make up the command line, one word per string.  The first word on
  the command line, i.e.  the name of the program invoked (usually
  \code{lisp}) is stored in \code{*command-line-utility-name*}.  The
  value of \code{*command-line-switches*} is a list of
  \code{command-line-switch} structures, with a structure for each
  word on the command line starting with a hyphen.  All the command
  line words between the program name and the first switch are stored
  in \code{*command-line-words*}.
\end{defvar}

The following functions may be used to examine \code{command-line-switch}
structures.
\begin{defun}{extensions:}{cmd-switch-name}{\args{\var{switch}}}
  
  Returns the name of the switch, less the preceding hyphen and
  trailing equal sign (if any).
\end{defun}
\begin{defun}{extensions:}{cmd-switch-value}{\args{\var{switch}}}
  
  Returns the value designated using an embedded equal sign, if any.
  If the switch has no equal sign, then this is null.
\end{defun}
\begin{defun}{extensions:}{cmd-switch-words}{\args{\var{switch}}}
  
  Returns a list of the words between this switch and the next switch
  or the end of the command line.
\end{defun}
\begin{defun}{extensions:}{cmd-switch-arg}{\args{\var{switch}}}
  
  Returns the first non-null value from \code{cmd-switch-value}, the
  first element in \code{cmd-switch-words}, or the first word in
  \var{command-line-words}.
\end{defun}

\begin{defun}{extensions:}{get-command-line-switch}{\args{\var{sname}}}
  
  This function takes the name of a switch as a string and returns the
  value of the switch given on the command line.  If no value was
  specified, then any following words are returned.  If there are no
  following words, then \true{} is returned.  If the switch was not
  specified, then \false{} is returned.
\end{defun}

\begin{defmac}{extensions:}{defswitch}{%
    \args{\var{name} \ampoptional{} \var{function}}}
  
  This macro causes \var{function} to be called when the switch
  \var{name} appears in the command line.  Name is a simple-string
  that does not begin with a hyphen (unless the switch name really
  does begin with one.)
  
  If \var{function} is not supplied, then the switch is parsed into
  \var{command-line-switches}, but otherwise ignored.  This suppresses
  the undefined switch warning which would otherwise take place.  The
  warning can also be globally suppressed by
  \var{complain-about-illegal-switches}.
\end{defmac}


\section{Useful Variables}

\begin{defvar}{system:}{stdin}
  \defvarx[system:]{stdout} \defvarx[system:]{stderr}
  
  Streams connected to the standard input, output and error file
  descriptors.
\end{defvar}

\begin{defvar}{system:}{tty}
  
  A stream connected to \file{/dev/tty}.
\end{defvar}

\begin{defvar}{extensions:}{environment-list}
  The environment variables inherited by the current process, as a
  keyword-indexed alist. For example, to access the DISPLAY
  environment variable, you could use

\begin{lisp}
   (cdr (assoc :display ext:*environment-list*))
\end{lisp}

  Note that the case of the variable name is preserved when converting
  to a keyword.  Therefore, you need to specify the keyword properly for
  variable names containing lower-case letters,
\end{defvar}


\section{Lisp Equivalents for C Routines}

The UNIX documentation describes the system interface in terms of C
procedure headers.  The corresponding Lisp function will have a somewhat
different interface, since Lisp argument passing conventions and
datatypes are different.

The main difference in the argument passing conventions is that Lisp does not
support passing values by reference.  In Lisp, all argument and results are
passed by value.  Interface functions take some fixed number of arguments and
return some fixed number of values.  A given ``parameter'' in the C
specification will appear as an argument, return value, or both, depending on
whether it is an In parameter, Out parameter, or In/Out parameter.  The basic
transformation one makes to come up with the Lisp equivalent of a C routine is
to remove the Out parameters from the call, and treat them as extra return
values.  In/Out parameters appear both as arguments and return values.  Since
Out and In/Out parameters are only conventions in C, you must determine the
usage from the documentation.

Thus, the C routine declared as

\begin{example}
kern_return_t lookup(servport, portsname, portsid)
        port        servport;
        char        *portsname;
        int        *portsid;        /* out */
 {
  ...
  *portsid = <expression to compute portsid field>
  return(KERN_SUCCESS);
 }
\end{example}

has as its Lisp equivalent something like

\begin{lisp}
(defun lookup (ServPort PortsName)
  ...
  (values
   success
   <expression to compute portsid field>))
\end{lisp}

If there are multiple out or in-out arguments, then there are multiple
additional returns values.

Fortunately, \cmucl{} programmers rarely have to worry about the
nuances of this translation process, since the names of the arguments and
return values are documented in a way so that the \code{describe} function
(and the \hemlock{} \code{Describe Function Call} command, invoked with
\b{C-M-Shift-A}) will list this information.  Since the names of arguments
and return values are usually descriptive, the information that
\code{describe} prints is usually all one needs to write a
call. Most programmers use this on-line documentation nearly
all of the time, and thereby avoid the need to handle bulky
manuals and perform the translation from barbarous tongues.


\section{Type Translations}
\cindex{aliens}
\cpsubindex{types}{alien}
\cpsubindex{types}{foreign language}

Lisp data types have very different representations from those used by
conventional languages such as C.  Since the system interfaces are
designed for conventional languages, Lisp must translate objects to and
from the Lisp representations.  Many simple objects have a direct
translation: integers, characters, strings and floating point numbers
are translated to the corresponding Lisp object.  A number of types,
however, are implemented differently in Lisp for reasons of clarity and
efficiency.

Instances of enumerated types are expressed as keywords in Lisp.
Records, arrays, and pointer types are implemented with the \alien{}
facility (\pxlref{aliens}).  Access functions are defined
for these types which convert fields of records, elements of arrays,
or data referenced by pointers into Lisp objects (possibly another
object to be referenced with another access function).

One should dispose of \alien{} objects created by constructor
functions or returned from remote procedure calls when they are no
longer of any use, freeing the virtual memory associated with that
object.  Since \alien{}s contain pointers to non-Lisp data, the
garbage collector cannot do this itself.  If the memory
was obtained from \funref{make-alien} or from a foreign function call
to a routine that used \code{malloc}, then \funref{free-alien} should
be used.


\section{System Area Pointers}
\label{system-area-pointers}

\cindex{pointers}\cpsubindex{malloc}{C function}\cpsubindex{free}{C function}
Note that in some cases an address is represented by a Lisp integer, and in
other cases it is represented by a real pointer.  Pointers are usually used
when an object in the current address space is being referred to.  The MACH
virtual memory manipulation calls must use integers, since in principle the
address could be in any process, and Lisp cannot abide random pointers.
Because these types are represented differently in Lisp, one must explicitly
coerce between these representations.

System Area Pointers (SAPs) provide a mechanism that bypasses the
\alien{} type system and accesses virtual memory directly.  A SAP is a
raw byte pointer into the \code{lisp} process address space.  SAPs are
represented with a pointer descriptor, so SAP creation can cause
consing.  However, the compiler uses a non-descriptor representation
for SAPs when possible, so the consing overhead is generally minimal.
\xlref{non-descriptor}.

\begin{defun}{system:}{sap-int}{\args{\var{sap}}}
  \defunx[system:]{int-sap}{\args{\var{int}}}
  
  The function \code{sap-int} is used to generate an integer
  corresponding to the system area pointer, suitable for passing to
  the kernel interfaces (which want all addresses specified as
  integers).  The function \code{int-sap} is used to do the opposite
  conversion.  The integer representation of a SAP is the byte offset
  of the SAP from the start of the address space.
\end{defun}

\begin{defun}{system:}{sap+}{\args{\var{sap} \var{offset}}}
  
  This function adds a byte \var{offset} to \var{sap}, returning a new
  SAP.
\end{defun}

\begin{defun}{system:}{sap-ref-8}{\args{\var{sap} \var{offset}}}
  \defunx[system:]{sap-ref-16}{\args{\var{sap} \var{offset}}}
  \defunx[system:]{sap-ref-32}{\args{\var{sap} \var{offset}}
  \defunx[system:]{sap-ref-64}{\args{\var{sap} \var{offset}}}

  These functions return the 8, 16, 32 or 64 bit unsigned integer at
  \var{offset} from \var{sap}.  The \var{offset} is always a byte
  offset, regardless of the number of bits accessed.  \code{setf} may
  be used with the these functions to deposit values into virtual
  memory.
\end{defun}

\begin{defun}{system:}{signed-sap-ref-8}{\args{\var{sap} \var{offset}}}
  \defunx[system:]{signed-sap-ref-16}{\args{\var{sap} \var{offset}}}
  \defunx[system:]{signed-sap-ref-32}{\args{\var{sap} \var{offset}}}
  \defunx[system:]{signed-sap-ref-64}{\args{\var{sap} \var{offset}}}

  These functions are the same as the above unsigned operations,
  except that they sign-extend, returning a negative number if the
  high bit is set.
\end{defun}


\section{Unix System Calls}

You probably won't have much cause to use them, but all the Unix system
calls are available.  The Unix system call functions are in the
\code{Unix} package.  The name of the interface for a particular system
call is the name of the system call prepended with \code{unix-}.  The
system usually defines the associated constants without any prefix name.
To find out how to use a particular system call, try using
\code{describe} on it.  If that is unhelpful, look at the source in
\file{unix.lisp} or consult your system maintainer.

The Unix system calls indicate an error by returning \false{} as the
first value and the Unix error number as the second value.  If the call
succeeds, then the first value will always be non-\nil, often \code{t}.

For example, to use the \code{chdir} syscall: 

\begin{lisp}
(multiple-value-bind (success errno)
    (unix:unix-chdir "/tmp")
  (unless success
     (error "Can't change working directory: ~a"
            (unix:get-unix-error-msg errno))))
\end{lisp}

\begin{defun}{Unix:}{get-unix-error-msg}{\args{\var{error}}}

  This function returns a string describing the Unix error number
  \var{error} (this is similar to the Unix function \code{perror}). 
\end{defun}


\section{File Descriptor Streams}
\label{sec:fds}

Many of the UNIX system calls return file descriptors.  Instead of using other
UNIX system calls to perform I/O on them, you can create a stream around them.
For this purpose, fd-streams exist.  See also \funref{read-n-bytes}.

\begin{defun}{system:}{make-fd-stream}{%
    \args{\var{descriptor}} \keys{\kwd{input} \kwd{output}
      \kwd{element-type}} \morekeys{\kwd{buffering} \kwd{name}
      \kwd{file} \kwd{original}} \yetmorekeys{\kwd{delete-original}
      \kwd{auto-close}} \yetmorekeys{\kwd{timeout} \kwd{pathname}}}
  
  This function creates a file descriptor stream using
  \var{descriptor}.  If \kwd{input} is non-\nil, input operations are
  allowed.  If \kwd{output} is non-\nil, output operations are
  allowed.  The default is input only.  These keywords are defined:
  \begin{Lentry}
  \item[\kwd{element-type}] is the type of the unit of transaction for
    the stream, which defaults to \code{string-char}.  See the \clisp{}
    description of \code{open} for valid values.
  
  \item[\kwd{buffering}] is the kind of output buffering desired for
    the stream.  Legal values are \kwd{none} for no buffering,
    \kwd{line} for buffering up to each newline, and \kwd{full} for
    full buffering.
  
  \item[\kwd{name}] is a simple-string name to use for descriptive
    purposes when the system prints an fd-stream.  When printing
    fd-streams, the system prepends the streams name with \code{Stream
      for }.  If \var{name} is unspecified, it defaults to a string
    containing \var{file} or \var{descriptor}, in order of preference.
  
  \item[\kwd{file}, \kwd{original}] \var{file} specifies the defaulted
    namestring of the associated file when creating a file stream
    (must be a \code{simple-string}). \var{original} is the
    \code{simple-string} name of a backup file containing the original
    contents of \var{file} while writing \var{file}.
  
    When you abort the stream by passing \true{} to \code{close} as
    the second argument, if you supplied both \var{file} and
    \var{original}, \code{close} will rename the \var{original} name
    to the \var{file} name.  When you \code{close} the stream
    normally, if you supplied \var{original}, and
    \var{delete-original} is non-\nil, \code{close} deletes
    \var{original}.  If \var{auto-close} is true (the default), then
    \var{descriptor} will be closed when the stream is garbage
    collected.
  
  \item[\kwd{pathname}]: The original pathname passed to open and
    returned by \code{pathname}; not defaulted or translated.
  
  \item[\kwd{timeout}] if non-null, then \var{timeout} is an integer
    number of seconds after which an input wait should time out.  If a
    read does time out, then the \code{system:io-timeout} condition is
    signalled.
  \end{Lentry}
\end{defun}

\begin{defun}{system:}{fd-stream-p}{\args{\var{object}}}
  
  This function returns \true{} if \var{object} is an fd-stream, and
  \nil{} if not.  Obsolete: use the portable \code{(typep x
    'file-stream)}.
\end{defun}

\begin{defun}{system:}{fd-stream-fd}{\args{\var{stream}}}
  
  This returns the file descriptor associated with \var{stream}.
\end{defun}


\section{Unix Signals}
\cindex{unix signals} \cindex{signals}

\cmucl{} allows access to all the Unix signals that can be generated
under Unix.  It should be noted that if this capability is abused, it is
possible to completely destroy the running Lisp.  The following macros and
functions allow access to the Unix interrupt system.  The signal names as
specified in section 2 of the {\em Unix Programmer's Manual} are exported
from the Unix package.

\subsection{Changing Signal Handlers}
\label{signal-handlers}

\begin{defmac}{system:}{with-enabled-interrupts}{
    \args{\var{specs} \amprest{} \var{body}}}
  
  This macro should be called with a list of signal specifications,
  \var{specs}.  Each element of \var{specs} should be a list of
  two\hide{ or three} elements: the first should be the Unix signal
  for which a handler should be established, the second should be a
  function to be called when the signal is received\hide{, and the
    third should be an optional character used to generate the signal
    from the keyboard.  This last item is only useful for the SIGINT,
    SIGQUIT, and SIGTSTP signals.}  One or more signal handlers can be
  established in this way.  \code{with-enabled-interrupts} establishes
  the correct signal handlers and then executes the forms in
  \var{body}.  The forms are executed in an unwind-protect so that the
  state of the signal handlers will be restored to what it was before
  the \code{with-enabled-interrupts} was entered.  A signal handler
  function specified as NIL will set the Unix signal handler to the
  default which is normally either to ignore the signal or to cause a
  core dump depending on the particular signal.
\end{defmac}

\begin{defmac}{system:}{without-interrupts}{\args{\amprest{} \var{body}}}
  
  It is sometimes necessary to execute a piece a code that can not be
  interrupted.  This macro the forms in \var{body} with interrupts
  disabled.  Note that the Unix interrupts are not actually disabled,
  rather they are queued until after \var{body} has finished
  executing.
\end{defmac}

\begin{defmac}{system:}{with-interrupts}{\args{\amprest{} \var{body}}}
  
  When executing an interrupt handler, the system disables interrupts,
  as if the handler was wrapped in in a \code{without-interrupts}.
  The macro \code{with-interrupts} can be used to enable interrupts
  while the forms in \var{body} are evaluated.  This is useful if
  \var{body} is going to enter a break loop or do some long
  computation that might need to be interrupted.
\end{defmac}

\begin{defmac}{system:}{without-hemlock}{\args{\amprest{} \var{body}}}
  
  For some interrupts, such as SIGTSTP (suspend the Lisp process and
  return to the Unix shell) it is necessary to leave Hemlock and then
  return to it.  This macro executes the forms in \var{body} after
  exiting Hemlock.  When \var{body} has been executed, control is
  returned to Hemlock.
\end{defmac}

\begin{defun}{system:}{enable-interrupt}{%
    \args{\var{signal} \var{function}\hide{ \ampoptional{}
        \var{character}}}}
  
  This function establishes \var{function} as the handler for
  \var{signal}.
  \hide{The optional \var{character} can be specified
    for the SIGINT, SIGQUIT, and SIGTSTP signals and causes that
    character to generate the appropriate signal from the keyboard.}
  Unless you want to establish a global signal handler, you should use
  the macro \code{with-enabled-interrupts} to temporarily establish a
  signal handler.  \hide{Without \var{character},}
  \code{enable-interrupt} returns the old function associated with the
  signal.  \hide{When \var{character} is specified for SIGINT,
    SIGQUIT, or SIGTSTP, it returns the old character code.}
\end{defun}

\begin{defun}{system:}{ignore-interrupt}{\args{\var{signal}}}
  
  Ignore-interrupt sets the Unix signal mechanism to ignore
  \var{signal} which means that the Lisp process will never see the
  signal.  Ignore-interrupt returns the old function associated with
  the signal or \false{} if none is currently defined.
\end{defun}

\begin{defun}{system:}{default-interrupt}{\args{\var{signal}}}
  
  Default-interrupt can be used to tell the Unix signal mechanism to
  perform the default action for \var{signal}.  For details on what
  the default action for a signal is, see section 2 of the {\em Unix
    Programmer's Manual}.  In general, it is likely to ignore the
  signal or to cause a core dump.
\end{defun}


\subsection{Examples of Signal Handlers}

The following code is the signal handler used by the Lisp system for the
SIGINT signal.

\begin{lisp}
(defun ih-sigint (signal code scp)
  (declare (ignore signal code scp))
  (without-hemlock
   (with-interrupts
    (break "Software Interrupt" t))))
\end{lisp}

The \code{without-hemlock} form is used to make sure that Hemlock is exited before
a break loop is entered.  The \code{with-interrupts} form is used to enable
interrupts because the user may want to generate an interrupt while in the
break loop.  Finally, break is called to enter a break loop, so the user
can look at the current state of the computation.  If the user proceeds
from the break loop, the computation will be restarted from where it was
interrupted.

The following function is the Lisp signal handler for the SIGTSTP signal
which suspends a process and returns to the Unix shell.

\begin{lisp}
(defun ih-sigtstp (signal code scp)
  (declare (ignore signal code scp))
  (without-hemlock
   (Unix:unix-kill (Unix:unix-getpid) Unix:sigstop)))
\end{lisp}

Lisp uses this interrupt handler to catch the SIGTSTP signal because it is
necessary to get out of Hemlock in a clean way before returning to the shell.

To set up these interrupt handlers, the following is recommended:

\begin{lisp}
(with-enabled-interrupts ((Unix:SIGINT #'ih-sigint)
                          (Unix:SIGTSTP #'ih-sigtstp))
  <user code to execute with the above signal handlers enabled.>
)
\end{lisp}

\input{serve-event}
\chapter{Alien Objects}
\label{aliens}

\credits{by Robert MacLachlan and William Lott}


\section{Introduction to Aliens}

Because of Lisp's emphasis on dynamic memory allocation and garbage
collection, Lisp implementations use unconventional memory representations
for objects.  This representation mismatch creates problems when a Lisp
program must share objects with programs written in another language.  There
are three different approaches to establishing communication:

\begin{itemize}
\item The burden can be placed on the foreign program (and programmer) by
requiring the use of Lisp object representations.  The main difficulty with
this approach is that either the foreign program must be written with Lisp
interaction in mind, or a substantial amount of foreign ``glue'' code must be
written to perform the translation.

\item The Lisp system can automatically convert objects back and forth
between the Lisp and foreign representations.  This is convenient, but
translation becomes prohibitively slow when large or complex data structures
must be shared.

\item The Lisp program can directly manipulate foreign objects through the
use of extensions to the Lisp language.  Most Lisp systems make use of
this approach, but the language for describing types and expressing
accesses is often not powerful enough for complex objects to be easily
manipulated.
\end{itemize}

\cmucl{} relies primarily on the automatic conversion and direct manipulation
approaches: Aliens of simple scalar types are automatically converted,
while complex types are directly manipulated in their foreign
representation.  Any foreign objects that can't automatically be
converted into Lisp values are represented by objects of type
\code{alien-value}.  Since Lisp is a dynamically typed language, even
foreign objects must have a run-time type; this type information is
provided by encapsulating the raw pointer to the foreign data within an
\code{alien-value} object.

The Alien type language and operations are most similar to those of the
C language, but Aliens can also be used when communicating with most
other languages that can be linked with C.


\section{Alien Types}

Alien types have a description language based on nested list structure.  For
example:

\begin{example}
struct foo \{
    int a;
    struct foo *b[100];
\};
\end{example}

has the corresponding Alien type:

\begin{lisp}
(struct foo
  (a int)
  (b (array (* (struct foo)) 100)))
\end{lisp}


\subsection{Defining Alien Types}

Types may be either named or anonymous.  With structure and union
types, the name is part of the type specifier, allowing recursively
defined types such as:

\begin{lisp}
(struct foo (a (* (struct foo))))
\end{lisp}

An anonymous structure or union type is specified by using the name
\nil{}.  The \funref{with-alien} macro defines a local scope which
``captures'' any named type definitions.  Other types are not
inherently named, but can be given named abbreviations using
\code{def-alien-type}.

\begin{defmac}{alien:}{def-alien-type}{name type}  
  This macro globally defines \var{name} as a shorthand for the Alien
  type \var{type}.  When introducing global structure and union type
  definitions, \var{name} may be \nil, in which case the name to
  define is taken from the type's name.
\end{defmac}


\subsection{Alien Types and Lisp Types}

The Alien types form a subsystem of the \cmucl{} type system.  An
\code{alien} type specifier provides a way to use any Alien type as a
Lisp type specifier.  For example

\begin{lisp}
(typep foo '(alien (* int)))
\end{lisp}

can be used to determine whether \code{foo} is a pointer to an
\code{int}.  \code{alien} type specifiers can be used in the same ways
as ordinary type specifiers (like \code{string}.)  Alien type
declarations are subject to the same precise type checking as any
other declaration (\pxlref{precise-type-checks}.)

Note that the Alien type system overlaps with normal Lisp type
specifiers in some cases.  For example, the type specifier
\code{(alien single-float)} is identical to \code{single-float}, since
Alien floats are automatically converted to Lisp floats.  When
\code{type-of} is called on an Alien value that is not automatically
converted to a Lisp value, then it will return an \code{alien} type
specifier.


\subsection{Alien Type Specifiers}

Some Alien type names are \clisp{} symbols, but the names are
still exported from the \code{alien} package, so it is legal to say
\code{alien:single-float}.  These are the basic Alien type specifiers: 

\begin{deftp}{Alien type}{*}{%
    \args{\var{type}}}
  
  A pointer to an object of the specified \var{type}.  If \var{type}
  is \true, then it means a pointer to anything, similar to
  ``\code{void *}'' in ANSI C.  Currently, the only way to detect a
  null pointer is:
\begin{lisp}
  (zerop (sap-int (alien-sap \var{ptr})))
\end{lisp}
\xlref{system-area-pointers}
\end{deftp}

\begin{deftp}{Alien type}{array}{\var{type} \mstar{\var{dimension}}} 

  An array of the specified \var{dimensions}, holding elements of type
  \var{type}.  Note that \code{(* int)} and \code{(array int)} are
  considered to be different types when type checking is done; pointer
  and array types must be explicitly coerced using \code{cast}.
  
  Arrays are accessed using \code{deref}, passing the indices as
  additional arguments.  Elements are stored in column-major order (as
  in C), so the first dimension determines only the size of the memory
  block, and not the layout of the higher dimensions.  An array whose
  first dimension is variable may be specified by using \nil{} as the
  first dimension.  Fixed-size arrays can be allocated as array
  elements, structure slots or \code{with-alien} variables.  Dynamic
  arrays can only be allocated using \funref{make-alien}.
\end{deftp}

\begin{deftp}{Alien type}{struct}{\var{name} 
    \mstar{(\var{field} \var{type} \mopt{\var{bits}})}}
  
  A structure type with the specified \var{name} and \var{fields}.
  Fields are allocated at the same positions used by the
  implementation's C compiler.  \var{bits} is intended for C-like bit
  field support, but is currently unused.  If \var{name} is \false,
  then the type is anonymous.
  
  If a named Alien \code{struct} specifier is passed to
  \funref{def-alien-type} or \funref{with-alien}, then this defines,
  respectively, a new global or local Alien structure type.  If no
  \var{fields} are specified, then the fields are taken from the
  current (local or global) Alien structure type definition of
  \var{name}.
\end{deftp}

\begin{deftp}{Alien type}{union}{\var{name} 
    \mstar{(\var{field} \var{type} \mopt{\var{bits}})}}
  
  Similar to \code{struct}, but defines a union type.  All fields are
  allocated at the same offset, and the size of the union is the size
  of the largest field.  The programmer must determine which field is
  active from context.
\end{deftp}

\begin{deftp}{Alien type}{enum}{\var{name} \mstar{\var{spec}}}
  
  An enumeration type that maps between integer values and keywords.
  If \var{name} is \false, then the type is anonymous.  Each
  \var{spec} is either a keyword, or a list \code{(\var{keyword}
    \var{value})}.  If \var{integer} is not supplied, then it defaults
  to one greater than the value for the preceding spec (or to zero if
  it is the first spec.)
\end{deftp}

\begin{deftp}{Alien type}{signed}{\mopt{\var{bits}}}  
  A signed integer with the specified number of bits precision.  The
  upper limit on integer precision is determined by the machine's word
  size.  If no size is specified, the maximum size will be used.
\end{deftp}

\begin{deftp}{Alien type}{integer}{\mopt{\var{bits}}}  
  Identical to \code{signed}---the distinction between \code{signed}
  and \code{integer} is purely stylistic.
\end{deftp}

\begin{deftp}{Alien type}{unsigned}{\mopt{\var{bits}}}
  Like \code{signed}, but specifies an unsigned integer.
\end{deftp}

\begin{deftp}{Alien type}{boolean}{\mopt{\var{bits}}}
  Similar to an enumeration type that maps \code{0} to \false{} and
  all other values to \true.  \var{bits} determines the amount of
  storage allocated to hold the truth value.
\end{deftp}

\begin{deftp}{Alien type}{single-float}{}
  A floating-point number in IEEE single format.
\end{deftp}

\begin{deftp}{Alien type}{double-float}{}
  A floating-point number in IEEE double format.
\end{deftp}

\begin{deftp}{Alien type}{function}{\var{result-type} \mstar{\var{arg-type}}}
  \label{alien-function-types}
  A Alien function that takes arguments of the specified
  \var{arg-types} and returns a result of type \var{result-type}.
  Note that the only context where a \code{function} type is directly
  specified is in the argument to \code{alien-funcall} (see section
  \funref{alien-funcall}.)  In all other contexts, functions are
  represented by function pointer types: \code{(* (function ...))}.
\end{deftp}

\begin{deftp}{Alien type}{system-area-pointer}{}
  A pointer which is represented in Lisp as a
  \code{system-area-pointer} object (\pxlref{system-area-pointers}.)
\end{deftp}


\subsection{The C-Call Package}

The \code{c-call} package exports these type-equivalents to the C type
of the same name: \code{char}, \code{short}, \code{int}, \code{long},
\code{unsigned-char}, \code{unsigned-short}, \code{unsigned-int},
\code{unsigned-long}, \code{float}, \code{double}.  \code{c-call} also
exports these types:

\begin{deftp}{Alien type}{void}{}
  This type is used in function types to declare that no useful value
  is returned.  Evaluation of an \code{alien-funcall} form will return
  zero values.
\end{deftp}

\begin{deftp}{Alien type}{c-string}{}
  This type is similar to \code{(* char)}, but is interpreted as a
  null-terminated string, and is automatically converted into a Lisp
  string when accessed.  If the pointer is C \code{NULL} (or 0), then
  accessing gives Lisp \false.
  
  With Unicode, a Lisp string is not the same as a C string since a
  Lisp string uses two bytes for each character.  In this case, a
  C string is converted to a Lisp string by taking each byte of the
  C-string and applying \code{code-char} to create each character of
  the Lisp string.

  Similarly, a Lisp string is converted to a C string by taking the
  low 8 bits of the \code{char-code} of each character and assigning
  that to each byte of the C string.

  In either case, \code{string-encode} and \code{string-decode} may be
  useful to convert Unicode Lisp strings to or from C strings.

  Assigning a Lisp string to a \code{c-string} structure field or
  variable stores the contents of the string to the memory already
  pointed to by that variable.  When an Alien of type \code{(* char)}
  is assigned to a \code{c-string}, then the \code{c-string} pointer
  is assigned to.  This allows \code{c-string} pointers to be
  initialized.  For example:

\begin{lisp}
  (def-alien-type nil (struct foo (str c-string)))
  
  (defun make-foo (str)
    (let ((my-foo (make-alien (struct foo))))
      (setf (slot my-foo 'str) (make-alien char (length str)))
      (setf (slot my-foo 'str) str)
      my-foo))
\end{lisp}

Storing Lisp \false{} writes C \code{NULL} to the \code{c-string}
pointer.
\end{deftp}



\section{Alien Operations}

This section describes the basic operations on Alien values.

\subsection{Alien Access Operations}

\begin{defun}{alien:}{deref}{\args{\var{pointer-or-array} \amprest{} \var{indices}}}
  
  This function returns the value pointed to by an Alien pointer or
  the value of an Alien array element.  If a pointer, an optional
  single index can be specified to give the equivalent of C pointer
  arithmetic; this index is scaled by the size of the type pointed to.
  If an array, the number of indices must be the same as the number of
  dimensions in the array type.  \code{deref} can be set with
  \code{setf} to assign a new value.
\end{defun}
 
\begin{defun}{alien:}{slot}{\args{\var{struct-or-union} \var{slot-name}}}
  
  This function extracts the value of slot \var{slot-name} from the an
  Alien \code{struct} or \code{union}.  If \var{struct-or-union} is a
  pointer to a structure or union, then it is automatically
  dereferenced.  This can be set with \code{setf} to assign a new
  value.  Note that \var{slot-name} is evaluated, and need not be a
  compile-time constant (but only constant slot accesses are
  efficiently compiled.)
\end{defun}


\subsection{Alien Coercion Operations}

\begin{defmac}{alien:}{addr}{\var{alien-expr}}
  
  This macro returns a pointer to the location specified by
  \var{alien-expr}, which must be either an Alien variable, a use of
  \code{deref}, a use of \code{slot}, or a use of
  \funref{extern-alien}.
\end{defmac}

\begin{defmac}{alien:}{cast}{\var{alien} \var{new-type}}
  
  This macro converts \var{alien} to a new Alien with the specified
  \var{new-type}.  Both types must be an Alien pointer, array or
  function type.  Note that the result is not \code{eq} to the
  argument, but does refer to the same data bits.
\end{defmac}

\begin{defmac}{alien:}{sap-alien}{\var{sap} \var{type}}
  \defunx[alien:]{alien-sap}{\var{alien-value}}
  
  \code{sap-alien} converts \var{sap} (a system area pointer
  \pxlref{system-area-pointers}) to an Alien value with the specified
  \var{type}.  \var{type} is not evaluated.

\code{alien-sap} returns the SAP which points to \var{alien-value}'s
data.

The \var{type} to \code{sap-alien} and the type of the \var{alien-value} to
\code{alien-sap} must some Alien pointer, array or record type.
\end{defmac}


\subsection{Alien Dynamic Allocation}

Dynamic Aliens are allocated using the \code{malloc} library, so foreign code
can call \code{free} on the result of \code{make-alien}, and Lisp code can
call \code{free-alien} on objects allocated by foreign code.

\begin{defmac}{alien:}{make-alien}{\var{type} \mopt{\var{size}}}
  
  This macro returns a dynamically allocated Alien of the specified
  \var{type} (which is not evaluated.)  The allocated memory is not
  initialized, and may contain arbitrary junk.  If supplied,
  \var{size} is an expression to evaluate to compute the size of the
  allocated object.  There are two major cases:
  \begin{itemize}
  \item When \var{type} is an array type, an array of that type is
    allocated and a \var{pointer} to it is returned.  Note that you
    must use \code{deref} to change the result to an array before you
    can use \code{deref} to read or write elements:

\begin{lisp}
(defvar *foo* (make-alien (array char 10)))

(type-of *foo*) \result{} (alien (* (array (signed 8) 10)))

(setf (deref (deref foo) 0) 10) \result{} 10
\end{lisp}

    If supplied, \var{size} is used as the first dimension for the
    array.
    
  \item When \var{type} is any other type, then then an object for
    that type is allocated, and a \var{pointer} to it is returned.  So
    \code{(make-alien int)} returns a \code{(* int)}.  If \var{size}
    is specified, then a block of that many objects is allocated, with
    the result pointing to the first one.
  \end{itemize}
\end{defmac}
 
\begin{defun}{alien:}{free-alien}{\var{alien}}

  This function frees the storage for \var{alien} (which must have
  been allocated with \code{make-alien} or \code{malloc}.)
\end{defun}

See also \funref{with-alien}, which stack-allocates Aliens.


\section{Alien Variables}

Both local (stack allocated) and external (C global) Alien variables are
supported.


\subsection{Local Alien Variables}

\begin{defmac}{alien:}{with-alien}{\mstar{(\var{name} \var{type} 
      \mopt{\var{initial-value}})} \mstar{form}}
  
  This macro establishes local alien variables with the specified
  Alien types and names for dynamic extent of the body.  The variable
  \var{names} are established as symbol-macros; the bindings have
  lexical scope, and may be assigned with \code{setq} or \code{setf}.
  This form is analogous to defining a local variable in C: additional
  storage is allocated, and the initial value is copied.
  
  \code{with-alien} also establishes a new scope for named structures
  and unions.  Any \var{type} specified for a variable may contain
  name structure or union types with the slots specified.  Within the
  lexical scope of the binding specifiers and body, a locally defined
  structure type \var{foo} can be referenced by its name using:
\begin{lisp}
  (struct foo)
\end{lisp}
\end{defmac}


\subsection{External Alien Variables} 
\label{external-aliens}

External Alien names are strings, and Lisp names are symbols.  When an
external Alien is represented using a Lisp variable, there must be a
way to convert from one name syntax into the other.  The macros
\code{extern-alien}, \code{def-alien-variable} and
\funref{def-alien-routine} use this conversion heuristic:
\begin{itemize}
\item Alien names are converted to Lisp names by uppercasing and
  replacing underscores with hyphens.
  
\item Conversely, Lisp names are converted to Alien names by
  lowercasing and replacing hyphens with underscores.
  
\item Both the Lisp symbol and Alien string names may be separately
  specified by using a list of the form:
\begin{lisp}
  (\var{alien-string} \var{lisp-symbol})
\end{lisp}
\end{itemize}

\begin{defmac}{alien:}{def-alien-variable}{\var{name} \var{type}}
  
  This macro defines \var{name} as an external Alien variable of the
  specified Alien \var{type}.  \var{name} and \var{type} are not
  evaluated.  The Lisp name of the variable (see above) becomes a
  global Alien variable in the Lisp namespace.  Global Alien variables
  are effectively ``global symbol macros''; a reference to the
  variable fetches the contents of the external variable.  Similarly,
  setting the variable stores new contents---the new contents must be
  of the declared \var{type}.
  
  For example, it is often necessary to read the global C variable
  \code{errno} to determine why a particular function call failed.  It
  is possible to define errno and make it accessible from Lisp by the
  following:
\begin{lisp}
(def-alien-variable "errno" int)

;; Now it is possible to get the value of the C variable errno simply by
;; referencing that Lisp variable:
;;
(print errno)
\end{lisp}
\end{defmac}

\begin{defmac}{alien:}{extern-alien}{\var{name} \var{type}}
  
  This macro returns an Alien with the specified \var{type} which
  points to an externally defined value.  \var{name} is not evaluated,
  and may be specified either as a string or a symbol.  \var{type} is
  an unevaluated Alien type specifier.
\end{defmac}


\section{Alien Data Structure Example}

Now that we have Alien types, operations and variables, we can manipulate
foreign data structures.  This C declaration can be translated into the
following Alien type:

\begin{lisp}
struct foo \{
    int a;
    struct foo *b[100];
\};

 \myequiv

(def-alien-type nil
  (struct foo
    (a int)
    (b (array (* (struct foo)) 100))))
\end{lisp}

With this definition, the following C expression can be translated in this way:

\begin{example}
struct foo f;
f.b[7].a

 \myequiv

(with-alien ((f (struct foo)))
  (slot (deref (slot f 'b) 7) 'a)
  ;;
  ;; Do something with f...
  )
\end{example}


Or consider this example of an external C variable and some accesses:

\begin{example}
struct c_struct \{
        short x, y;
        char a, b;
        int z;
        c_struct *n;
\};

extern struct c_struct *my_struct;

my_struct->x++;
my_struct->a = 5;
my_struct = my_struct->n;
\end{example}

which can be made be manipulated in Lisp like this:

\begin{lisp}
(def-alien-type nil
  (struct c-struct
          (x short)
          (y short)
          (a char)
          (b char)
          (z int)
          (n (* c-struct))))

(def-alien-variable "my_struct" (* c-struct))

(incf (slot my-struct 'x))
(setf (slot my-struct 'a) 5)
(setq my-struct (slot my-struct 'n))
\end{lisp}


\section{Loading Unix Object Files}

\cmucl{} is able to load foreign object files at runtime, using the
function \code{load-foreign}. This function is able to load shared
libraries (that are typically named \verb|.so|) via the dlopen
mechanism. It can also load \verb|.a| or \verb|.o| object files by
calling the linker on the files and libraries to create a loadable
object file. Once loaded, the external symbols that define routines
and variables are made available for future external references (e.g.
by \code{extern-alien}.) \code{load-foreign} must be run before any of
the defined symbols are referenced.

Note that if a Lisp core image is saved (using \funref{save-lisp}), all
loaded foreign code is lost when the image is restarted. 


\begin{defun}{ext:}{load-foreign}{%
    \args{\var{files} \keys{\kwd{libraries} \kwd{base-file} \kwd{env}}}}
  
  \var{files} is a \code{simple-string} or list of
  \code{simple-string}s specifying the names of the object files. If
  \var{files} is a simple-string, the file that it designates is
  loaded using the platform's dlopen mechanism. If it is a list of
  strings, the platform linker \code{ld} is invoked to transform the
  object files into a loadable object file. \var{libraries} is a list
  of \code{simple-string}s specifying libraries in a format that the
  platform linker expects. The default value for \var{libraries} is
  \code{("-lc")} (i.e., the standard C library). \var{base-file} is
  the file to use for the initial symbol table information. The
  default is the Lisp start up code: \file{path:lisp}. \var{env}
  should be a list of simple strings in the format of Unix environment
  variables (i.e., \code{\var{A}=\var{B}}, where \var{A} is an
  environment variable and \var{B} is its value). The default value
  for \var{env} is the environment information available at the time
  Lisp was invoked. Unless you are certain that you want to change
  this, you should just use the default.
\end{defun}


\section{Alien Function Calls}

The foreign function call interface allows a Lisp program to call functions
written in other languages.  The current implementation of the foreign
function call interface assumes a C calling convention and thus routines
written in any language that adheres to this convention may be called from
Lisp.

Lisp sets up various interrupt handling routines and other environment
information when it first starts up, and expects these to be in place at all
times.  The C functions called by Lisp should either not change the
environment, especially the interrupt entry points, or should make sure
that these entry points are restored when the C function returns to Lisp.
If a C function makes changes without restoring things to the way they were
when the C function was entered, there is no telling what will happen.


\subsection{The alien-funcall Primitive}

\begin{defun}{alien:}{alien-funcall}{%
    \args{\var{alien-function} \amprest{} \var{arguments}}}
  
  This function is the foreign function call primitive:
  \var{alien-function} is called with the supplied \var{arguments} and
  its value is returned.  The \var{alien-function} is an arbitrary
  run-time expression; to call a constant function, use
  \funref{extern-alien} or \code{def-alien-routine}.
  
  The type of \var{alien-function} must be \code{(alien (function
    ...))} or \code{(alien (* (function ...)))},
  \xlref{alien-function-types}.  The function type is used to
  determine how to call the function (as though it was declared with
  a prototype.)  The type need not be known at compile time, but only
  known-type calls are efficiently compiled.  Limitations:
  \begin{itemize}
  \item Structure type return values are not implemented.
  \item Passing of structures by value is not implemented.
  \end{itemize}
\end{defun}

Here is an example which allocates a \code{(struct foo)}, calls a foreign
function to initialize it, then returns a Lisp vector of all the
\code{(* (struct foo))} objects filled in by the foreign call:

\begin{lisp}
;; Allocate a foo on the stack.
(with-alien ((f (struct foo)))
  ;;
  ;; Call some C function to fill in foo fields.
  (alien-funcall (extern-alien "mangle_foo" (function void (* foo)))
                 (addr f))
  ;;
  ;; Find how many foos to use by getting the A field.
  (let* ((num (slot f 'a))
         (result (make-array num)))
    ;;
    ;; Get a pointer to the array so that we don't have to keep
    ;; extracting it:
    (with-alien ((a (* (array (* (struct foo)) 100)) (addr (slot f 'b))))
      ;;
      ;; Loop over the first N elements and stash them in the
      ;; result vector.
      (dotimes (i num)
        (setf (svref result i) (deref (deref a) i)))
      result)))
\end{lisp}


\subsection{The def-alien-routine Macro}

\begin{defmac}{alien:}{def-alien-routine}{\var{name} \var{result-type}
    \mstar{(\var{aname} \var{atype} \mopt{style})}}
  
  This macro is a convenience for automatically generating Lisp
  interfaces to simple foreign functions.  The primary feature is the
  parameter style specification, which translates the C
  pass-by-reference idiom into additional return values.
  
  \var{name} is usually a string external symbol, but may also be a
  symbol Lisp name or a list of the foreign name and the Lisp name.
  If only one name is specified, the other is automatically derived,
  (\pxlref{external-aliens}.)
  
  \var{result-type} is the Alien type of the return value.  Each
  remaining subform specifies an argument to the foreign function.
  \var{aname} is the symbol name of the argument to the constructed
  function (for documentation) and \var{atype} is the Alien type of
  corresponding foreign argument.  The semantics of the actual call
  are the same as for \funref{alien-funcall}.  \var{style} should be
  one of the following:
  \begin{Lentry}
  \item[\kwd{in}] specifies that the argument is passed by value.
    This is the default.  \kwd{in} arguments have no corresponding
    return value from the Lisp function.
  
  \item[\kwd{out}] specifies a pass-by-reference output value.  The
    type of the argument must be a pointer to a fixed sized object
    (such as an integer or pointer).  \kwd{out} and \kwd{in-out}
    cannot be used with pointers to arrays, records or functions.  An
    object of the correct size is allocated, and its address is passed
    to the foreign function.  When the function returns, the contents
    of this location are returned as one of the values of the Lisp
    function.
  
  \item[\kwd{copy}] is similar to \kwd{in}, but the argument is copied
    to a pre-allocated object and a pointer to this object is passed
    to the foreign routine.
  
  \item[\kwd{in-out}] is a combination of \kwd{copy} and \kwd{out}.
    The argument is copied to a pre-allocated object and a pointer to
    this object is passed to the foreign routine.  On return, the
    contents of this location is returned as an additional value.
  \end{Lentry}
  Any efficiency-critical foreign interface function should be inline
  expanded by preceding \code{def-alien-routine} with:

\begin{lisp}
(declaim (inline \var{lisp-name}))
\end{lisp}

  In addition to avoiding the Lisp call overhead, this allows
  pointers, word-integers and floats to be passed using non-descriptor
  representations, avoiding consing (\pxlref{non-descriptor}.)
\end{defmac}


\subsection{def-alien-routine Example}

Consider the C function \code{cfoo} with the following calling convention:

\begin{example}
/* a for update
 * i out
 */
void cfoo (char *str, char *a, int *i);
\end{example}

which can be described by the following call to \code{def-alien-routine}:

\begin{lisp}
(def-alien-routine "cfoo" void
  (str c-string)
  (a char :in-out)
  (i int :out))
\end{lisp}

The Lisp function \code{cfoo} will have two arguments (\var{str} and \var{a})
and two return values (\var{a} and \var{i}).


\subsection{Calling Lisp from C}

% Calling Lisp functions from C is sometimes possible, but is rather hackish.
% See \code{funcall0} ... \code{funcall3} in the \file{lisp/arch.h}.  The
% arguments must be valid \cmucl{} object descriptors (e.g.  fixnums must be
% left-shifted by 2.)  See \file{compiler/generic/objdef.lisp} or the derived
% file \file{lisp/internals.h} for details of the object representation.
% \file{lisp/internals.h} is mechanically generated, and is not part of the
% source distribution.  It is distributed in the \file{docs/} directory of the
% binary distribution.

% Note that the garbage collector moves objects, and won't be able to fix up any
% references in C variables, so either turn GC off or don't keep Lisp pointers
% in C data unless they are to statically allocated objects.  You can use
% \funref{purify} to place live data structures in static space so that they
% won't move during GC.

\cmucl{} supports calling Lisp from C via the \funref{def-callback}
macro:

\begin{defmac}{alien:}{def-callback}{\var{name} (\var{return-type}
    \mstar{(arg-name arg-type)}) \ampbody{} \var{body}}
  This macro defines a Lisp function that can be called from C and a
  Lisp variable.  The arguments to the function must be alien types,
  and the return type must also be an alien type.  This Lisp function
  can be accessed via the \funref{callback} macro.

  \var{name} is the name of the Lisp function.  It is also the name of
  a variable to be used by the \code{callback} macro.

  \var{return-type} is the return type of the function.  This must be
  a recognized alien type.

  \var{arg-name} specifies the name of the argument to the function,
  and the argument has type \var{arg-type}, which must be an alien type.
\end{defmac}

\begin{defmac}{alien:}{callback}{\var{callback-symbol}}
  This macro extracts the appropriate information for the function
  named \var{callback-symbol} so that it can be called by a C
  function.  \var{callback-symbol} must be a symbol created by the
  \code{def-callback} macro.
\end{defmac}

\begin{defmac}{alien:}{callback-funcall}{\var{callback-name} \amprest{}
    \var{args}}
  This macro does the necessary stuff to call the callback named
  \var{callback-name} with the given arguments.
\end{defmac}

\subsection{Callback Example}

Here is a simple example of using callbacks.
\begin{lisp}
(use-package :alien)
(use-package :c-call)

(def-callback foo (int (arg1 int) (arg2 int))
  (format t "~&foo: ~S, ~S~%" arg1 arg2)
  (+ arg1 arg2))

(defun test-foo ()
  (callback-funcall foo 555 444444))
\end{lisp}

In this example, the callback function \code{foo} is defined which
takes two C \code{int} parameters and returns a \code{int}.  As this
shows, we can use arbitrary Lisp inside the function.

The function \code{test-foo} shows how we can call this callback
function from Lisp.  The macro \code{callback} extracts the necessary
information for the callback function \code{foo} which can be
converted into a pointer which we can call via \code{alien-funcall}.

The following code is a more complete example where a foreign routine
calls our Lisp routine.
\begin{lisp}
(use-package :alien)
(use-package :c-call)

(def-alien-routine qsort void
  (base (* t))
  (nmemb int)
  (size int)
  (compar (* (function int (* t) (* t)))))

(def-callback my< (int (arg1 (* double))
                       (arg2 (* double)))
  (let ((a1 (deref arg1))
        (a2 (deref arg2)))
    (cond ((= a1 a2)  0)
          ((< a1 a2) -1)
          (t         +1))))

(defun test-qsort ()
  (let ((a (make-array 10 :element-type 'double-float
                       :initial-contents '(0.1d0 0.5d0 0.2d0 1.2d0 1.5d0
                                           2.5d0 0.0d0 0.1d0 0.2d0 0.3d0))))
    (print a)
    (qsort (sys:vector-sap a)
           (length a)
           (alien-size double :bytes)
           (alien:callback my<))
    (print a)))
\end{lisp}

We define the alien routine, \code{qsort}, and a callback, \code{my<},
to determine whether two \code{double}'s are less than, greater than
or equal to each other.

The test function \code{test-qsort} shows how we can call the alien
sort routine with our Lisp comparison routine to produce a sorted
array.

\subsection{Accessing Lisp Arrays}

Due to the way \cmucl{} manages memory, the amount of memory that can
be dynamically allocated by \code{malloc} or \funref{make-alien} is
limited\footnote{\cmucl{} mmaps a large piece of memory for its own
use and this memory is typically about 256~MB above the start of the C
heap. Thus, only about 256~MB of memory can be dynamically allocated.
In earlier versions, this limit was closer to 8~MB.}.

To overcome this limitation, it is possible to access the content of
Lisp arrays which are limited only by the amount of physical memory
and swap space available.  However, this technique is only useful if
the foreign function takes pointers to memory instead of allocating
memory for itself.  In latter case, you will have to modify the
foreign functions.

This technique takes advantage of the fact that \cmucl{} has
specialized array types (\pxlref{specialized-array-types}) that match
a typical C array.  For example, a \code{(simple-array double-float
  (100))} is stored in memory in essentially the same way as the C
array \code{double x[100]} would be.  The following function allows us
to get the physical address of such a Lisp array:

\begin{example}
(defun array-data-address (array)
  "Return the physical address of where the actual data of an array is
stored.

ARRAY must be a specialized array type in \cmucl{}.  This means ARRAY
must be an array of one of the following types:

                  double-float
                  single-float
                  (unsigned-byte 32)
                  (unsigned-byte 16)
                  (unsigned-byte  8)
                  (signed-byte 32)
                  (signed-byte 16)
                  (signed-byte  8)
"
  (declare (type (or (simple-array (signed-byte 8))
                     (simple-array (signed-byte 16))
                     (simple-array (signed-byte 32))
                     (simple-array (unsigned-byte 8))
                     (simple-array (unsigned-byte 16))
                     (simple-array (unsigned-byte 32))
                     (simple-array single-float)
                     (simple-array double-float)
                     (simple-array (complex single-float))
                     (simple-array (complex double-float)))
                 array)
           (optimize (speed 3) (safety 0))
           (ext:optimize-interface (safety 3)))
  ;; with-array-data will get us to the actual data.  However, because
  ;; the array could have been displaced, we need to know where the
  ;; data starts.
  (lisp::with-array-data ((data array)
                          (start)
                          (end))
    (declare (ignore end))
    ;; DATA is a specialized simple-array.  Memory is laid out like this:
    ;;
    ;;   byte offset    Value
    ;;        0         type code (should be 70 for double-float vector)
    ;;        4         4 * number of elements in vector
    ;;        8         1st element of vector
    ;;      ...         ...
    ;;
    (let ((addr (+ 8 (logandc1 7 (kernel:get-lisp-obj-address data))))
          (type-size
           (let ((type (array-element-type data)))
             (cond ((or (equal type '(signed-byte 8))
                        (equal type '(unsigned-byte 8)))
                    1)
                   ((or (equal type '(signed-byte 16))
                        (equal type '(unsigned-byte 16)))
                    2)
                   ((or (equal type '(signed-byte 32))
                        (equal type '(unsigned-byte 32)))
                    4)
                   ((equal type 'single-float)
                    4)
                   ((equal type 'double-float)
                    8)
                   (t
                    (error "Unknown specialized array element type"))))))
      (declare (type (unsigned-byte 32) addr)
               (optimize (speed 3) (safety 0) (ext:inhibit-warnings 3)))
      (system:int-sap (the (unsigned-byte 32)
                        (+ addr (* type-size start)))))))
\end{example}

We note, however, that the system function
\findexed{system:vector-sap} will do the same thing as above does.

Assume we have the C function below that we wish to use:

\begin{example}
  double dotprod(double* x, double* y, int n)
  \{
    int k;
    double sum = 0;

    for (k = 0; k < n; ++k) \{
      sum += x[k] * y[k];
    \}
    return sum;
  \}
\end{example}

The following example generates two large arrays in Lisp, and calls the C
function to do the desired computation.  This would not have been
possible using \code{malloc} or \code{make-alien} since we need about
16 MB of memory to hold the two arrays.

\begin{example}
  (alien:def-alien-routine "dotprod" c-call:double
    (x (* double-float) :in)
    (y (* double-float) :in)
    (n c-call:int :in))
    
  (defun test-dotprod ()
    (let ((x (make-array 10000 :element-type 'double-float 
                         :initial-element 2d0))
          (y (make-array 10000 :element-type 'double-float
                         :initial-element 10d0)))
        (sys:without-gcing
          (let ((x-addr (sys:vector-sap x))
                (y-addr (sys:vector-sap y)))
            (dotprod x-addr y-addr 10000)))))
\end{example}

In this example, we have used \code{sys:vector-sap} instead of
\code{array-data-address}, but we could have used \code{(sys:int-sap
  (array-data-address x))} as well.

Also, we have wrapped the inner \code{let} expression in a
\code{sys:without-gcing} that disables garbage collection for the
duration of the body.  This will prevent garbage collection from
moving \code{x} and \code{y} arrays after we have obtained the (now
erroneous) addresses but before the call to \code{dotprod} is made.


\section{Step-by-Step Alien Example}

This section presents a complete example of an interface to a somewhat
complicated C function.  This example should give a fairly good idea
of how to get the effect you want for almost any kind of C function.
Suppose you have the following C function which you want to be able to
call from Lisp in the file \file{test.c}:

\begin{verbatim}                
struct c_struct
{
  int x;
  char *s;
};
 
struct c_struct *c_function (i, s, r, a)
    int i;
    char *s;
    struct c_struct *r;
    int a[10];
{
  int j;
  struct c_struct *r2;
 
  printf("i = %d\n", i);
  printf("s = %s\n", s);
  printf("r->x = %d\n", r->x);
  printf("r->s = %s\n", r->s);
  for (j = 0; j < 10; j++) printf("a[%d] = %d.\n", j, a[j]);
  r2 = (struct c_struct *) malloc (sizeof(struct c_struct));
  r2->x = i + 5;
  r2->s = "A C string";
  return(r2);
};
\end{verbatim}

It is possible to call this function from Lisp using the file \file{test.lisp}
whose contents is:

\begin{lisp}
;;; -*- Package: test-c-call -*-
(in-package "TEST-C-CALL")
(use-package "ALIEN")
(use-package "C-CALL")

;;; Define the record c-struct in Lisp.
(def-alien-type nil
    (struct c-struct
            (x int)
            (s c-string)))

;;; Define the Lisp function interface to the C routine.  It returns a
;;; pointer to a record of type c-struct.  It accepts four parameters:
;;; i, an int; s, a pointer to a string; r, a pointer to a c-struct
;;; record; and a, a pointer to the array of 10 ints.
;;;
;;; The INLINE declaration eliminates some efficiency notes about heap
;;; allocation of Alien values.
(declaim (inline c-function))
(def-alien-routine c-function
    (* (struct c-struct))
  (i int)
  (s c-string)
  (r (* (struct c-struct)))
  (a (array int 10)))

;;; A function which sets up the parameters to the C function and
;;; actually calls it.
(defun call-cfun ()
  (with-alien ((ar (array int 10))
               (c-struct (struct c-struct)))
    (dotimes (i 10)                     ; Fill array.
      (setf (deref ar i) i))
    (setf (slot c-struct 'x) 20)
    (setf (slot c-struct 's) "A Lisp String")

    (with-alien ((res (* (struct c-struct))
                 (c-function 5 "Another Lisp String" (addr c-struct) ar)))
      (format t "Returned from C function.~%")
      (multiple-value-prog1
          (values (slot res 'x)
                  (slot res 's))
        ;;              
        ;; Deallocate result {\em after} we are done using it.
        (free-alien res)))))
\end{lisp}

To execute the above example, it is necessary to compile the C routine as
follows:

\begin{example}
cc -c test.c
\end{example}

In order to enable incremental loading with some linkers, you may need to say:

\begin{example}
cc -G 0 -c test.c
\end{example}

Once the C code has been compiled, you can start up Lisp and load it in:

\begin{example}
% lisp
;;; Lisp should start up with its normal prompt.

;;; Compile the Lisp file.  This step can be done separately.  You don't have
;;; to recompile every time.
* (compile-file "test.lisp")

;;; Load the foreign object file to define the necessary symbols.  This must
;;; be done before loading any code that refers to these symbols.  next block
;;; of comments are actually the output of LOAD-FOREIGN.  Different linkers
;;; will give different warnings, but some warning about redefining the code
;;; size is typical.
* (load-foreign "test.o")

;;; Running library:load-foreign.csh...
;;; Loading object file...
;;; Parsing symbol table...
Warning:  "_gp" moved from #x00C082C0 to #x00C08460.
Warning:  "end" moved from #x00C00340 to #x00C004E0.

;;; o.k. now load the compiled Lisp object file.
* (load "test")

;;; Now we can call the routine that sets up the parameters and calls the C
;;; function.
* (test-c-call::call-cfun)

;;; The C routine prints the following information to standard output.
i = 5
s = Another Lisp string
r->x = 20
r->s = A Lisp string
a[0] = 0.
a[1] = 1.
a[2] = 2.
a[3] = 3.
a[4] = 4.
a[5] = 5.
a[6] = 6.
a[7] = 7.
a[8] = 8.
a[9] = 9.
;;; Lisp prints out the following information.
Returned from C function.
;;; Return values from the call to test-c-call::call-cfun.
10
"A C string"
*
\end{example}

If any of the foreign functions do output, they should not be called
from within \hemlock{}. Depending on the situation, various strange
behavior occurs. Under X, the output goes to the window in which Lisp
was started; on a terminal, the output will overwrite the \hemlock{}
screen image; in a \hemlock{} slave, standard output is
\file{/dev/null} by default, so any output is discarded.

\chapter{Interprocess Communication under LISP}
\label{remote}

\credits{by William Lott and Bill Chiles}


\cmucl{} offers a facility for interprocess communication (IPC)
on top of using Unix system calls and the complications of that level
of IPC.  There is a simple remote-procedure-call (RPC) package build
on top of TCP/IP sockets.


\section{The REMOTE Package}

The \code{remote} package provides simple RPC facility including
interfaces for creating servers, connecting to already existing
servers, and calling functions in other Lisp processes.  The routines
for establishing a connection between two processes,
\code{create-request-server} and \code{connect-to-remote-server},
return \var{wire} structures.  A wire maintains the current state of
a connection, and all the RPC forms require a wire to indicate where
to send requests.


\subsection{Connecting Servers and Clients}

Before a client can connect to a server, it must know the network address on
which the server accepts connections.  Network addresses consist of a host
address or name, and a port number.  Host addresses are either a string of the
form \code{VANCOUVER.SLISP.CS.CMU.EDU} or a 32 bit unsigned integer.  Port
numbers are 16 bit unsigned integers.  Note: \var{port} in this context has
nothing to do with Mach ports and message passing.

When a process wants to receive connection requests (that is, become a
server), it first picks an integer to use as the port.  Only one server
(Lisp or otherwise) can use a given port number on a given machine at
any particular time.  This can be an iterative process to find a free
port: picking an integer and calling \code{create-request-server}.  This
function signals an error if the chosen port is unusable.  You will
probably want to write a loop using \code{handler-case}, catching
conditions of type error, since this function does not signal more
specific conditions.

\begin{defun}{wire:}{create-request-server}{%
    \args{\var{port} \ampoptional{} \var{on-connect}}}

  \code{create-request-server} sets up the current Lisp to accept
  connections on the given port.  If port is unavailable for any
  reason, this signals an error.  When a client connects to this port,
  the acceptance mechanism makes a wire structure and invokes the
  \var{on-connect} function.  Invoking this function has a couple of
  purposes, and \var{on-connect} may be \nil{} in which case the
  system foregoes invoking any function at connect time.
  
  The \var{on-connect} function is both a hook that allows you access
  to the wire created by the acceptance mechanism, and it confirms the
  connection.  This function takes two arguments, the wire and the
  host address of the connecting process.  See the section on host
  addresses below.  When \var{on-connect} is \nil, the request server
  allows all connections.  When it is non-\nil, the function returns
  two values, whether to accept the connection and a function the
  system should call when the connection terminates.  Either value may
  be \nil, but when the first value is \nil, the acceptance mechanism
  destroys the wire.
  
  \code{create-request-server} returns an object that
  \code{destroy-request-server} uses to terminate a connection.
\end{defun}

\begin{defun}{wire:}{destroy-request-server}{\args{\var{server}}}
  
  \code{destroy-request-server} takes the result of
  \code{create-request-server} and terminates that server.  Any
  existing connections remain intact, but all additional connection
  attempts will fail.
\end{defun}

\begin{defun}{wire:}{connect-to-remote-server}{%
    \args{\var{host} \var{port} \ampoptional{} \var{on-death}}}
  
  \code{connect-to-remote-server} attempts to connect to a remote
  server at the given \var{port} on \var{host} and returns a wire
  structure if it is successful.  If \var{on-death} is non-\nil, it is
  a function the system invokes when this connection terminates.
\end{defun}


\subsection{Remote Evaluations}

After the server and client have connected, they each have a wire
allowing function evaluation in the other process.  This RPC mechanism
has three flavors: for side-effect only, for a single value, and for
multiple values.

Only a limited number of data types can be sent across wires as
arguments for remote function calls and as return values: integers
inclusively less than 32 bits in length, symbols, lists, and
\var{remote-objects} (\pxlref{remote-objs}).  The system sends symbols
as two strings, the package name and the symbol name, and if the
package doesn't exist remotely, the remote process signals an error.
The system ignores other slots of symbols.  Lists may be any tree of
the above valid data types.  To send other data types you must
represent them in terms of these supported types.  For example, you
could use \code{prin1-to-string} locally, send the string, and use
\code{read-from-string} remotely.

\begin{defmac}{wire:}{remote}{%
    \args{\var{wire} \mstar{call-specs}}}
  
  The \code{remote} macro arranges for the process at the other end of
  \var{wire} to invoke each of the functions in the \var{call-specs}.
  To make sure the system sends the remote evaluation requests over
  the wire, you must call \code{wire-force-output}.
  
  Each of \var{call-specs} looks like a function call textually, but
  it has some odd constraints and semantics.  The function position of
  the form must be the symbolic name of a function.  \code{remote}
  evaluates each of the argument subforms for each of the
  \var{call-specs} locally in the current context, sending these
  values as the arguments for the functions.
  
  Consider the following example:

\begin{verbatim}
(defun write-remote-string (str)
  (declare (simple-string str))
  (wire:remote wire
    (write-string str)))
\end{verbatim}

  The value of \code{str} in the local process is passed over the wire
  with a request to invoke \code{write-string} on the value.  The
  system does not expect to remotely evaluate \code{str} for a value
  in the remote process.
\end{defmac}

\begin{defun}{wire:}{wire-force-output}{\args{\var{wire}}}
  
  \code{wire-force-output} flushes all internal buffers associated
  with \var{wire}, sending the remote requests.  This is necessary
  after a call to \code{remote}.
\end{defun}

\begin{defmac}{wire:}{remote-value}{\args{\var{wire} \var{call-spec}}}
  
  The \code{remote-value} macro is similar to the \code{remote} macro.
  \code{remote-value} only takes one \var{call-spec}, and it returns
  the value returned by the function call in the remote process.  The
  value must be a valid type the system can send over a wire, and
  there is no need to call \code{wire-force-output} in conjunction
  with this interface.
  
  If client unwinds past the call to \code{remote-value}, the server
  continues running, but the system ignores the value the server sends
  back.
  
  If the server unwinds past the remotely requested call, instead of
  returning normally, \code{remote-value} returns two values, \nil{}
  and \true.  Otherwise this returns the result of the remote
  evaluation and \nil.
\end{defmac}

\begin{defmac}{wire:}{remote-value-bind}{%
    \args{\var{wire} (\mstar{variable}) \var{remote-form}
      \mstar{local-forms}}}
  
  \code{remote-value-bind} is similar to \code{multiple-value-bind}
  except the values bound come from \var{remote-form}'s evaluation in
  the remote process.  The \var{local-forms} execute in an implicit
  \code{progn}.
  
  If the client unwinds past the call to \code{remote-value-bind}, the
  server continues running, but the system ignores the values the
  server sends back.
  
  If the server unwinds past the remotely requested call, instead of
  returning normally, the \var{local-forms} never execute, and
  \code{remote-value-bind} returns \nil.
\end{defmac}


\subsection{Remote Objects}
\label{remote-objs}

The wire mechanism only directly supports a limited number of data
types for transmission as arguments for remote function calls and as
return values: integers inclusively less than 32 bits in length,
symbols, lists.  Sometimes it is useful to allow remote processes to
refer to local data structures without allowing the remote process
to operate on the data.  We have \var{remote-objects} to support
this without the need to represent the data structure in terms of
the above data types, to send the representation to the remote
process, to decode the representation, to later encode it again, and
to send it back along the wire.

You can convert any Lisp object into a remote-object.  When you send
a remote-object along a wire, the system simply sends a unique token
for it.  In the remote process, the system looks up the token and
returns a remote-object for the token.  When the remote process
needs to refer to the original Lisp object as an argument to a
remote call back or as a return value, it uses the remote-object it
has which the system converts to the unique token, sending that
along the wire to the originating process.  Upon receipt in the
first process, the system converts the token back to the same
(\code{eq}) remote-object.

\begin{defun}{wire:}{make-remote-object}{\args{\var{object}}}
  
  \code{make-remote-object} returns a remote-object that has
  \var{object} as its value.  The remote-object can be passed across
  wires just like the directly supported wire data types.
\end{defun}

\begin{defun}{wire:}{remote-object-p}{\args{\var{object}}}
  
  The function \code{remote-object-p} returns \true{} if \var{object}
  is a remote object and \nil{} otherwise.
\end{defun}

\begin{defun}{wire:}{remote-object-local-p}{\args{\var{remote}}}
  
  The function \code{remote-object-local-p} returns \true{} if
  \var{remote} refers to an object in the local process.  This is can
  only occur if the local process created \var{remote} with
  \code{make-remote-object}.
\end{defun}

\begin{defun}{wire:}{remote-object-eq}{\args{\var{obj1} \var{obj2}}}
  
  The function \code{remote-object-eq} returns \true{} if \var{obj1} and
  \var{obj2} refer to the same (\code{eq}) lisp object, regardless of
  which process created the remote-objects.
\end{defun}

\begin{defun}{wire:}{remote-object-value}{\args{\var{remote}}}
  
  This function returns the original object used to create the given
  remote object.  It is an error if some other process originally
  created the remote-object.
\end{defun}

\begin{defun}{wire:}{forget-remote-translation}{\args{\var{object}}}
  
  This function removes the information and storage necessary to
  translate remote-objects back into \var{object}, so the next
  \code{gc} can reclaim the memory.  You should use this when you no
  longer expect to receive references to \var{object}.  If some remote
  process does send a reference to \var{object},
  \code{remote-object-value} signals an error.
\end{defun}


% This stuff has been moved to internet.tex.  *** Remove me someday ***
% \subsection{Host Addresses}

% The operating system maintains a database of all the valid host
% addresses.  You can use this database to convert between host names
% and addresses and vice-versa.

% \begin{defun}{ext:}{lookup-host-entry}{\args{\var{host}}}
  
%   \code{lookup-host-entry} searches the database for the given
%   \var{host} and returns a host-entry structure for it.  If it fails
%   to find \var{host} in the database, it returns \nil.  \var{Host} is
%   either the address (as an integer) or the name (as a string) of the
%   desired host.
% \end{defun}

% \begin{defun}{ext:}{host-entry-name}{\args{\var{host-entry}}}
%   \defunx[ext:]{host-entry-aliases}{\args{\var{host-entry}}}
%   \defunx[ext:]{host-entry-addr-list}{\args{\var{host-entry}}}
%   \defunx[ext:]{host-entry-addr}{\args{\var{host-entry}}}

%   \code{host-entry-name}, \code{host-entry-aliases}, and
%   \code{host-entry-addr-list} each return the indicated slot from the
%   host-entry structure.  \code{host-entry-addr} returns the primary
%   (first) address from the list returned by
%   \code{host-entry-addr-list}.
% \end{defun}


\section{The WIRE Package}

The \code{wire} package provides for sending data along wires.  The
\code{remote} package sits on top of this package.  All data sent
with a given output routine must be read in the remote process with
the complementary fetching routine.  For example, if you send so a
string with \code{wire-output-string}, the remote process must know
to use \code{wire-get-string}.  To avoid rigid data transfers and
complicated code, the interface supports sending
\var{tagged} data.  With tagged data, the system sends a tag
announcing the type of the next data, and the remote system takes
care of fetching the appropriate type.

When using interfaces at the wire level instead of the RPC level,
the remote process must read everything sent by these routines.  If
the remote process leaves any input on the wire, it will later
mistake the data for an RPC request causing unknown lossage.


\subsection{Untagged Data}

When using these routines both ends of the wire know exactly what types are
coming and going and in what order. This data is restricted to the following
types:

\begin{itemize}
\item
8 bit unsigned bytes.

\item
32 bit unsigned bytes.

\item
32 bit integers.

\item
simple-strings less than 65535 in length.
\end{itemize}

\begin{defun}{wire:}{wire-output-byte}{\args{\var{wire} \var{byte}}}
  \defunx[wire:]{wire-get-byte}{\args{\var{wire}}}
  \defunx[wire:]{wire-output-number}{\args{\var{wire} \var{number}}}
  \defunx[wire:]{wire-get-number}{\args{\var{wire} \ampoptional{}
      \var{signed}}}
  \defunx[wire:]{wire-output-string}{\args{\var{wire} \var{string}}}
  \defunx[wire:]{wire-get-string}{\args{\var{wire}}}
  
  These functions either output or input an object of the specified
  data type.  When you use any of these output routines to send data
  across the wire, you must use the corresponding input routine
  interpret the data.
\end{defun}


\subsection{Tagged Data}

When using these routines, the system automatically transmits and interprets
the tags for you, so both ends can figure out what kind of data transfers
occur.  Sending tagged data allows a greater variety of data types: integers
inclusively less than 32 bits in length, symbols, lists, and \var{remote-objects}
(\pxlref{remote-objs}).  The system sends symbols as two strings, the
package name and the symbol name, and if the package doesn't exist remotely,
the remote process signals an error.  The system ignores other slots of
symbols.  Lists may be any tree of the above valid data types.  To send other
data types you must represent them in terms of these supported types.  For
example, you could use \code{prin1-to-string} locally, send the string, and use
\code{read-from-string} remotely.

\begin{defun}{wire:}{wire-output-object}{%
    \args{\var{wire} \var{object} \ampoptional{} \var{cache-it}}}
  \defunx[wire:]{wire-get-object}{\args{\var{wire}}}
  
  The function \code{wire-output-object} sends \var{object} over
  \var{wire} preceded by a tag indicating its type.
  
  If \var{cache-it} is non-\nil, this function only sends \var{object}
  the first time it gets \var{object}.  Each end of the wire
  associates a token with \var{object}, similar to remote-objects,
  allowing you to send the object more efficiently on successive
  transmissions.  \var{cache-it} defaults to \true{} for symbols and
  \nil{} for other types.  Since the RPC level requires function
  names, a high-level protocol based on a set of function calls saves
  time in sending the functions' names repeatedly.
  
  The function \code{wire-get-object} reads the results of
  \code{wire-output-object} and returns that object.
\end{defun}


\subsection{Making Your Own Wires}

You can create wires manually in addition to the \code{remote}
package's interface creating them for you. To create a wire, you need
a Unix {\em file descriptor}. If you are unfamiliar with Unix file
descriptors, see section 2 of the Unix manual pages.

\begin{defun}{wire:}{make-wire}{\args{\var{descriptor}}}

  The function \code{make-wire} creates a new wire when supplied with
  the file descriptor to use for the underlying I/O operations.
\end{defun}

\begin{defun}{wire:}{wire-p}{\args{\var{object}}}
  
  This function returns \true{} if \var{object} is indeed a wire,
  \nil{} otherwise.
\end{defun}

\begin{defun}{wire:}{wire-fd}{\args{\var{wire}}}
  
  This function returns the file descriptor used by the \var{wire}.
\end{defun}


\section{Out-Of-Band Data}

The TCP/IP protocol allows users to send data asynchronously, otherwise
known as \var{out-of-band} data.  When using this feature, the operating
system interrupts the receiving process if this process has chosen to be
notified about out-of-band data.  The receiver can grab this input
without affecting any information currently queued on the socket.
Therefore, you can use this without interfering with any current
activity due to other wire and remote interfaces.

Unfortunately, most implementations of TCP/IP are broken, so use of
out-of-band data is limited for safety reasons.  You can only reliably
send one character at a time.

The Wire package is built on top of \cmucl{}s networking support. In
view of this, it is possible to use the routines described in section
\ref{internet-oob} for handling and sending out-of-band data. These
all take a Unix file descriptor instead of a wire, but you can fetch a
wire's file descriptor with \code{wire-fd}.

\input{internet}
\input{debug-internals}
\chapter{Cross-Referencing Facility}
\label{xref}
\cindex{cross-referencing}
\credits{by Eric Marsden}

The \cmucl{} cross-referencing facility (abbreviated XREF) assists in
the analysis of static dependency relationships in a program. It
provides introspection capabilities such as the ability to know which
functions may call a given function, and the program contexts in which
a particular global variable is used. The compiler populates a
database of cross-reference information, which can be queried by the
user to know:

\begin{itemize}
\item
the list of program contexts (functions, macros, top-level forms)
where a given function may be called at runtime, either directly or
indirectly (via its function-object);

\item
the list of program contexts where a given global variable may be
read;

\item
the list of program contexts that bind a global variable;

\item
the list of program contexts where a given global variable may be
modified during the execution of the program.
\end{itemize}

A global variable is either a dynamic variable or a constant variable,
for instance declared using \code{defvar} or \code{defparameter} or
\code{defconstant}.


\section{Populating the cross-reference database}

\begin{defvar}{c:}{record-xref-info}
   When non-NIL, code that is compiled (either using
   \code{compile-file}, or by calling \code{compile} from the
   listener), will be analyzed for cross-references. Defaults to
   \nil{}.
\end{defvar}

Cross-referencing information is only generated by the compiler; the
interpreter does not populate the cross-reference database. XREF
analysis is independent of whether the compiler is generating native
code or byte code, and of whether it is compiling from a file, from a
stream, or is invoked interactively from the listener. 

Alternatively, the \kwd{:xref} option to \code{compile-file} may be
specified to populate the cross-reference database when compiling a
file.  In this case, loading the generated fasl file in a fresh lisp
will also populate the cross-reference database.

\begin{defun}{xref:}{init-xref-database}{}
  Reinitializes the database of cross-references. This can be used to
  reclaim the space occupied by the database contents, or to discard
  stale cross-reference information.
\end{defun}



\section{Querying the cross-reference database}

\cmucl{} provides a number of functions in the XREF package that may
be used to query the cross-reference database:

\begin{defun}{xref:}{who-calls}{\args \var{function}}
   Returns the list of xref-contexts where \var{function} (either a
   symbol that names a function, or a function object) may be called
   at runtime. XREF does not record calls to macro-functions (such as
   \code{defun}) or to special forms (such as \code{eval-when}).
\end{defun}

\begin{defun}{xref:}{who-references}{\args \var{global-variable}}
   Returns the list of program contexts that may reference
   \var{global-variable}. 
\end{defun}

\begin{defun}{xref:}{who-binds}{\args \var{global-variable}}
  Returns a list of program contexts where the specified global
  variable may be bound at runtime (for example using \code{LET}).
\end{defun}

\begin{defun}{xref:}{who-sets}{\args \var{global-variable}}
  Returns a list of program contexts where the given global variable
  may be modified at runtime (for example using \code{SETQ}). 
\end{defun}

An \textit{xref-context} is the originating site of a cross-reference.
It identifies a portion of a program, and is defined by an
\code{xref-context} structure, that comprises a name, a source file and a
source-path. 

\begin{defun}{xref:}{xref-context-name}{\args \var{context}}
  Returns the name slot of an xref-context, which is one of:
\begin{itemize}
\item
a global function, which is named by a symbol or by a list of the form
\code{(setf\ foo)}. 

\item
a macro, named by a list \verb|(:macro foo)|.

\item
an inner function (\code{flet}, \code{labels}, or anonymous lambdas) that
is named by a list of the form \code{(:internal outer inner)}.

\item
a method, named by a list of the form
\verb|(:method foo (specializer1 specializer2)|. 

\item
a string \verb|"Top-Level Form"| that identifies a reference from a
top-level form. Note that multiple references from top-level forms
will only be listed once. 

\item
a compiler-macro, named by a string of the form
\verb|(:compiler-macro foo)|. 

\item
a string such as \verb|"DEFSTRUCT FOO"|, identifying a reference from
within a structure accessor or constructor or copier.

\item
a string such as 
\begin{verbatim}
  "Creation Form for #<KERNEL::CLASS-CELL STRUCT-FOO>"
\end{verbatim}

\item
a string such as \verb|"defun foo"|, or \verb|"defmethod bar (t)"|,
that identifies a reference from within code that has been generated
by the compiler for that form. For example, the compilation of a
\code{defclass} form causes accessor functions to be generated by the
compiler; this code is compiler-generated (it does not appear in the
source file), and so is identified by the XREF facility by a string. 
\end{itemize}
\end{defun}


\begin{defun}{xref:}{xref-context-file}{\var{context}}
  Return the truename (in the sense of the variable
   \vindexed{compile-file-truename}) of the source file from which the
   referencing forms were compiled. This slot will be \nil{} if the
   code was compiled from a stream, or interactively from the
   listener.
\end{defun}

\begin{defun}{xref:}{xref-context-source-path}{\var{context}}
  Return a list of positive integers identifying the form that
  contains the cross-reference. The first integer in the source-path
  is the number of the top-level form containing the cross-reference
  (for example, 2 identifies the second top-level form in the source
  file). The second integer in the source-path identifies the form
  within this top-level form that contains the cross-reference, and so
  on. This function will always return \nil{} if the file slot of an
  xref-context is \nil{}.

% While walking the top-level form, count one in depth-first order for
% each subform that is a cons.
\end{defun}




\section{Example usage}

In this section, we will illustrate use of the XREF facility on a
number of simple examples.

Consider the following program fragment, that defines a global
variable and a function.

\begin{verbatim}
  (defvar *variable-one* 42)
  
  (defun function-one (x)
     (princ (* x *variable-one*)))
\end{verbatim}

We save this code in a file named \code{example.lisp}, enable
cross-referencing, clear any previous cross-reference information,
compile the file, and can then query the cross-reference database
(output has been modified for readability).

\begin{verbatim}
  USER> (setf c:*record-xref-info* t)
  USER> (xref:init-xref-database)
  USER> (compile-file "example")
  USER> (xref:who-calls 'princ)
  (#<xref-context function-one in #p"example.lisp">)
  USER> (xref:who-references '*variable-one*)
  (#<xref-context function-one in #p"example.lisp">)
\end{verbatim}

From this example, we see that the compiler has noted the call to the
global function \code{princ} in \code{function-one}, and the reference
to the global variable \code{*variable-one*}. 

Suppose that we add the following code to the previous file. 

\begin{verbatim}
(defconstant +constant-one+ 1)
  
(defstruct struct-one
  slot-one
  (slot-two +constant-one+ :type integer)
  (slot-three 42 :read-only t))

(defmacro with-different-one (&body body)
  `(let ((*variable-one* 666))
      ,@body))

(defun get-variable-one () *variable-one*)

(defun (setf get-variable-one) (new-value)
  (setq *variable-one* new-value))
\end{verbatim}

In the following example, we detect references x and y.


% FIXME add function with LABELS, a binding, a set



The following function illustrates the effect that various forms of
optimization carried out by the \cmucl{} compiler can have on the
cross-references that are reported for a particular program. The
compiler is able to detect that the evaluated condition is always
false, and that the first clause of the \code{if} will never be taken
(this optimization is called dead-code elimination). XREF will
therefore not register a call to the function \code{sin} from the
function \code{foo}. Likewise, no calls to the functions \code{sqrt}
and \code{\textless} are registered, because the compiler has eliminated the
code that evaluates the condition. Finally, no call to the function
\code{expt} is generated, because the compiler was able to evaluate
the result of the expression \code{(expt 3 2)} at compile-time (though
a process called constant-folding).

\begin{verbatim}
;; zero call references are registered for this function!
(defun constantly-nine (x)
  (if (< (sqrt x) 0)
      (sin x)
      (expt 3 2)))
\end{verbatim}


\section{Limitations of the cross-referencing facility}

No cross-reference information is available for interpreted functions.
The cross-referencing database is not persistent: unless you save an
image using \code{save-lisp}, the database will be empty each time
\cmucl{} is restarted. There is no mechanism that saves
cross-reference information in FASL files, so loading a system from
compiled code will not populate the cross-reference database. The XREF
database currently accumulates ``stale'' information: when compiling a
file, it does not delete any cross-references that may have previously
been generated for that file. This latter limitation will be removed
in a future release. 

The cross-referencing facility is only able to analyze the static
dependencies in a program; it does not provide any information about
runtime (dynamic) dependencies. For instance, XREF is able to identify
the list of program contexts where a given function may be called, but
is not able to determine which contexts will be activated when the
program is executed with a specific set of input parameters. However,
the static analysis that is performed by the \cmucl{} compiler does
allow XREF to provide more information than would be available from a
mere syntactic analysis of a program. References that occur from
within unreachable code will not be displayed by XREF, because the
\cmucl{} compiler deletes dead code before cross-references are
analyzed. Certain ``trivial'' function calls (where the result of the
function call can be evaluated at compile-time) may be eliminated by
optimizations carried out by the compiler; see the example below.

If you examine the entire database of cross-reference information (by
accessing undocumented internals of the XREF package), you will note
that XREF notes ``bogus'' cross-references to function calls that are
inserted by the compiler. For example, in safe code, the \cmucl{}
compiler inserts a call to an internal function called
\code{c::\%verify-argument-count}, so that the number of arguments
passed to the function is checked each time it is called. The XREF
facility does not distinguish between user code and these forms that
are introduced during compilation. This limitation should not be
visible if you use the documented functions in the XREF package. 

As of the 18e release of \cmucl{}, the cross-referencing facility is
experimental; expect details of its implementation to change in future
releases. In particular, the names given to CLOS methods and to inner
functions will change in future releases. 


\chapter{Internationalization}
\label{i18n}
\cindex{Internationalization}

\cmucl{} supports internationalization by supporting Unicode
characters internally and by adding support for external formats to
convert from the internal format to an appropriate external character
coding format.

To understand the support for Unicode, we refer the reader to the
\ifpdf
\href{http://www.unicode.org/}{Unicode standard}.
\else
\emph{Unicode standard} at \url{http://www.unicode.org}
\fi
\section{Changes}

To support internationalization, the following changes to Common Lisp
functions have been done.


\subsection{Design Choices}

To support Unicode, there are many approaches.  One choice is to
support both 8-bit \code{base-char} and a 21-bit (or larger)
\code{character} since Unicode codepoints use 21 bits.  This generally
means strings are much larger, and complicates the compiler by having
to support both \code{base-char} and \code{character} types and the
corresponding string types.  This also adds complexity for the user to
understand the difference between the different string and character
types.

Another choice is to have just one character and string type that can
hold the entire Unicode codepoint.  While simplifying the compiler and
reducing the burden on the user, this significantly increases memory
usage for strings.

The solution chosen by \cmucl{} is to tradeoff the size and complexity
by having only 16-bit characters.  Most of the important languages can
be encoded using only 16-bits.  The rest of the codepoints are for
rare languages or ancient scripts.  Thus, the memory usage is
significantly reduced while still supporting the the most important
languages.  Compiler complexity is also reduced since \code{base-char}
and \code{character} are the same as are the string types..  But we
still want to support the full Unicode character set.  This is
achieved by making strings be UTF-16 strings internally.  Hence, Lisp
strings are UTF-16 strings, and Lisp characters are UTF-16 code-units.


\subsection{Characters}
\label{sec:i18n:characters}

Characters are now 16 bits long instead of 8 bits, and \code{base-char}
and \code{character} types are the same.  This difference is
naturally indicated by changing \code{char-code-limit} from 256 to
65536.

\subsection{Strings}
\label{sec:i18n:strings}

In \cmucl{} there is only one type of string---\code{base-string} and
\code{string} are the same.  

Internally, the strings are encoded using UTF-16.  This means that in
some rare cases the number of Lisp characters in a string is not the
same as the number of codepoints in the string.


\section{External Formats}

To be able to communicate to the external world, \cmucl{} supports
external formats to convert to and from the external world to
\cmucl{}'s string format.  The external format is specified in several
ways.  The standard streams \var{*standard-input*},
\var{*standard-output*}, and \var{*standard-error*} take the format
from the value specified by \var{*default-external-format*}.  The
default value of \var{*default-external-format*} is \kwd{iso8859-1}.

For files, \code{OPEN} takes the \kwd{external-format}
parameter to specify the format.  The default external format is
\kwd{default}. 

\subsection{Available External Formats}

The available external formats are listed below in
Table~\ref{table:external-formats}.  The first column gives the
external format, and the second column gives a list of aliases that
can be used for this format.  The set of aliases can be changed by
changing the \file{aliases} file.

For all of these formats, if an illegal sequence is encountered, no
error or warning is signaled.  Instead, the offending sequence is
silently replaced with the Unicode REPLACEMENT CHARACTER (U$+$FFFD).

\begin{table}
  \centering
  \begin{tabular}{|l|l|p{3in}|}
    \hline
    \textbf{Format} & \textbf{Aliases} & \textbf{Description} \\
    \hline
    \hline
    \kwd{iso8859-1} & \kwd{latin1} \kwd{latin-1} \kwd{iso-8859-1} & ISO8859-1 \\
    \hline
    \kwd{iso8859-2} & \kwd{latin2} \kwd{latin-2} \kwd{iso-8859-2} & ISO8859-2 \\
    \hline
    \kwd{iso8859-3} & \kwd{latin3} \kwd{latin-3} \kwd{iso-8859-3} & ISO8859-3 \\
    \hline
    \kwd{iso8859-4} & \kwd{latin4} \kwd{latin-4} \kwd{iso-8859-4} & ISO8859-4 \\
    \hline
    \kwd{iso8859-5} & \kwd{cyrillic} \kwd{iso-8859-5} & ISO8859-5 \\
    \hline
    \kwd{iso8859-6} & \kwd{arabic} \kwd{iso-8859-6} & ISO8859-6 \\
    \hline
    \kwd{iso8859-7} & \kwd{greek} \kwd{iso-8859-7} & ISO8859-7 \\
    \hline
    \kwd{iso8859-8} & \kwd{hebrew} \kwd{iso-8859-8} & ISO8859-8 \\
    \hline
    \kwd{iso8859-9} & \kwd{latin5} \kwd{latin-5} \kwd{iso-8859-9} & ISO8859-9 \\
    \hline
    \kwd{iso8859-10} & \kwd{latin6} \kwd{latin-6} \kwd{iso-8859-10} & ISO8859-10 \\
    \hline
    \kwd{iso8859-13} & \kwd{latin7} \kwd{latin-7} \kwd{iso-8859-13} & ISO8859-13 \\
    \hline
    \kwd{iso8859-14} & \kwd{latin8} \kwd{latin-8} \kwd{iso-8859-14} & ISO8859-14 \\
    \hline
    \kwd{iso8859-15} & \kwd{latin9} \kwd{latin-9} \kwd{iso-8859-15} & ISO8859-15 \\
    \hline
    \kwd{utf-8} & \kwd{utf} \kwd{utf8} & UTF-8 \\
    \hline
    \kwd{utf-16} & \kwd{utf16} & UTF-16 with optional BOM \\
    \hline
    \kwd{utf-16-be} & \kwd{utf-16be} \kwd{utf16-be} & UTF-16 big-endian (without BOM) \\
    \hline
    \kwd{utf-16-le} & \kwd{utf-16le} \kwd{utf16-le} & UTF-16 little-endian (without BOM) \\
    \hline
    \kwd{utf-32} & \kwd{utf32} & UTF-32 with optional BOM \\
    \hline
    \kwd{utf-32-be} & \kwd{utf-32be} \kwd{utf32-be} & UTF-32 big-endian (without BOM) \\
    \hline
    \kwd{utf-32-le} & \kwd{utf-32le} \kwd{utf32-le} & UTF-32 little-endian (without BOM) \\
    \hline
    \kwd{cp1250} & & \\
    \hline
    \kwd{cp1251} & & \\
    \hline
    \kwd{cp1252} & \kwd{windows-1252} \kwd{windows-cp1252} \kwd{windows-latin1} & \\
    \hline
    \kwd{cp1253} & & \\
    \hline
    \kwd{cp1254} & & \\
    \hline
    \kwd{cp1255} & & \\
    \hline
    \kwd{cp1256} & & \\
    \hline
    \kwd{cp1257} & & \\
    \hline
    \kwd{cp1258} & & \\
    \hline
    \kwd{koi8-r} & & \\
    \hline
    \kwd{mac-cyrillic} & & \\
    \hline
    \kwd{mac-greek} & & \\
    \hline
    \kwd{mac-icelandic} & & \\
    \hline
    \kwd{mac-latin2} & & \\
    \hline
    \kwd{mac-roman} & & \\
    \hline
    \kwd{mac-turkish} & & \\
    \hline
  \end{tabular}
  \caption{External formats}
  \label{table:external-formats}
\end{table}

\subsection{Composing External Formats}

A composing external format is an external format that converts between
one codepoint and another, rather than between codepoints and octets.
A composing external format must be used in conjunction with another
(octet-producing) external format.  This is specified by
using a list as the external format.  For example, we can use
\code{'(\kwd{latin1} \kwd{crlf})} as the external format. In this
particular example, the external format is latin1, but whenever a
carriage-return/linefeed sequence is read, it is converted to the Lisp
\lispchar{Newline} character.  Conversely, whenever a string is written,
a Lisp \lispchar{Newline} character is converted to a
carriage-return/linefeed sequence.  Without the \kwd{crlf} composing
format, the carriage-return and linefeed will be read in as separate
characters, and on output the Lisp \lispchar{Newline} character is
output as a single linefeed character.

Table~\ref{table:composing-formats} lists the available composing formats.

\begin{table}
  \centering
  \begin{tabular}{|l|l|p{3in}|}
    \hline
    \textbf{Format} & \textbf{Aliases} & \textbf{Description} \\
    \hline
    \hline
    \kwd{crlf} & \kwd{dos} & Composing format for converting to/from DOS (CR/LF)
    end-of-line sequence to \lispchar{Newline}\\
    \kwd{cr} & \kwd{mac} & Composing format for converting to/from DOS (CR/LF)
    end-of-line sequence to \lispchar{Newline}\\
    \hline
    \kwd{beta-gk} & & Composing format that translates (lower-case) Beta
    code (an ASCII encoding of ancient Greek) \\
    \hline
    \kwd{final-sigma} & & Composing format that attempts to detect sigma in
    word-final position and change it from U+3C3 to U+3C2\\
    \hline
  \end{tabular}
  \caption{Composing external formats}
  \label{table:composing-formats}
\end{table}

\section{Dictionary}

\subsection{Variables}

\begin{defvar}{extensions:}{default-external-format}
   This is the default external format to use for all newly opened
   files.  It is also the default format to use for
   \var{*standard-input*}, \var{*standard-output*}, and
   \var{*standard-error*}.  The default value is \kwd{iso8859-1}.

   Setting this will cause the standard streams to start using the new
   format immediately.  If a stream has been created with external
   format \kwd{default}, then setting \var{*default-external-format*}
   will cause all subsequent input and output to use the new value of
   \var{*default-external-format*}.
\end{defvar}
\subsection{Characters}

Remember that \cmucl{}'s characters are only 16-bits long but Unicode
codepoints are up to 21 bits long.  Hence there are codepoints that
cannot be represented via Lisp characters.  Operating on individual
characters is not recommended.  Operations on strings are better.
(This would be true even if \cmucl{}'s characters could hold a
full Unicode codepoint.)

\begin{defun}{}{char-equal}{\amprest{} \var{characters}}
   \defunx{char-not-equal}{\amprest{} \var{characters}}
   \defunx{char-lessp}{\amprest{} \var{characters}}
   \defunx{char-greaterp}{\amprest{} \var{characters}}
   \defunx{char-not-greaterp}{\amprest{} \var{characters}}
   \defunx{char-not-lessp}{\amprest{} \var{characters}}
   For the comparison, the characters are converted to lowercase and
   the corresponding \code{char-code} are compared.
\end{defun}

\begin{defun}{}{alpha-char-p}{\args \var{character}}
  Returns non-nil{} if the Unicode category is a letter category.
\end{defun}

\begin{defun}{}{alphanumericp}{\args \var{character}}
  Returns non-nil{} if the Unicode category is a letter category or an ASCII
  digit.
\end{defun}

\begin{defun}{}{digit-char-p}{\args \var{character} \ampoptional{} \var{radix}}
   Only recognizes ASCII digits (and ASCII letters if the radix is larger
   than 10).
\end{defun}

\begin{defun}{}{graphic-char-p}{\args \var{character}}
  Returns non-nil{} if the Unicode category is a graphic category.
\end{defun}

\begin{defun}{}{upper-case-p}{\args \var{character}}
  \defunx{lower-case-p}{\args \var{character}}
  Returns non-nil{} if the Unicode category is an uppercase
  (lowercase) character.
\end{defun}

\begin{defun}{lisp:}{title-case-p}{\args \var{character}}
  Returns non-nil{} if the Unicode category is a titlecase character.
\end{defun}

\begin{defun}{}{both-case-p}{\args \var{character}}
  Returns non-nil{} if the Unicode category is an uppercase,
  lowercase, or titlecase character.
\end{defun}

\begin{defun}{}{char-upcase}{\args \var{character}}
  \defunx{char-downcase}{\args \var{character}}
  The Unicode uppercase (lowercase) letter is returned.
\end{defun}

\begin{defun}{lisp:}{char-titlecase}{\args \var{character}}
  The Unicode titlecase letter is returned.
\end{defun}

\begin{defun}{}{char-name}{\args \var{char}}
   If possible the name of the character \var{char} is returned.  If
   there is a Unicode name, the Unicode name is returned, except
   spaces are converted to underscores and the string is capitalized
   via \code{string-capitalize}.  If there is no Unicode name, the
   form \lispchar{U+xxxx} is returned where ``xxxx'' is the
   \code{char-code} of the character, in hexadecimal.
\end{defun}

\begin{defun}{}{name-char}{\args \var{name}}
  The inverse to \code{char-name}.  If no character has the name
  \var{name}, then \nil{} is returned.  Unicode names are not
  case-sensitive, and spaces and underscores are optional.
\end{defun}
\subsection{Strings}

Strings in \cmucl{} are UTF-16 strings.  That is, for Unicode code
points greater than 65535, surrogate pairs are used.  We refer the
reader to the Unicode standard for more information about surrogate
pairs.  We just want to make a note that because of the UTF-16
encoding of strings, there is a distinction between Lisp characters
and Unicode codepoints.  The standard string operations know about
this encoding and handle the surrogate pairs correctly.


\begin{defun}{}{string-upcase}{\args \var{string} \keys{\kwd{start}
      \kwd{end} \kwd{casing}}}
  \defunx{string-downcase}{\args \var{string} \keys{\kwd{start}
      \kwd{end} \kwd{casing}}}
  \defunx{string-capitalize}{\args \var{string} \keys{\kwd{start}
      \kwd{end} \kwd{casing}}}
  The case of the \var{string} is changed appropriately.  Surrogate
  pairs are handled correctly.  The conversion to the appropriate case
  is done based on the Unicode conversion.  The additional argument
  \kwd{casing} controls how case conversion is done.  The default
  value is \kwd{simple}, which uses simple Unicode case conversion.
  If \kwd{casing} is \kwd{full}, then full Unicode case conversion is
  done where the string may actually increase in length.
\end{defun}

\begin{defun}{}{nstring-upcase}{\args \var{string} \keys{\kwd{start} \kwd{end}}}
  \defunx{nstring-downcase}{\args \var{string} \keys{\kwd{start} \kwd{end}}}
  \defunx{nstring-capitalize}{\args \var{string} \keys{\kwd{start}
      \kwd{end}}}
  The case of the \var{string} is changed appropriately.  Surrogate
  pairs are handled correctly.  The conversion to the appropriate case
  is done based on the Unicode conversion.  (Full casing is not
  available because the string length cannot be increased when needed.)
\end{defun}

\begin{defun}{}{string=}{\args \var{s1} \var{s2} \keys{\kwd{start1}
      \kwd{end1} \kwd{start2} \kwd{end2}}}
  \defunx{string/=}{\args \var{s1} \var{s2} \keys{\kwd{start1} \kwd{end1} \kwd{start2} \kwd{end2}}}
  \defunx{string$<$}{\args \var{s1} \var{s2} \keys{\kwd{start1} \kwd{end1} \kwd{start2} \kwd{end2}}}
  \defunx{string$>$}{\args \var{s1} \var{s2} \keys{\kwd{start1} \kwd{end1} \kwd{start2} \kwd{end2}}}
  \defunx{string$<$=}{\args \var{s1} \var{s2} \keys{\kwd{start1} \kwd{end1} \kwd{start2} \kwd{end2}}}
  \defunx{string$>$=}{\args \var{s1} \var{s2} \keys{\kwd{start1} \kwd{end1} \kwd{start2} \kwd{end2}}}
  The string comparison is done in codepoint order.  (This is
  different from just comparing the order of the individual characters
  due to surrogate pairs.)  Unicode collation is not done.
\end{defun}

\begin{defun}{}{string-equal}{\args \var{s1} \var{s2} \keys{\kwd{start1}
      \kwd{end1} \kwd{start2} \kwd{end2}}}
  \defunx{string-not-equal}{\args \var{s1} \var{s2} \keys{\kwd{start1} \kwd{end1} \kwd{start2} \kwd{end2}}}
  \defunx{string-lessp}{\args \var{s1} \var{s2} \keys{\kwd{start1} \kwd{end1} \kwd{start2} \kwd{end2}}}
  \defunx{string-greaterp}{\args \var{s1} \var{s2} \keys{\kwd{start1} \kwd{end1} \kwd{start2} \kwd{end2}}}
  \defunx{string-not-greaterp}{\args \var{s1} \var{s2} \keys{\kwd{start1} \kwd{end1} \kwd{start2} \kwd{end2}}}
  \defunx{string-not-lessp}{\args \var{s1} \var{s2} \keys{\kwd{start1} \kwd{end1} \kwd{start2} \kwd{end2}}}
  Each codepoint in each string is converted to lowercase and the
  appropriate comparison of the codepoint values is done.  Unicode
  collation is not done.
\end{defun}

\begin{defun}{}{string-left-trim}{\args \var{bag} \var{string}}
  \defunx{string-right-trim}{\args \var{bag} \var{string}}
  \defunx{string-trim}{\args \var{bag} \var{string}}
  Removes any characters in \code{bag} from the left, right, or both
  ends of the string \code{string}, respectively.  This has potential
  problems if you want to remove a surrogate character from the
  string, since a single character cannot represent a surrogate.  As
  an extension, if \code{bag} is a string, we properly handle
  surrogate characters in the \code{bag}.
\end{defun}

\subsection{Sequences}

Since strings are also sequences, the sequence functions can be used
on strings.  We note here some issues with these functions.  Most
issues are due to the fact that strings are UTF-16 strings and
characters are UTF-16 code units, not Unicode codepoints.

\begin{defun}{}{remove-duplicates}{\args \var{sequence}
    \keys{\kwd{from-end} \kwd{test} \kwd{test-not} \kwd{start}
      \kwd{end} \kwd{key}}}
  \defunx{delete-duplicates}{\args \var{sequence}
    \keys{\kwd{from-end} \kwd{test} \kwd{test-not} \kwd{start}
      \kwd{end} \kwd{key}}}
  Because of surrogate pairs these functions may remove a high or low
  surrogate value, leaving the string in an invalid state.  Use these
  functions carefully with strings.
\end{defun}


\subsection{Reader}

To support Unicode characters, the reader has been extended to
recognize characters written in hexadecimal.  Thus \lispchar{U+41} is
the ASCII capital letter ``A'', since 41 is the hexadecimal code for
that letter.  The Unicode name of the character is also recognized,
except spaces in the name are replaced by underscores.

Recall, however, that characters in \cmucl{} are only 16 bits long so
many Unicode characters cannot be represented.  However, strings can
represent all Unicode characters.

When symbols are read, the symbol name is converted to Unicode NFC
form before interning the symbol into the package.  Hence,
\code{symbol-name (intern ``string'')} may produce a string that is
not \code{string=} to ``string''.  However, after conversion to NFC
form, the strings will be identical.

\subsection{Printer}

When printing characters, if the character is a graphic character, the
character is printed.  Thus \lispchar{U+41} is printed as
\lispchar{A}.  If the character is not a graphic character, the Lisp
name (e.g., \lispchar{Tab}) is used if possible;
if there is no Lisp name, the Unicode name is used.  If there is no
Unicode name, the hexadecimal char-code is
printed.  For example, \lispchar{U+34e}, which is not a graphic
character, is printed as \lispchar{Combining\_Upwards\_Arrow\_Below},
and \lispchar{U+9f} which is not a graphic character and does not have a
Unicode name, is printed as \lispchar{U+009F}.

\subsection{Miscellaneous}


\subsubsection{Files}

\cmucl{} loads external formats using the search-list
\file{ext-formats:}.  The \file{aliases} file is also located using
this search-list.

The Unicode data base is stored in compressed form in the file
\file{ext-formats:unidata.bin}.  If this file is not found, Unicode
support is severely reduced; you can only use ASCII characters.

\begin{defun}{}{open}{\args \var{filename} \amprest \var{options}
    \keys{\kwd{direction} \kwd{element-type} \kwd{if-exists}
      \kwd{if-does-not-exist} \morekeys \kwd{class} \kwd{mapped}
      \kwd{input-handle} \kwd{output-handle}
      \yetmorekeys \kwd{external-format} \kwd{decoding-error}
      \kwd{encoding-error}}}

    The main options are covered elsewhere.  Here we describe the
    options specific to Unicode.  The option \kwd{external-format}
    specifies the external format to use for reading and writing the
    file.  The external format is a keyword.

    The options \kwd{decoding-error} and \kwd{encoding-error} are used
    to specify how encoding and decoding errors are handled.  The
    default value on \nil means the external format handles errors
    itself and typically replaces invalid sequences with the Unicode
    replacement character.

    Otherwise, the value for \code{decoding-error} is either a
    character, a symbol or a function.  If a character is
    specified. it is used as the replacement character for any invalid
    decoding.  If a symbol or a function is given, it must be a
    function of three arguments: a message string to be printed, the
    offending octet, and the number of octets read.  If the function
    returns, it should return two values: the code point to use as the
    replacement character and the number of octets read.  In addition,
    \true{} may be specified.  This indicates that a continuable error
    is signaled, which, if continued, the Unicode replacement
    character is used.

    For \code{encoding-error}, a character, symbol, or function can be
    specified, like \code{decoding-error}, with the same meaning.  The
    function, however, takes two arguments:  a format message string
    and the incorrect codepoint.  If the function returns, it should
    be the replacement codepoint.
\end{defun}    
      
\subsubsection{Utilities}

\begin{defun}{stream:}{set-system-external-format}{\var{terminal} \ampoptional{} \var{filenames}}
  This function changes the external format used for
  \var{*standard-input*}, \var{*standard-output*}, and
  \var{*standard-error*} to the external format specified by
  \var{terminal}.  Additionally, the Unix file name encoding can be
  set to the value specified by \var{filenames} if non-\nil.
\end{defun}

\begin{defun}{extensions:}{list-all-external-formats}{}
  list all of the vailable external formats.  A list is returned where
  each element is a list of the external format name and a list of
  aliases for the format.  No distinction is made between external
  formats and composing external formats.
\end{defun}

\begin{defun}{extensions:}{describe-external-format}{external-format}
  Print a description of the given \var{external-format}.  This may
  cause the external format to be loaded (silently) if it is not
  already loaded.
\end{defun}

Since strings are UTF-16 and hence may contain surrogate pairs, some
utility functions are provided to make access easier.

\begin{defun}{lisp:}{codepoint}{\args \var{string} \var{i}
    \ampoptional{} \var{end}}
  Return the codepoint value from \var{string} at position \var{i}.
  If code unit at that position is a surrogate value, it is combined
  with either the previous or following code unit (when possible) to
  compute the codepoint.  The first return value is the codepoint
  itself.  The second return value is \nil{} if the position is not a
  surrogate pair.  Otherwise, $+1$ or $-1$ is returned if the position
  is the high (leading) or low (trailing) surrogate value, respectively.

  This is useful for iterating through a string in codepoint sequence.
\end{defun}

\begin{defun}{lisp:}{surrogates-to-codepoint}{\args \var{hi} \var{lo}}
  Convert the given \var{hi} and \var{lo} surrogate characters to the
  corresponding codepoint value
\end{defun}

\begin{defun}{lisp:}{surrogates}{\args \var{codepoint}}
  Convert the given \var{codepoint} value to the corresponding high
  and low surrogate characters.  If the codepoint is less than 65536,
  the second value is \nil{} since the codepoint does not need to be
  represented as a surrogate pair.
\end{defun}

\begin{defun}{stream:}{string-encode}{\args \var{string}
    \var{external-format} \ampoptional{} (\var{start} 0) \var{end}}
  \code{string-encode} encodes \var{string} using the format
  \var{external-format}, producing an array of octets.  Each octet is
  converted to a character via \code{code-char} and the resulting
  string is returned.

  The optional argument \var{start}, defaulting to 0, specifies the
  starting index and \var{end}, defaulting to the length of the
  string, is the end of the string.
\end{defun}

\begin{defun}{stream:}{string-decode}{\args \var{string}
    \var{external-format} \ampoptional{} (\var{start} 0) \var{end}}
  \code{string-decode} decodes \var{string} using the format
  \var{external-format} and produces a new string.  Each character of
  \var{string} is converted to octet (by \code{char-code}) and the
  resulting array of octets is used by the external format to produce
  a string.  This is the inverse of \code{string-encode}.

  The optional argument \var{start}, defaulting to 0, specifies the
  starting index and \var{end}, defaulting to the length of the
  string, is the end of the string.

  \var{string} must consist of characters whose \code{char-code} is
  less than 256.
\end{defun}

\begin{defun}{}{string-to-octets}{\args \var{string} \keys{\kwd{start}
      \kwd{end} \kwd{external-format} \kwd{buffer} \kwd{buffer-start}
      \kwd{error}}}
  \code{string-to-octets} converts \var{string} to a sequence of
  octets according to the external format specified by
  \var{external-format}.  The string to be converted is bounded by
  \var{start}, which defaults to 0, and \var{end}, which defaults to
  the length of the string.  If \var{buffer} is specified, the octets
  are placed in \var{buffer}.  If \var{buffer} is not specified, a new
  array is allocated to hold the octets.  \var{buffer-start} specifies
  where in the buffer the first octet will be placed.

  An error method may also be specified by \var{error}.  Any errors
  encountered while converting the string to octets will be handled
  according to error.  If \nil{}, a replacement character is converted
  to octets in place of the error.  Otherwise, \var{error} should be a
  symbol or function that will be called when the error occurs.  The
  function takes two arguments:  an error string and the character
  that caused the error.  It should return a replacement character.
  
  Three values are returned: The buffer, the number of valid octets
  written, and the number of characters converted.  Note that the
  actual number of octets written may be greater than the returned
  value, These represent the partial octets of the next character to
  be converted, but there was not enough room to hold the complete set
  of octets.
\end{defun}

\begin{defun}{}{octets-to-string}{\args \var{octets} \keys{\kwd{start}
      \kwd{end} \kwd{external-format} \kwd{string} \kwd{s-start}
      \kwd{s-end} \kwd{state}}}
  \code{octets-to-string} converts the sequence of octets in
  \var{octets} to a string.  \var{octets} must be a
  \code{(simple-array (unsigned-byte 8) (*))}.  The octets to be
  converted are bounded by \var{start} and \var{end}, which default to
  0 and the length of the array, respectively.  The conversion is
  performed according to the external format specified by
  \var{external-format}.  If \var{string} is specified, the octets are
  converted and stored in \var{string}, starting at \var{s-start}
  (defaulting to 0) and ending just before \var{s-end} (defaulting to
  the end of \var{string}.  \var{string} must be \code{simple-string}.
  If the bounded string is not large enough to hold all of the
  characters, then some octets will not be converted.  If \var{string}
  is not specified, a new string is created.

  The \var{state} is used as the initial state of for the external
  format.  This is useful when converting buffers of octets where the
  buffers are not on character boundaries, and state information is
  needed between buffers.

  Four values are returned: the string, the number of characters
  written to the string, and the number of octets consumed to produce
  the characters, and the final state of external format after
  converting the octets.
\end{defun}

\section{Writing External Formats}

\subsection{External Formats}
Users may write their own external formats.  It is probably easiest to
look at existing external formats to see how do this.

An external format basically needs two functions:
\code{octets-to-code} to convert octets to Unicode codepoints and
\code{code-to-octets} to convert Unicode codepoints to octets.  The
external format is defined using the macro
\code{stream::define-external-format}.

\begin{defmac}[base]{stream:}{define-external-format}{\args \var{name}
    (\keys{\var{base} \var{min} \var{max} \var{size}
      \var{documentation}})
    (\amprest{} \var{slots})
    \morekeys{\var{octets-to-code} \var{code-to-octets}
              \var{flush-state} \var{copy-state}}}


  If \kwd{base} is not given, this defines a new external format of
  the name \kwd{name}. \var{min}, \var{max}, and \var{size} are the
  minimum and maximum number of octets that make up a character.
  (\code{\kwd{size} n} is just a short cut for \code{\kwd{min} n
    \kwd{max} n}.)  The description of the external format can be
  given using \kwd{documentation}.  The arguments \var{octets-to-code}
  and \var{code-to-octets} are not optional in this case.  They
  specify how to convert octets to codepoints and vice versa,
  respectively.  These should be backquoted forms for the body of a
  function to do the conversion.  See the description below for these
  functions.  Some good examples are the external format for
  \kwd{utf-8} or \kwd{utf-16}.  The \kwd{slots} argument is a list of
  read-only slots, similar to defstruct.  The slot names are available
  as local variables inside the \var{code-to-octets} and
  \var{octets-to-code} bodies. 

  If \kwd{base} is given, then an external format is defined with the
  name \kwd{name} that is based on a previously defined format
  \kwd{base}. The \var{slots} are inherited from the \kwd{base} format
  by default, although the definition may alter their values and add
  new slots. See, for example, the \kwd{mac-greek} external format. 

\end{defmac}

\begin{defmac}{}{octets-to-code}{\args \var{state} \var{input}
    \var{unput} \var{error} \amprest{} \var{args}}
  This defines a form to be used by an external format to convert
  octets to a code point.  \var{state} is a form that can be used by
  the body to access the state variable of a stream.  This can be used
  for any reason to hold anything needed by \code{octets-to-code}.
  \var{input} is a form that returns one octet from the input stream.
  \var{unput} will put back \var{N} octets to the stream.  \var{args} is a
  list of variables that need to be defined for any symbols in the
  body of the macro.

  \var{error} controls how errors are handled.  If \nil, some suitable
  replacement character is used.  That is, any errors are silently
  ignored and replaced by some replacement character.  If non-\nil,
  \var{error} is a symbol or function that is called to handle the
  error.  This function takes three arguments: a message string, the
  invalid octet (or \nil), and a count of the number of octets that
  have been read so far.  If the function returns, it should be the
  codepoint of the desired replacement character.
\end{defmac}

\begin{defmac}{}{code-to-octets}{\args \var{code} \var{state}
    \var{output} \var{error} \amprest{} \var{args}}
  Defines a form to be used by the external format to convert a code
  point to octets for output.  \var{code} is the code point to be
  converted.  \var{state} is a form to access the current value of the
  stream's state variable.  \var{output} is a form that writes one
  octet to the output stream.

  Similar to \code{octets-to-code}, \var{error} indicates how errors
  should be handled.  If \nil, some default replacement character is
  substituted.  If non-\nil, \var{error} should be a symbol or
  function.   This function takes two arguments:  a message string and
  the invalid codepoint.  If the function returns, it should be the
  codepoint that will be substituted for the invalid codepoint.
\end{defmac}

\begin{defmac}{}{flush-state}{\args \var{state}
    \var{output} \var{error} \amprest{} \var{args}}
  Defines a form to be used by the external format to flush out
  any state when an output stream is closed.  Similar to
  \code{code-to-octets}, but there is no code point to be output.  The
  \var{error} argument indicates how to handle errors.  If \nil, some
  default replacement character is used.  Otherwise, \var{error} is a
  symbol or function that will be called with a message string and
  codepoint of the offending state.  If the function returns, it
  should be the codepoint of a suitable replacement.

  If \code{flush-state} is \false, then nothing special is needed to
  flush the state to the output.

  This is called only when an output character stream is being closed.
\end{defmac}

\begin{defmac}{}{copy-state}{\args \var{state} \amprest{} \var{args}}
  Defines a form to copy any state needed by the external format.
  This should probably be a deep copy so that if the original
  state is modified, the copy is not.

  If not given, then nothing special is needed to copy the state
  either because there is no state for the external format or that no
  special copier is needed.
\end{defmac}

\subsection{Composing External Formats}

\begin{defmac}{stream:}{define-composing-external-format}{\args \var{name}
    (\keys{\var{min} \var{max} \var{size} \var{documentation}}) \var{input}
    \var{output}}
  This is the same as \code{define-external-format}, except that a
  composing external format is created.
\end{defmac}


\twocolumn
\cindex{Function Index}
\printindex[funs]

\twocolumn
\cindex{Variable Index}
\printindex[vars]

\twocolumn
\cindex{Type Index}
\printindex[types]

\onecolumn
\cindex{Concept Index}
\printindex[concept]

\end{document}
