\chapter{UNIX Interface}
\label{unix-interface}

\credits{by Robert MacLachlan, Skef Wholey, Bill Chiles and William Lott}


\cmucl{} attempts to make the full power of the underlying
environment available to the Lisp programmer. This is done using
combination of hand-coded interfaces and foreign function calls to C
libraries. Although the techniques differ, the style of interface is
similar. This chapter provides an overview of the facilities available
and general rules for using them, as well as describing specific
features in detail. It is assumed that the reader has a working
familiarity with Unix and X11, as well as access to the standard
system documentation.


\section{Reading the Command Line}

The shell parses the command line with which Lisp is invoked, and
passes a data structure containing the parsed information to Lisp.
This information is then extracted from that data structure and put
into a set of Lisp data structures.

\begin{defvar}{extensions:}{command-line-strings}
  \defvarx[extensions:]{command-line-utility-name}
  \defvarx[extensions:]{command-line-words}
  \defvarx[extensions:]{command-line-switches}
  
  The value of \code{*command-line-words*} is a list of strings that
  make up the command line, one word per string.  The first word on
  the command line, i.e.  the name of the program invoked (usually
  \code{lisp}) is stored in \code{*command-line-utility-name*}.  The
  value of \code{*command-line-switches*} is a list of
  \code{command-line-switch} structures, with a structure for each
  word on the command line starting with a hyphen.  All the command
  line words between the program name and the first switch are stored
  in \code{*command-line-words*}.
\end{defvar}

The following functions may be used to examine \code{command-line-switch}
structures.
\begin{defun}{extensions:}{cmd-switch-name}{\args{\var{switch}}}
  
  Returns the name of the switch, less the preceding hyphen and
  trailing equal sign (if any).
\end{defun}
\begin{defun}{extensions:}{cmd-switch-value}{\args{\var{switch}}}
  
  Returns the value designated using an embedded equal sign, if any.
  If the switch has no equal sign, then this is null.
\end{defun}
\begin{defun}{extensions:}{cmd-switch-words}{\args{\var{switch}}}
  
  Returns a list of the words between this switch and the next switch
  or the end of the command line.
\end{defun}
\begin{defun}{extensions:}{cmd-switch-arg}{\args{\var{switch}}}
  
  Returns the first non-null value from \code{cmd-switch-value}, the
  first element in \code{cmd-switch-words}, or the first word in
  \var{command-line-words}.
\end{defun}

\begin{defun}{extensions:}{get-command-line-switch}{\args{\var{sname}}}
  
  This function takes the name of a switch as a string and returns the
  value of the switch given on the command line.  If no value was
  specified, then any following words are returned.  If there are no
  following words, then \true{} is returned.  If the switch was not
  specified, then \false{} is returned.
\end{defun}

\begin{defmac}{extensions:}{defswitch}{%
    \args{\var{name} \ampoptional{} \var{function}}}
  
  This macro causes \var{function} to be called when the switch
  \var{name} appears in the command line.  Name is a simple-string
  that does not begin with a hyphen (unless the switch name really
  does begin with one.)
  
  If \var{function} is not supplied, then the switch is parsed into
  \var{command-line-switches}, but otherwise ignored.  This suppresses
  the undefined switch warning which would otherwise take place.  The
  warning can also be globally suppressed by
  \var{complain-about-illegal-switches}.
\end{defmac}


\section{Useful Variables}

\begin{defvar}{system:}{stdin}
  \defvarx[system:]{stdout} \defvarx[system:]{stderr}
  
  Streams connected to the standard input, output and error file
  descriptors.
\end{defvar}

\begin{defvar}{system:}{tty}
  
  A stream connected to \file{/dev/tty}.
\end{defvar}

\begin{defvar}{extensions:}{environment-list}
  The environment variables inherited by the current process, as a
  keyword-indexed alist. For example, to access the DISPLAY
  environment variable, you could use

\begin{lisp}
   (cdr (assoc :display ext:*environment-list*))
\end{lisp}

  Note that the case of the variable name is preserved when converting
  to a keyword.  Therefore, you need to specify the keyword properly for
  variable names containing lower-case letters,
\end{defvar}


\section{Lisp Equivalents for C Routines}

The UNIX documentation describes the system interface in terms of C
procedure headers.  The corresponding Lisp function will have a somewhat
different interface, since Lisp argument passing conventions and
datatypes are different.

The main difference in the argument passing conventions is that Lisp does not
support passing values by reference.  In Lisp, all argument and results are
passed by value.  Interface functions take some fixed number of arguments and
return some fixed number of values.  A given ``parameter'' in the C
specification will appear as an argument, return value, or both, depending on
whether it is an In parameter, Out parameter, or In/Out parameter.  The basic
transformation one makes to come up with the Lisp equivalent of a C routine is
to remove the Out parameters from the call, and treat them as extra return
values.  In/Out parameters appear both as arguments and return values.  Since
Out and In/Out parameters are only conventions in C, you must determine the
usage from the documentation.

Thus, the C routine declared as

\begin{example}
kern_return_t lookup(servport, portsname, portsid)
        port        servport;
        char        *portsname;
        int        *portsid;        /* out */
 {
  ...
  *portsid = <expression to compute portsid field>
  return(KERN_SUCCESS);
 }
\end{example}

has as its Lisp equivalent something like

\begin{lisp}
(defun lookup (ServPort PortsName)
  ...
  (values
   success
   <expression to compute portsid field>))
\end{lisp}

If there are multiple out or in-out arguments, then there are multiple
additional returns values.

Fortunately, \cmucl{} programmers rarely have to worry about the
nuances of this translation process, since the names of the arguments and
return values are documented in a way so that the \code{describe} function
(and the \hemlock{} \code{Describe Function Call} command, invoked with
\b{C-M-Shift-A}) will list this information.  Since the names of arguments
and return values are usually descriptive, the information that
\code{describe} prints is usually all one needs to write a
call. Most programmers use this on-line documentation nearly
all of the time, and thereby avoid the need to handle bulky
manuals and perform the translation from barbarous tongues.


\section{Type Translations}
\cindex{aliens}
\cpsubindex{types}{alien}
\cpsubindex{types}{foreign language}

Lisp data types have very different representations from those used by
conventional languages such as C.  Since the system interfaces are
designed for conventional languages, Lisp must translate objects to and
from the Lisp representations.  Many simple objects have a direct
translation: integers, characters, strings and floating point numbers
are translated to the corresponding Lisp object.  A number of types,
however, are implemented differently in Lisp for reasons of clarity and
efficiency.

Instances of enumerated types are expressed as keywords in Lisp.
Records, arrays, and pointer types are implemented with the \alien{}
facility (\pxlref{aliens}).  Access functions are defined
for these types which convert fields of records, elements of arrays,
or data referenced by pointers into Lisp objects (possibly another
object to be referenced with another access function).

One should dispose of \alien{} objects created by constructor
functions or returned from remote procedure calls when they are no
longer of any use, freeing the virtual memory associated with that
object.  Since \alien{}s contain pointers to non-Lisp data, the
garbage collector cannot do this itself.  If the memory
was obtained from \funref{make-alien} or from a foreign function call
to a routine that used \code{malloc}, then \funref{free-alien} should
be used.


\section{System Area Pointers}
\label{system-area-pointers}

\cindex{pointers}\cpsubindex{malloc}{C function}\cpsubindex{free}{C function}
Note that in some cases an address is represented by a Lisp integer, and in
other cases it is represented by a real pointer.  Pointers are usually used
when an object in the current address space is being referred to.  The MACH
virtual memory manipulation calls must use integers, since in principle the
address could be in any process, and Lisp cannot abide random pointers.
Because these types are represented differently in Lisp, one must explicitly
coerce between these representations.

System Area Pointers (SAPs) provide a mechanism that bypasses the
\alien{} type system and accesses virtual memory directly.  A SAP is a
raw byte pointer into the \code{lisp} process address space.  SAPs are
represented with a pointer descriptor, so SAP creation can cause
consing.  However, the compiler uses a non-descriptor representation
for SAPs when possible, so the consing overhead is generally minimal.
\xlref{non-descriptor}.

\begin{defun}{system:}{sap-int}{\args{\var{sap}}}
  \defunx[system:]{int-sap}{\args{\var{int}}}
  
  The function \code{sap-int} is used to generate an integer
  corresponding to the system area pointer, suitable for passing to
  the kernel interfaces (which want all addresses specified as
  integers).  The function \code{int-sap} is used to do the opposite
  conversion.  The integer representation of a SAP is the byte offset
  of the SAP from the start of the address space.
\end{defun}

\begin{defun}{system:}{sap+}{\args{\var{sap} \var{offset}}}
  
  This function adds a byte \var{offset} to \var{sap}, returning a new
  SAP.
\end{defun}

\begin{defun}{system:}{sap-ref-8}{\args{\var{sap} \var{offset}}}
  \defunx[system:]{sap-ref-16}{\args{\var{sap} \var{offset}}}
  \defunx[system:]{sap-ref-32}{\args{\var{sap} \var{offset}}
  \defunx[system:]{sap-ref-64}{\args{\var{sap} \var{offset}}}

  These functions return the 8, 16, 32 or 64 bit unsigned integer at
  \var{offset} from \var{sap}.  The \var{offset} is always a byte
  offset, regardless of the number of bits accessed.  \code{setf} may
  be used with the these functions to deposit values into virtual
  memory.
\end{defun}

\begin{defun}{system:}{signed-sap-ref-8}{\args{\var{sap} \var{offset}}}
  \defunx[system:]{signed-sap-ref-16}{\args{\var{sap} \var{offset}}}
  \defunx[system:]{signed-sap-ref-32}{\args{\var{sap} \var{offset}}}
  \defunx[system:]{signed-sap-ref-64}{\args{\var{sap} \var{offset}}}

  These functions are the same as the above unsigned operations,
  except that they sign-extend, returning a negative number if the
  high bit is set.
\end{defun}


\section{Unix System Calls}

You probably won't have much cause to use them, but all the Unix system
calls are available.  The Unix system call functions are in the
\code{Unix} package.  The name of the interface for a particular system
call is the name of the system call prepended with \code{unix-}.  The
system usually defines the associated constants without any prefix name.
To find out how to use a particular system call, try using
\code{describe} on it.  If that is unhelpful, look at the source in
\file{unix.lisp} or consult your system maintainer.

The Unix system calls indicate an error by returning \false{} as the
first value and the Unix error number as the second value.  If the call
succeeds, then the first value will always be non-\nil, often \code{t}.

For example, to use the \code{chdir} syscall: 

\begin{lisp}
(multiple-value-bind (success errno)
    (unix:unix-chdir "/tmp")
  (unless success
     (error "Can't change working directory: ~a"
            (unix:get-unix-error-msg errno))))
\end{lisp}

\begin{defun}{Unix:}{get-unix-error-msg}{\args{\var{error}}}

  This function returns a string describing the Unix error number
  \var{error} (this is similar to the Unix function \code{perror}). 
\end{defun}


\section{File Descriptor Streams}
\label{sec:fds}

Many of the UNIX system calls return file descriptors.  Instead of using other
UNIX system calls to perform I/O on them, you can create a stream around them.
For this purpose, fd-streams exist.  See also \funref{read-n-bytes}.

\begin{defun}{system:}{make-fd-stream}{%
    \args{\var{descriptor}} \keys{\kwd{input} \kwd{output}
      \kwd{element-type}} \morekeys{\kwd{buffering} \kwd{name}
      \kwd{file} \kwd{original}} \yetmorekeys{\kwd{delete-original}
      \kwd{auto-close}} \yetmorekeys{\kwd{timeout} \kwd{pathname}}}
  
  This function creates a file descriptor stream using
  \var{descriptor}.  If \kwd{input} is non-\nil, input operations are
  allowed.  If \kwd{output} is non-\nil, output operations are
  allowed.  The default is input only.  These keywords are defined:
  \begin{Lentry}
  \item[\kwd{element-type}] is the type of the unit of transaction for
    the stream, which defaults to \code{string-char}.  See the \clisp{}
    description of \code{open} for valid values.
  
  \item[\kwd{buffering}] is the kind of output buffering desired for
    the stream.  Legal values are \kwd{none} for no buffering,
    \kwd{line} for buffering up to each newline, and \kwd{full} for
    full buffering.
  
  \item[\kwd{name}] is a simple-string name to use for descriptive
    purposes when the system prints an fd-stream.  When printing
    fd-streams, the system prepends the streams name with \code{Stream
      for }.  If \var{name} is unspecified, it defaults to a string
    containing \var{file} or \var{descriptor}, in order of preference.
  
  \item[\kwd{file}, \kwd{original}] \var{file} specifies the defaulted
    namestring of the associated file when creating a file stream
    (must be a \code{simple-string}). \var{original} is the
    \code{simple-string} name of a backup file containing the original
    contents of \var{file} while writing \var{file}.
  
    When you abort the stream by passing \true{} to \code{close} as
    the second argument, if you supplied both \var{file} and
    \var{original}, \code{close} will rename the \var{original} name
    to the \var{file} name.  When you \code{close} the stream
    normally, if you supplied \var{original}, and
    \var{delete-original} is non-\nil, \code{close} deletes
    \var{original}.  If \var{auto-close} is true (the default), then
    \var{descriptor} will be closed when the stream is garbage
    collected.
  
  \item[\kwd{pathname}]: The original pathname passed to open and
    returned by \code{pathname}; not defaulted or translated.
  
  \item[\kwd{timeout}] if non-null, then \var{timeout} is an integer
    number of seconds after which an input wait should time out.  If a
    read does time out, then the \code{system:io-timeout} condition is
    signalled.
  \end{Lentry}
\end{defun}

\begin{defun}{system:}{fd-stream-p}{\args{\var{object}}}
  
  This function returns \true{} if \var{object} is an fd-stream, and
  \nil{} if not.  Obsolete: use the portable \code{(typep x
    'file-stream)}.
\end{defun}

\begin{defun}{system:}{fd-stream-fd}{\args{\var{stream}}}
  
  This returns the file descriptor associated with \var{stream}.
\end{defun}


\section{Unix Signals}
\cindex{unix signals} \cindex{signals}

\cmucl{} allows access to all the Unix signals that can be generated
under Unix.  It should be noted that if this capability is abused, it is
possible to completely destroy the running Lisp.  The following macros and
functions allow access to the Unix interrupt system.  The signal names as
specified in section 2 of the {\em Unix Programmer's Manual} are exported
from the Unix package.

\subsection{Changing Signal Handlers}
\label{signal-handlers}

\begin{defmac}{system:}{with-enabled-interrupts}{
    \args{\var{specs} \amprest{} \var{body}}}
  
  This macro should be called with a list of signal specifications,
  \var{specs}.  Each element of \var{specs} should be a list of
  two\hide{ or three} elements: the first should be the Unix signal
  for which a handler should be established, the second should be a
  function to be called when the signal is received\hide{, and the
    third should be an optional character used to generate the signal
    from the keyboard.  This last item is only useful for the SIGINT,
    SIGQUIT, and SIGTSTP signals.}  One or more signal handlers can be
  established in this way.  \code{with-enabled-interrupts} establishes
  the correct signal handlers and then executes the forms in
  \var{body}.  The forms are executed in an unwind-protect so that the
  state of the signal handlers will be restored to what it was before
  the \code{with-enabled-interrupts} was entered.  A signal handler
  function specified as NIL will set the Unix signal handler to the
  default which is normally either to ignore the signal or to cause a
  core dump depending on the particular signal.
\end{defmac}

\begin{defmac}{system:}{without-interrupts}{\args{\amprest{} \var{body}}}
  
  It is sometimes necessary to execute a piece a code that can not be
  interrupted.  This macro the forms in \var{body} with interrupts
  disabled.  Note that the Unix interrupts are not actually disabled,
  rather they are queued until after \var{body} has finished
  executing.
\end{defmac}

\begin{defmac}{system:}{with-interrupts}{\args{\amprest{} \var{body}}}
  
  When executing an interrupt handler, the system disables interrupts,
  as if the handler was wrapped in in a \code{without-interrupts}.
  The macro \code{with-interrupts} can be used to enable interrupts
  while the forms in \var{body} are evaluated.  This is useful if
  \var{body} is going to enter a break loop or do some long
  computation that might need to be interrupted.
\end{defmac}

\begin{defmac}{system:}{without-hemlock}{\args{\amprest{} \var{body}}}
  
  For some interrupts, such as SIGTSTP (suspend the Lisp process and
  return to the Unix shell) it is necessary to leave Hemlock and then
  return to it.  This macro executes the forms in \var{body} after
  exiting Hemlock.  When \var{body} has been executed, control is
  returned to Hemlock.
\end{defmac}

\begin{defun}{system:}{enable-interrupt}{%
    \args{\var{signal} \var{function}\hide{ \ampoptional{}
        \var{character}}}}
  
  This function establishes \var{function} as the handler for
  \var{signal}.
  \hide{The optional \var{character} can be specified
    for the SIGINT, SIGQUIT, and SIGTSTP signals and causes that
    character to generate the appropriate signal from the keyboard.}
  Unless you want to establish a global signal handler, you should use
  the macro \code{with-enabled-interrupts} to temporarily establish a
  signal handler.  \hide{Without \var{character},}
  \code{enable-interrupt} returns the old function associated with the
  signal.  \hide{When \var{character} is specified for SIGINT,
    SIGQUIT, or SIGTSTP, it returns the old character code.}
\end{defun}

\begin{defun}{system:}{ignore-interrupt}{\args{\var{signal}}}
  
  Ignore-interrupt sets the Unix signal mechanism to ignore
  \var{signal} which means that the Lisp process will never see the
  signal.  Ignore-interrupt returns the old function associated with
  the signal or \false{} if none is currently defined.
\end{defun}

\begin{defun}{system:}{default-interrupt}{\args{\var{signal}}}
  
  Default-interrupt can be used to tell the Unix signal mechanism to
  perform the default action for \var{signal}.  For details on what
  the default action for a signal is, see section 2 of the {\em Unix
    Programmer's Manual}.  In general, it is likely to ignore the
  signal or to cause a core dump.
\end{defun}


\subsection{Examples of Signal Handlers}

The following code is the signal handler used by the Lisp system for the
SIGINT signal.

\begin{lisp}
(defun ih-sigint (signal code scp)
  (declare (ignore signal code scp))
  (without-hemlock
   (with-interrupts
    (break "Software Interrupt" t))))
\end{lisp}

The \code{without-hemlock} form is used to make sure that Hemlock is exited before
a break loop is entered.  The \code{with-interrupts} form is used to enable
interrupts because the user may want to generate an interrupt while in the
break loop.  Finally, break is called to enter a break loop, so the user
can look at the current state of the computation.  If the user proceeds
from the break loop, the computation will be restarted from where it was
interrupted.

The following function is the Lisp signal handler for the SIGTSTP signal
which suspends a process and returns to the Unix shell.

\begin{lisp}
(defun ih-sigtstp (signal code scp)
  (declare (ignore signal code scp))
  (without-hemlock
   (Unix:unix-kill (Unix:unix-getpid) Unix:sigstop)))
\end{lisp}

Lisp uses this interrupt handler to catch the SIGTSTP signal because it is
necessary to get out of Hemlock in a clean way before returning to the shell.

To set up these interrupt handlers, the following is recommended:

\begin{lisp}
(with-enabled-interrupts ((Unix:SIGINT #'ih-sigint)
                          (Unix:SIGTSTP #'ih-sigtstp))
  <user code to execute with the above signal handlers enabled.>
)
\end{lisp}
